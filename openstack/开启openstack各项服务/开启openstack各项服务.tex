% !TeX spellcheck = en_US
%% 字体:方正静蕾简体
%%		 方正粗宋
\documentclass[a4paper,left=1.5cm,right=1.5cm,11pt]{article}

\usepackage[utf8]{inputenc}
\usepackage{fontspec}
\usepackage{cite}
\usepackage{xeCJK}
\usepackage{indentfirst}
\usepackage{titlesec}
\usepackage{etoolbox}%
\makeatletter
\patchcmd{\ttlh@hang}{\parindent\z@}{\parindent\z@\leavevmode}{}{}%
\patchcmd{\ttlh@hang}{\noindent}{}{}{}%
\makeatother
\usepackage{hyperref}
\usepackage{longtable}
\usepackage{empheq}
\usepackage{graphicx}
\usepackage{float}
\usepackage{rotating}
\usepackage{subfigure}
\usepackage{tabu}
\usepackage{amsmath}
\usepackage{setspace}
\usepackage{amsfonts}
\usepackage{appendix}
\usepackage{listings}
\usepackage{xcolor}
\usepackage{geometry}
\setcounter{secnumdepth}{4}
%\titleformat*{\section}{\LARGE}
%\renewcommand\refname{参考文献}
%\titleformat{\chapter}{\centering\bfseries\huge}{}{0.7em}{}{}
\titleformat{\section}{\LARGE\bf}{\thesection}{1em}{}{}
\titleformat{\subsection}{\Large\bfseries}{\thesubsection}{1em}{}{}
\titleformat{\subsubsection}{\large\bfseries}{\thesubsubsection}{1em}{}{}
\renewcommand{\contentsname}{{ \centerline{目{  } 录}}}
\setCJKfamilyfont{cjkhwxk}{STXINGKA.TTF}
%\setCJKfamilyfont{cjkhwxk}{华文行楷}
%\setCJKfamilyfont{cjkfzcs}{方正粗宋简体}
%\newcommand*{\cjkfzcs}{\CJKfamily{cjkfzcs}}
\newcommand*{\cjkhwxk}{\CJKfamily{cjkhwxk}}
%\newfontfamily\wryh{Microsoft YaHei}
%\newfontfamily\hwzs{华文中宋}
%\newfontfamily\hwst{华文宋体}
%\newfontfamily\hwfs{华文仿宋}
%\newfontfamily\jljt{方正静蕾简体}
%\newfontfamily\hwxk{华文行楷}
\newcommand{\verylarge}{\fontsize{60pt}{\baselineskip}\selectfont}  
\newcommand{\chuhao}{\fontsize{44.9pt}{\baselineskip}\selectfont}  
\newcommand{\xiaochu}{\fontsize{38.5pt}{\baselineskip}\selectfont}  
\newcommand{\yihao}{\fontsize{27.8pt}{\baselineskip}\selectfont}  
\newcommand{\xiaoyi}{\fontsize{25.7pt}{\baselineskip}\selectfont}  
\newcommand{\erhao}{\fontsize{23.5pt}{\baselineskip}\selectfont}  
\newcommand{\xiaoerhao}{\fontsize{19.3pt}{\baselineskip}\selectfont} 
\newcommand{\sihao}{\fontsize{14pt}{\baselineskip}\selectfont}      % 字号设置  
\newcommand{\xiaosihao}{\fontsize{12pt}{\baselineskip}\selectfont}  % 字号设置  
\newcommand{\wuhao}{\fontsize{10.5pt}{\baselineskip}\selectfont}    % 字号设置  
\newcommand{\xiaowuhao}{\fontsize{9pt}{\baselineskip}\selectfont}   % 字号设置  
\newcommand{\liuhao}{\fontsize{7.875pt}{\baselineskip}\selectfont}  % 字号设置  
\newcommand{\qihao}{\fontsize{5.25pt}{\baselineskip}\selectfont}    % 字号设置 

\usepackage{diagbox}
\usepackage{multirow}
\boldmath
\XeTeXlinebreaklocale "zh"
\XeTeXlinebreakskip = 0pt plus 1pt minus 0.1pt
\definecolor{cred}{rgb}{0.8,0.8,0.8}
\definecolor{cgreen}{rgb}{0,0.3,0}
\definecolor{cpurple}{rgb}{0.5,0,0.35}
\definecolor{cdocblue}{rgb}{0,0,0.3}
\definecolor{cdark}{rgb}{0.95,1.0,1.0}
\lstset{
	language=python,
	numbers=left,
	numberstyle=\tiny\color{black},
	showspaces=false,
	showstringspaces=false,
	basicstyle=\scriptsize,
	keywordstyle=\color{purple},
	commentstyle=\color{cgreen},
	stringstyle=\color{blue},
	frame=lines,
	% escapeinside=``,
	extendedchars=true, 
	xleftmargin=1em,
	xrightmargin=1em, 
	backgroundcolor=\color{cred},
	aboveskip=1em,
	breaklines=true,
	tabsize=4
} 

%\newfontfamily{\consolas}{Consolas}
%\newfontfamily{\monaco}{Monaco}
%\setmonofont[Mapping={}]{Consolas}	%英文引号之类的正常显示,相当于设置英文字体
%\setsansfont{Consolas} %设置英文字体 Monaco, Consolas,  Fantasque Sans Mono
%\setmainfont{Times New Roman}
%\setCJKmainfont{STZHONGS.TTF}
%\setmonofont{Consolas}
% \newfontfamily{\consolas}{YaHeiConsolas.ttf}
\newfontfamily{\monaco}{MONACO.TTF}
\setCJKmainfont{STZHONGS.TTF}
%\setmainfont{MONACO.TTF}
%\setsansfont{MONACO.TTF}

\newcommand{\fic}[1]{\begin{figure}[H]
		\center
		\includegraphics[width=0.8\textwidth]{#1}
	\end{figure}}
	
\newcommand{\sizedfic}[2]{\begin{figure}[H]
		\center
		\includegraphics[width=#1\textwidth]{#2}
	\end{figure}}

\newcommand{\codefile}[1]{\lstinputlisting{#1}}

\newcommand{\interval}{\vspace{0.5em}}

\newcommand{\tablestart}{
	\interval
	\begin{longtable}{p{2cm}p{10cm}}
	\hline}
\newcommand{\tableend}{
	\hline
	\end{longtable}
	\interval}

% 改变段间隔
\setlength{\parskip}{0.2em}
\linespread{1.1}

\usepackage{lastpage}
\usepackage{fancyhdr}
\pagestyle{fancy}
\lhead{\space \qquad \space}
\chead{openstack的常用操作 \qquad}
\rhead{\qquad\thepage/\pageref{LastPage}}

\begin{document}

\tableofcontents

\clearpage

\section{完整的服务}
	包含ironic完整的服务如下图所示:
	\fic{1.png}

\section{设置环境变量}
	\begin{lstlisting}
	export HOST_IP=$(ifconfig -a | grep -C 1 eth0 | grep "inet addr" | cut -d":" -f2 | cut -d" " -f1)
	export OS_AUTH_URL=http://$HOST_IP:5000/v3
	export OS_PROJECT_DOMAIN_ID=default         
	export OS_PROJECT_NAME=admin     
	export OS_USER_DOMAIN_ID=default
	export OS_IDENTITY_API_VERSION=3
	export OS_REGION_NAME=RegionOne
	export OS_USERNAME=admin
	export OS_PASSWORD=p1111111        
	export OS_TENANT_NAME=admin     
	export OS_VOLUME_API_VERSION=2
	\end{lstlisting}

\section{打开各个网桥}
	\begin{lstlisting}
	sudo ip link set br-ex up
	sudo ip link set br-int up
	sudo ip link set br-tun up
	sudo ip link set brbm up
	\end{lstlisting}

\section{配置screen}
	\begin{lstlisting}
	# 配置screen
	# =============
	
	SCREEN_NAME="stack"
	# 建立SCREEN_NAME的screen,创建一个叫shell的窗口,窗口中所要执行的shell为/bin/bash,同时detach这个视窗
	screen -d -m -S $SCREEN_NAME -t shell -s /bin/bash
    sleep 1

    # Set a reasonable status bar
    SCREEN_HARDSTATUS=${SCREEN_HARDSTATUS:-}
    if [ -z "$SCREEN_HARDSTATUS" ]; then
        SCREEN_HARDSTATUS='%{= .} %-Lw%{= .}%> %n%f %t*%{= .}%+Lw%< %-=%{g}(%{d}%H/%l%{g})'
    fi
	# 恢复离线的screen作业stack,并且设置状态栏的样式
    screen -r $SCREEN_NAME -X hardstatus alwayslastline "$SCREEN_HARDSTATUS"
	# 将PROMPT_COMMAND的值设为“/bin/true”
    screen -r $SCREEN_NAME -X setenv PROMPT_COMMAND /bin/true

	# 删除screenrc文件
	rm -f /home/pengsida/devstack/stack-screenrc

	# Initialize the directory for service status check
	SCREEN_NAME=${SCREEN_NAME:-stack}
    SERVICE_DIR=${SERVICE_DIR:-"/opt/stack/status"}
    if [[ ! -d "$SERVICE_DIR/$SCREEN_NAME" ]]; then
        mkdir -p "$SERVICE_DIR/$SCREEN_NAME"
    fi
    rm -f "$SERVICE_DIR/$SCREEN_NAME"/*.failure
	\end{lstlisting}

\section{开启dstat}
	命令如下:
	\begin{lstlisting}
	# 配置dstat的screen窗口
	# ================

	# 指定stack这个screen,在其中创建一个叫dstat的窗口
	screen -S stack -X screen -t dstat
	# 指定dstat窗口的logfile
	screen -S stack -p dstat -X logfile /home/pengsida/LOGFILE/dstat.log
	screen -S stack -p dstat -X log on
	# 创建dstat.log这个文件
	touch /home/pengsida/LOGFILE/dstat.log
	# 为dstat.log这个文件创建一个符号链接dstat.log
	# bash -c 'cd '\''/home/pengsida/LOGFILE'\'' && ln -sf '\''dstat.log'\'' dstat.log'
	# 在dstat窗口中执行这条命令
	screen -S stack -p dstat -X stuff 'sudo /home/pengsida/dstat.sh /home/pengsida/LOGFILE/logs & sudo echo $! >/opt/stack/status/stack/dstat.pid; fg || echo "dstat failed to start. Exit code: $?" | tee "/opt/stack/status/stack/dstat.failure"^M'
	\end{lstlisting}

\section{配置环境变量}
	\begin{lstlisting}
	HOST_IP="192.168.122.65"
	export OS_IDENTITY_API_VERSION=3
    export OS_AUTH_URL=http://$HOST_IP/identity_admin
    export OS_USERNAME=admin
    export OS_USER_DOMAIN_ID=default
    export OS_PASSWORD=p1111111
    export OS_PROJECT_NAME=admin
    export OS_PROJECT_DOMAIN_ID=default
    export OS_REGION_NAME=RegionOne
	\end{lstlisting}

\section{开启keystone服务}
	命令如下:
	\begin{lstlisting}
	# 开启keystone服务
	# ================

	# 新建key窗口
	screen -S stack -X screen -t key
	# 设置key窗口的log file
	screen -S stack -p key -X logfile /home/pengsida/LOGFILE/logs/key.log
	screen -S stack -p key -X log on
	touch /home/pengsida/LOGFILE/logs/key.log
	# bash -c 'cd '\''/home/pengsida/LOGFILE'\'' && ln -sf '\''key.log'\'' key.log'
	# 在key窗口执行下列命令
	screen -S stack -p key -X stuff 'sudo tail -f /var/log/apache2/keystone.log | sed -u '\''s/\\\\x1b/\o033/g'\'' & echo $! >/opt/stack/status/stack/key.pid; fg || echo "key failed to start. Exit code: $?" | tee "/opt/stack/status/stack/key.failure"^M'

	# 新建key-access窗口并指定key-access窗口的log file
	screen -S stack -X screen -t key-access
	screen -S stack -p key-access -X logfile /home/pengsida/LOGFILE/key-access.log
	screen -S stack -p key-access -X log on
	touch /home/pengsida/LOGFILE/key-access.log

	# 在key-access窗口执行下列命令
	screen -S stack -p key-access -X stuff 'sudo tail -f /var/log/apache2/keystone_access.log | sed -u '\''s/\\\\x1b/\o033/g'\'' & echo $! >/opt/stack/status/stack/key-access.pid; fg || echo "key-access failed to start. Exit code: $?" | tee "/opt/stack/status/stack/key-access.failure"^M'

	KEYSTONE_SERVICE_URI=http://$HOST_IP/identity/v3/
	KEYSTONE_AUTH_URI=http://$HOST_IP/identity_admin
	ADMIN_PASSWORD=p1111111
	REGION_NAME=RegionOne

	curl -g -k --noproxy '*' -s -o /dev/null -w '%{http_code}' $KEYSTONE_SERVICE_URI
	sudo service memcached restart
	\end{lstlisting}

\section{开启swift服务}
	\begin{lstlisting}
	sudo service memcached restart
	sudo /etc/init.d/rsync restart
    swift-init --run-dir=/opt/stack/data/swift/run all restart
	swift-init --run-dir=/opt/stack/data/swift/run proxy stop
	swift-init --run-dir=/opt/stack/data/swift/run object stop
	swift-init --run-dir=/opt/stack/data/swift/run container stop
	swift-init --run-dir=/opt/stack/data/swift/run account stop

	# 配置screen窗口
	# ==================

	# 指定stack这个screen,在其中创建一个叫s-proxy的窗口
	screen -S stack -X screen -t s-proxy
	# 指定dstat窗口的logfile
	screen -S stack -p s-proxy -X logfile /home/pengsida/LOGFILE/logs/s-proxy.log
    screen -S stack -p s-proxy -X log on
	# touch /home/pengsida/LOGFILE/s-proxy.log
    bash -c 'cd '\''/home/pengsida/LOGFILE'\'' && ln -sf '\''s-proxy.log'\'' s-proxy.log'

	# 在s-proxy窗口中执行这条命令
	screen -S stack -p s-proxy -X stuff 'swift-proxy-server /etc/swift/proxy-server.conf -v & echo $! >/opt/stack/status/stack/s-proxy.pid; fg || echo "s-proxy failed to start. Exit code: $?" | tee "/opt/stack/status/stack/s-proxy.failure"^M'


	# 指定stack这个screen,在其中创建一个叫s-object的窗口
	screen -S stack -X screen -t s-object
	# 指定dstat窗口的logfile
	screen -S stack -p s-object -X logfile /home/pengsida/LOGFILE/logs/s-object.log
    screen -S stack -p s-object -X log on
	# touch /home/pengsida/LOGFILE/s-object.log
    bash -c 'cd '\''/home/pengsida/LOGFILE'\'' && ln -sf '\''s-object.log'\'' s-object.log'

	# 在s-object窗口中执行这条命令
	screen -S stack -p s-object -X stuff 'swift-object-server /etc/swift/object-server/1.conf -v & echo $! >/opt/stack/status/stack/s-object.pid; fg || echo "s-object failed to start. Exit code: $?" | tee "/opt/stack/status/stack/s-object.failure"^M'


	# 指定stack这个screen,在其中创建一个叫s-container的窗口
	screen -S stack -X screen -t s-container
	# 指定dstat窗口的logfile
	screen -S stack -p s-container -X logfile /home/pengsida/LOGFILE/logs/s-container.log
    screen -S stack -p s-container -X log on
	# touch /home/pengsida/LOGFILE/s-container.log
    bash -c 'cd '\''/home/pengsida/LOGFILE'\'' && ln -sf '\''s-container.log'\'' s-container.log'

	# 在s-container窗口中执行这条命令
	screen -S stack -p s-container -X stuff 'swift-container-server /etc/swift/container-server/1.conf -v & echo $! >/opt/stack/status/stack/s-container.pid; fg || echo "s-container failed to start. Exit code: $?" | tee "/opt/stack/status/stack/s-container.failure"^M'

	
	# 指定stack这个screen,在其中创建一个叫s-account的窗口
	screen -S stack -X screen -t s-account
	# 指定dstat窗口的logfile
	screen -S stack -p s-account -X logfile /home/pengsida/LOGFILE/logs/s-account.log
    screen -S stack -p s-account -X log on
	touch /home/pengsida/LOGFILE/logs/s-account.log
    bash -c 'cd '\''/home/pengsida/LOGFILE'\'' && ln -sf '\''s-account.log'\'' s-account.log'

	# 在s-account窗口中执行这条命令
	screen -S stack -p s-account -X stuff 'swift-account-server /etc/swift/account-server/1.conf -v & echo $! >/opt/stack/status/stack/s-account.pid; fg || echo "s-account failed to start. Exit code: $?" | tee "/opt/stack/status/stack/s-account.failure"^M'

	# swift_configure_tempurls
	# openstack object store account set --property Temp-URL-Key=p1111111
	\end{lstlisting}

\section{开启glance服务}
	\begin{lstlisting}
	screen -S stack -X screen -t g-reg
	screen -S stack -p g-reg -X logfile /home/pengsida/LOGFILE/logs/g-reg.log
    screen -S stack -p g-reg -X log on
	touch /home/pengsida/LOGFILE/g-reg.log
    # bash -c 'cd '\''/home/pengsida/LOGFILE'\'' && ln -sf '\''g-reg.log'\'' g-reg.log'
	screen -S stack -p g-reg -X stuff 'sudo /usr/local/bin/glance-registry --config-file=/etc/glance/glance-registry.conf & echo $! >/opt/stack/status/stack/g-reg.pid; fg || echo "g-reg failed to start. Exit code: $?" | tee "/opt/stack/status/stack/g-reg.failure"^M'

	
	screen -S stack -X screen -t g-api
	screen -S stack -p g-api -X logfile /home/pengsida/LOGFILE/g-api.log
    screen -S stack -p g-api -X log on
	touch /home/pengsida/LOGFILE/g-api.log
    # bash -c 'cd '\''/home/pengsida/LOGFILE'\'' && ln -sf '\''g-api.log'\'' g-api.log'
	screen -S stack -p g-api -X stuff 'sudo /usr/local/bin/glance-api --config-file=/etc/glance/glance-api.conf & echo $! >/opt/stack/status/stack/g-api.pid; fg || echo "g-api failed to start. Exit code: $?" | tee "/opt/stack/status/stack/g-api.failure"^M'

	# 检查服务是否启动
	curl -g -k --noproxy '*' -s -o /dev/null -w '%{http_code}' http://$HOST_IP:9292
	\end{lstlisting}

\section{开启nova服务}
	\begin{lstlisting}
	# Create a randomized default value for the key manager's fixed_key
	iniset /etc/nova/nova.conf key_manager fixed_key $(hexdump -n 32 -v -e '/1 "%02x"' /dev/urandom)

	# 开启nova-api服务
	screen -S stack -X screen -t n-api
	screen -S stack -p n-api -X logfile /home/pengsida/LOGFILE/n-api.log
    screen -S stack -p n-api -X log on
	touch /home/pengsida/LOGFILE/n-api.log
    # bash -c 'cd '\''/home/pengsida/LOGFILE'\'' && ln -sf '\''n-api.log'\'' n-api.log'
	screen -S stack -p n-api -X stuff '/usr/local/bin/nova-api & echo $! >/opt/stack/status/stack/n-api.pid; fg || echo "n-api failed to start. Exit code: $?" | tee "/opt/stack/status/stack/n-api.failure"^M'

	# 监测nova-api是否启动
	curl -g -k --noproxy '*' -s -o /dev/null -w '%{http_code}' http://$HOST_IP:8774
	\end{lstlisting}

\section{开启neutron服务}
	\begin{lstlisting}
	# 开启neutron-server服务
	screen -S stack -X screen -t q-svc
	screen -S stack -p q-svc -X logfile /home/pengsida/LOGFILE/q-svc.log
    screen -S stack -p q-svc -X log on
	touch /home/pengsida/LOGFILE/q-svc.log
    bash -c 'cd '\''/home/pengsida/LOGFILE'\'' && ln -sf '\''q-svc.log'\'' q-svc.log'
	screen -S stack -p q-svc -X stuff 'sudo /usr/local/bin/neutron-server --config-file /etc/neutron/neutron.conf --config-file /etc/neutron/plugins/ml2/ml2_conf.ini & echo $! >/opt/stack/status/stack/q-svc.pid; fg || echo "q-svc failed to start. Exit code: $?" | tee "/opt/stack/status/stack/q-svc.failure"^M'

	# 监测neutron-server服务是否启动
	# timeout 60 sh -c 'while ! wget  --no-proxy -q -O- http://$HOST_IP:9696; do sleep 0.5; done'

	# 开启neutron-openvswitch-agent服务
	screen -S stack -X screen -t q-agt
	screen -S stack -p q-agt -X logfile /home/pengsida/LOGFILE/q-agt.log
    screen -S stack -p q-agt -X log on
	touch /home/pengsida/LOGFILE/q-agt.log
    # bash -c 'cd '\''/home/pengsida/LOGFILE'\'' && ln -sf '\''q-agt.log'\'' q-agt.log'
	screen -S stack -p q-agt -X stuff 'sudo /usr/local/bin/neutron-openvswitch-agent --config-file /etc/neutron/neutron.conf --config-file /etc/neutron/plugins/ml2/ml2_conf.ini & echo $! >/opt/stack/status/stack/q-agt.pid; fg || echo "q-agt failed to start. Exit code: $?" | tee "/opt/stack/status/stack/q-agt.failure"^M'

	# 开启neutron-dhcp-agent服务
	screen -S stack -X screen -t q-dhcp
	screen -S stack -p q-dhcp -X logfile /home/pengsida/LOGFILE/q-dhcp.log
    screen -S stack -p q-dhcp -X log on
	touch /home/pengsida/LOGFILE/q-dhcp.log
    bash -c 'cd '\''/home/pengsida/LOGFILE'\'' && ln -sf '\''q-dhcp.log'\'' q-dhcp.log'
	screen -S stack -p q-dhcp -X stuff 'sudo /usr/local/bin/neutron-dhcp-agent --config-file /etc/neutron/neutron.conf --config-file /etc/neutron/dhcp_agent.ini & echo $! >/opt/stack/status/stack/q-dhcp.pid; fg || echo "q-dhcp failed to start. Exit code: $?" | tee "/opt/stack/status/stack/q-dhcp.failure"^M'

	# 开启neutron-l3-agent服务
	screen -S stack -X screen -t q-l3
	screen -S stack -p q-l3 -X logfile /home/pengsida/LOGFILE/q-l3.log
    screen -S stack -p q-l3 -X log on
	touch /home/pengsida/LOGFILE/q-l3.log
    bash -c 'cd '\''/home/pengsida/LOGFILE'\'' && ln -sf '\''q-l3.log'\'' q-l3.log'
	screen -S stack -p q-l3 -X stuff 'sudo /usr/local/bin/neutron-l3-agent --config-file /etc/neutron/neutron.conf --config-file /etc/neutron/l3_agent.ini & echo $! >/opt/stack/status/stack/q-l3.pid; fg || echo "q-l3 failed to start. Exit code: $?" | tee "/opt/stack/status/stack/q-l3.failure"^M'

	# 开启neutron-metadata-agent服务
	screen -S stack -X screen -t q-meta
	screen -S stack -p q-meta -X logfile /home/pengsida/LOGFILE/q-meta.log
    screen -S stack -p q-meta -X log on
	touch /home/pengsida/LOGFILE/q-meta.log
    bash -c 'cd '\''/home/pengsida/LOGFILE'\'' && ln -sf '\''q-meta.log'\'' q-meta.log'
	screen -S stack -p q-meta -X stuff 'sudo /usr/local/bin/neutron-metadata-agent --config-file /etc/neutron/neutron.conf --config-file /etc/neutron/metadata_agent.ini & echo $! >/opt/stack/status/stack/q-meta.pid; fg || echo "q-meta failed to start. Exit code: $?" | tee "/opt/stack/status/stack/q-meta.failure"^M'
	\end{lstlisting}

\section{开启nova服务}
	\begin{lstlisting}
	# 开启nova-conductor服务
	screen -S stack -X screen -t n-cond
	screen -S stack -p n-cond -X logfile /home/pengsida/LOGFILE/n-cond.log
    screen -S stack -p n-cond -X log on
	touch /home/pengsida/LOGFILE/n-cond.log
    # bash -c 'cd '\''/home/pengsida/LOGFILE'\'' && ln -sf '\''n-cond.log'\'' n-cond.log'
	screen -S stack -p n-cond -X stuff '/usr/local/bin/nova-conductor --config-file /etc/nova/nova.conf & echo $! >/opt/stack/status/stack/n-cond.pid; fg || echo "n-cond failed to start. Exit code: $?" | tee "/opt/stack/status/stack/n-cond.failure"^M'

	# 开启nova-scheduler服务
	screen -S stack -X screen -t n-sch
	screen -S stack -p n-sch -X logfile /home/pengsida/LOGFILE/n-sch.log
    screen -S stack -p n-sch -X log on
	touch /home/pengsida/LOGFILE/n-sch.log
    # bash -c 'cd '\''/home/pengsida/LOGFILE'\'' && ln -sf '\''n-sch.log'\'' n-sch.log'
	screen -S stack -p n-sch -X stuff '/usr/local/bin/nova-scheduler --config-file /etc/nova/nova.conf & echo $! >/opt/stack/status/stack/n-sch.pid; fg || echo "n-sch failed to start. Exit code: $?" | tee "/opt/stack/status/stack/n-sch.failure"^M'

	# 开启nova-consoleauth服务
	screen -S stack -X screen -t n-cauth
	screen -S stack -p n-cauth -X logfile /home/pengsida/LOGFILE/n-cauth.log
    screen -S stack -p n-cauth -X log on
	touch /home/pengsida/LOGFILE/n-cauth.log
    # bash -c 'cd '\''/home/pengsida/LOGFILE'\'' && ln -sf '\''n-cauth.log'\'' n-cauth.log'
	screen -S stack -p n-cauth -X stuff '/usr/local/bin/nova-consoleauth --config-file /etc/nova/nova.conf & echo $! >/opt/stack/status/stack/n-cauth.pid; fg || echo "n-cauth failed to start. Exit code: $?" | tee "/opt/stack/status/stack/n-cauth.failure"^M'

	# 开启nova-compute服务
	screen -S stack -X screen -t n-cpu
	screen -S stack -p n-cpu -X logfile /home/pengsida/LOGFILE/n-cpu.log
    screen -S stack -p n-cpu -X log on
	touch /home/pengsida/LOGFILE/n-cpu.log
    # bash -c 'cd '\''/home/pengsida/LOGFILE'\'' && ln -sf '\''n-cpu.log'\'' n-cpu.log'
	screen -S stack -p n-cpu -X stuff '/usr/local/bin/nova-compute --config-file /etc/nova/nova.conf & echo $! >/opt/stack/status/stack/n-cpu.pid; fg || echo "n-cpu failed to start. Exit code: $?" | tee "/opt/stack/status/stack/n-cpu.failure"^M'
	\end{lstlisting}

\section{开启placement服务}
	\begin{lstlisting}
	# 开启placement-api服务
	screen -S stack -X screen -t placement-api
	screen -S stack -p placement-api -X logfile /home/pengsida/LOGFILE/placement-api.log
    screen -S stack -p placement-api -X log on
	touch /home/pengsida/LOGFILE/placement-api.log
    bash -c 'cd '\''/home/pengsida/LOGFILE'\'' && ln -sf '\''placement-api.log'\'' placement-api.log'
	screen -S stack -p placement-api -X stuff 'sudo tail -f /var/log/apache2/placement-api.log | sed -u '\''s/\\\\x1b/\o033/g'\'' & echo $! >/opt/stack/status/stack/placement-api.pid; fg || echo "placement-api failed to start. Exit code: $?" | tee "/opt/stack/status/stack/placement-api.failure"^M'

	# 检测placement-api服务是否开启
	curl -g -k --noproxy '*' -s -o /dev/null -w '%{http_code}' http://$HOST_IP/placement
	\end{lstlisting}

\section{开启ironic服务}
	\begin{lstlisting}
	# 开启ironic服务
	# ====================

	# 开启ironic-api服务
	screen -S stack -X screen -t ir-api
	screen -S stack -p ir-api -X logfile /home/pengsida/LOGFILE/ir-api.log
    screen -S stack -p ir-api -X log on
	touch /home/pengsida/LOGFILE/ir-api.log
    # bash -c 'cd '\''/home/pengsida/LOGFILE'\'' && ln -sf '\''ir-api.log'\'' ir-api.log'
	screen -S stack -p ir-api -X stuff '/usr/local/bin/ironic-api --config-file=/etc/ironic/ironic.conf & echo $! >/opt/stack/status/stack/ir-api.pid; fg || echo "ir-api failed to start. Exit code: $?" | tee "/opt/stack/status/stack/ir-api.failure"^M'

	# 开启ironic-conductor服务
	screen -S stack -X screen -t ir-cond
	screen -S stack -p ir-cond -X logfile /home/pengsida/LOGFILE/ir-cond.log
    screen -S stack -p ir-cond -X log on
	touch /home/pengsida/LOGFILE/ir-cond.log
    # bash -c 'cd '\''/home/pengsida/LOGFILE'\'' && ln -sf '\''ir-cond.log'\'' ir-cond.log'
	screen -S stack -p ir-cond -X stuff '/usr/local/bin/ironic-conductor --config-file=/etc/ironic/ironic.conf & echo $! >/opt/stack/status/stack/ir-cond.pid; fg || echo "ir-cond failed to start. Exit code: $?" | tee "/opt/stack/status/stack/ir-cond.failure"^M'

	# restart_apache_server
	sudo service apache2 stop
	sudo service apache2 start
	\end{lstlisting}

\end{document}