% !TeX spellcheck = en_US
%% 字体:方正静蕾简体
%%		 方正粗宋
\documentclass[a4paper,left=1.5cm,right=1.5cm,11pt]{article}

\usepackage[utf8]{inputenc}
\usepackage{fontspec}
\usepackage{cite}
\usepackage{xeCJK}
\usepackage{indentfirst}
\usepackage{titlesec}
\usepackage{etoolbox}%
\makeatletter
\patchcmd{\ttlh@hang}{\parindent\z@}{\parindent\z@\leavevmode}{}{}%
\patchcmd{\ttlh@hang}{\noindent}{}{}{}%
\makeatother
\usepackage{hyperref}
\usepackage{longtable}
\usepackage{empheq}
\usepackage{graphicx}
\usepackage{float}
\usepackage{rotating}
\usepackage{subfigure}
\usepackage{tabu}
\usepackage{amsmath}
\usepackage{setspace}
\usepackage{amsfonts}
\usepackage{appendix}
\usepackage{listings}
\usepackage{xcolor}
\usepackage{geometry}
\setcounter{secnumdepth}{4}
%\titleformat*{\section}{\LARGE}
%\renewcommand\refname{参考文献}
%\titleformat{\chapter}{\centering\bfseries\huge}{}{0.7em}{}{}
\titleformat{\section}{\LARGE\bf}{\thesection}{1em}{}{}
\titleformat{\subsection}{\Large\bfseries}{\thesubsection}{1em}{}{}
\titleformat{\subsubsection}{\large\bfseries}{\thesubsubsection}{1em}{}{}
\renewcommand{\contentsname}{{ \centerline{目{  } 录}}}
\setCJKfamilyfont{cjkhwxk}{STXINGKA.TTF}
%\setCJKfamilyfont{cjkhwxk}{华文行楷}
%\setCJKfamilyfont{cjkfzcs}{方正粗宋简体}
%\newcommand*{\cjkfzcs}{\CJKfamily{cjkfzcs}}
\newcommand*{\cjkhwxk}{\CJKfamily{cjkhwxk}}
%\newfontfamily\wryh{Microsoft YaHei}
%\newfontfamily\hwzs{华文中宋}
%\newfontfamily\hwst{华文宋体}
%\newfontfamily\hwfs{华文仿宋}
%\newfontfamily\jljt{方正静蕾简体}
%\newfontfamily\hwxk{华文行楷}
\newcommand{\verylarge}{\fontsize{60pt}{\baselineskip}\selectfont}  
\newcommand{\chuhao}{\fontsize{44.9pt}{\baselineskip}\selectfont}  
\newcommand{\xiaochu}{\fontsize{38.5pt}{\baselineskip}\selectfont}  
\newcommand{\yihao}{\fontsize{27.8pt}{\baselineskip}\selectfont}  
\newcommand{\xiaoyi}{\fontsize{25.7pt}{\baselineskip}\selectfont}  
\newcommand{\erhao}{\fontsize{23.5pt}{\baselineskip}\selectfont}  
\newcommand{\xiaoerhao}{\fontsize{19.3pt}{\baselineskip}\selectfont} 
\newcommand{\sihao}{\fontsize{14pt}{\baselineskip}\selectfont}      % 字号设置  
\newcommand{\xiaosihao}{\fontsize{12pt}{\baselineskip}\selectfont}  % 字号设置  
\newcommand{\wuhao}{\fontsize{10.5pt}{\baselineskip}\selectfont}    % 字号设置  
\newcommand{\xiaowuhao}{\fontsize{9pt}{\baselineskip}\selectfont}   % 字号设置  
\newcommand{\liuhao}{\fontsize{7.875pt}{\baselineskip}\selectfont}  % 字号设置  
\newcommand{\qihao}{\fontsize{5.25pt}{\baselineskip}\selectfont}    % 字号设置 

\usepackage{diagbox}
\usepackage{multirow}
\boldmath
\XeTeXlinebreaklocale "zh"
\XeTeXlinebreakskip = 0pt plus 1pt minus 0.1pt
\definecolor{cred}{rgb}{0.8,0.8,0.8}
\definecolor{cgreen}{rgb}{0,0.3,0}
\definecolor{cpurple}{rgb}{0.5,0,0.35}
\definecolor{cdocblue}{rgb}{0,0,0.3}
\definecolor{cdark}{rgb}{0.95,1.0,1.0}
\lstset{
	language=bash,
	numbers=left,
	numberstyle=\tiny\color{black},
	showspaces=false,
	showstringspaces=false,
	basicstyle=\scriptsize,
	keywordstyle=\color{purple},
	commentstyle=\color{cgreen},
	stringstyle=\color{blue},
	frame=lines,
	% escapeinside=``,
	extendedchars=true, 
	xleftmargin=1em,
	xrightmargin=1em, 
	backgroundcolor=\color{cred},
	aboveskip=1em,
	breaklines=true,
	tabsize=4
} 

%\newfontfamily{\consolas}{Consolas}
%\newfontfamily{\monaco}{Monaco}
%\setmonofont[Mapping={}]{Consolas}	%英文引号之类的正常显示,相当于设置英文字体
%\setsansfont{Consolas} %设置英文字体 Monaco, Consolas,  Fantasque Sans Mono
%\setmainfont{Times New Roman}
%\setCJKmainfont{STZHONGS.TTF}
%\setmonofont{Consolas}
% \newfontfamily{\consolas}{YaHeiConsolas.ttf}
\newfontfamily{\monaco}{MONACO.TTF}
\setCJKmainfont{STZHONGS.TTF}
%\setmainfont{MONACO.TTF}
%\setsansfont{MONACO.TTF}

\newcommand{\fic}[1]{\begin{figure}[H]
		\center
		\includegraphics[width=0.8\textwidth]{#1}
	\end{figure}}
	
\newcommand{\sizedfic}[2]{\begin{figure}[H]
		\center
		\includegraphics[width=#1\textwidth]{#2}
	\end{figure}}

\newcommand{\codefile}[1]{\lstinputlisting{#1}}

\newcommand{\interval}{\vspace{0.5em}}

\newcommand{\tablestart}{
	\interval
	\begin{longtable}{p{2cm}p{10cm}}
	\hline}
\newcommand{\tableend}{
	\hline
	\end{longtable}
	\interval}

% 改变段间隔
\setlength{\parskip}{0.2em}
\linespread{1.1}

\usepackage{lastpage}
\usepackage{fancyhdr}
\pagestyle{fancy}
\lhead{\space \qquad \space}
\chead{keystone核心概念 \qquad}
\rhead{\qquad\thepage/\pageref{LastPage}}

\begin{document}

\tableofcontents

\clearpage

\section{keystone核心概念}
\subsection{user}
	user指代任何使用OpenStack的实体,可以是真正的用户,其他系统或者服务。
	当user请求访问 OpenStack 时,Keystone 会对其进行验证。\par

	除了 admin 和 demo,OpenStack 也为 nova、cinder、glance、neutron 服务创建了相应的user。 
	admin 也可以管理这些 User。\par

	查看所有的user的命令如下:
	\begin{lstlisting}
	openstack user list
	\end{lstlisting}

	如下图所示:
	\fic{2.png}

	创建相应的user的命令如下:
	\begin{lstlisting}
	openstack user create demo --password p1111111 --domain=default --email=demo@example.com
	openstack user create alt_demo --password p1111111 --domain=default --email=alt_demo@example.com
	openstack user create nova --password p1111111 --domain=Default
	openstack user create glance --password p1111111 --domain=Default
	openstack user create glance-swift --password p1111111 --domain=Default
	openstack user create neutron --password p1111111 --domain=Default
	openstack user create swift --password p1111111 --domain=Default
	openstack user create swiftusertest1 --password testing --domain=default --email=test@example.com
	openstack user create swiftusertest3 --password testing3 --domain=default --email=test3@example.com
	openstack user create swiftusertest2 --password testing2 --domain=default --email=test2@example.com
	openstack user create swiftusertest4 --password testing4 --domain=7101c820856a4114bd3eb89a3b96cdae --email=test4@example.com --or-show -f value -c id
	openstack user create placement --password p1111111 --domain=Default
	openstack user create ironic --password p1111111 --domain=Default
	\end{lstlisting}

\subsection{credentials}

\subsection{authentication}

\subsection{token}

\subsection{project}
	Project 用于将 OpenStack 的资源(计算、存储和网络)进行分组和隔离。 
	根据 OpenStack 服务的对象不同,Project 可以是一个客户(公有云,也叫租户)、部门或者项目组(私有云)。\par

	需要注意的地方:
	\begin{itemize}
		\item 资源的所有权是属于 Project 的,而不是 User。
		\item 在 OpenStack 的界面和文档中,Tenant / Project / Account 这几个术语是通用的,但长期看会倾向使用 Project。
		\item 每个 User(包括 admin)必须挂在 Project 里才能访问该 Project 的资源。 一个User可以属于多个 Project。
		\item admin 相当于 root 用户,具有最高权限。
	\end{itemize}

	查看所有的project的命令如下:
	\begin{lstlisting}
	openstack project list
	\end{lstlisting}

	如下图所示:
	\fic{3.png}

	创建相应的project的命令如下:
	\begin{lstlisting}
	openstack project create service --domain=Default
	openstack project create invisible_to_admin --domain=default
	openstack project create demo --domain=default
	openstack project create alt_demo --domain=default
	openstack project create swiftprojecttest1 --domain=default
	openstack project create swiftprojecttest2 --domain=default
	openstack project create swiftprojecttest4 --domain=7101c820856a4114bd3eb89a3b96cdae
	openstack project create service --domain=default
	\end{lstlisting}

\subsection{service}
	OpenStack 的 Service 包括 Compute (Nova)、Block Storage (Cinder)、Object Storage (Swift)、Image Service (Glance) 、Networking Service (Neutron) 等。\par

	每个 Service 都会提供若干个 Endpoint,User 通过 Endpoint 访问资源和执行操作。\par

	创建一个服务实体的例子如下:
	\begin{lstlisting}
	openstack service create network --name neutron '--description=Neutron Service' -f value -c id
	\end{lstlisting}

\subsection{endpoint}
	Endpoint 是一个网络上可访问的地址,通常是一个 URL。
	Service 通过 Endpoint 暴露自己的 API。 
	Keystone 负责管理和维护每个 Service 的 Endpoint。\par

	创建一个service的endpoint的例子如下:
	\begin{lstlisting}
	openstack endpoint create baremetal public http://10.250.1.3:6385 --region RegionOne
	openstack endpoint create baremetal admin http://10.250.1.3:6385 --region RegionOne
	openstack endpoint create baremetal internal http://10.250.1.3:6385 --region RegionOne
	\end{lstlisting}

	可以通过如下命令查看各个service的endpoint:
	\begin{lstlisting}
	openstack catalog list
	\end{lstlisting}

	如下图所示:
	\fic{1.png}

\subsection{role}
	安全包含两部分:Authentication(认证)和 Authorization(鉴权)。
	Authentication 解决的是“你是谁?”的问题,Authorization 解决的是“你能干什么?”的问题。
	Keystone 是借助 Role 来实现 Authorization。
	也就是说角色可以被指定给用户,使得该用户获得角色对应的操作权限。\par

	查看所有的role的命令如下:
	\begin{lstlisting}
	openstack role list
	\end{lstlisting}

	创建相应的role的命令如下:
	\begin{lstlisting}
	openstack role create service
	openstack role create ResellerAdmin
	openstack role create Member
	openstack role create member
	openstack role create anotherrole
	openstack role create anotherrole
	openstack role create baremetal_admin
	openstack role create baremetal_observer
	\end{lstlisting}

	Adds a role assignment to a user or group on a domain or project:
	\begin{lstlisting}
	openstack role add admin --user 10ec861de94c473ab6051644891c48da --project 554c828e8fd347e6be93c2e5e855fa1f
	openstack role add admin --user 34ee1ff9a328464987c2229dfa1ac3e5 --project 3b7952fbf71b43dabfc70008a5867ee8
	openstack role add service --user placement --project service --user-domain Default --project-domain Default
	openstack role add admin --user placement --project service --user-domain Default --project-domain Default
	openstack role add service --user ironic --project service --user-domain Default --project-domain Default
	openstack role add admin --user ironic --project service --user-domain Default --project-domain Default
	openstack role add baremetal_admin --user nova --project service
	openstack role add baremetal_observer --user demo --project demo
	\end{lstlisting}

\section{domain}
	keystone v3利用 Domain 实现真正的多租户(multi-tenancy)架构,Domain 担任 Project 的高层容器。
	云服务的客户是 Domain 的所有者,他们可以在自己的 Domain 中创建多个 Projects、Users、Groups 和 Roles。通过引入 Domain,云服务客户可以对其拥有的多个 Project 进行统一管理,而不必再向过去那样对每一个 Project 进行单独管理。\par

\section{group}
	Group 是一组 Users 的容器,可以向 Group 中添加用户,并直接给 Group 分配角色,那么在这个 Group 中的所有用户就都拥有了 Group 所拥有的角色权限。通过引入 Group 的概念,Keystone V3 实现了对用户组的管理,达到了同时管理一组用户权限的目的。
	这与 V2 中直接向 User/Project 指定 Role 不同,使得对云服务进行管理更加便捷。

\subsection{Domain、Group、Project、User 和 Role 的关系图}
	如图所示:
	\fic{5.png}

	在一个 Domain 中包含 3 个 Projects,可以通过 Group1 将 Role Sysadmin直接赋予 Domain,那么 Group1 中的所有用户将会对 Domain 中的所有 Projects 都拥有管理员权限。
	也可以通过 Group2 将 Role Engineer 只赋予 Project3,这样 Group2 中的 User 就只拥有对 Project3 相应的权限,而不会影响其它 Projects。

\section{参考资料}
	上述的笔记来自于这个网站:
	\href{https://www.ibm.com/developerworks/community/blogs/132cfa78-44b0-4376-85d0-d3096cd30d3f/entry/%E7%90%86%E8%A7%A3_Keystone_%E6%A0%B8%E5%BF%83%E6%A6%82%E5%BF%B5_%E6%AF%8F%E5%A4%A95%E5%88%86%E9%92%9F%E7%8E%A9%E8%BD%AC_OpenStack_18?lang=en}{理解keystone核心概念}
	
\end{document}