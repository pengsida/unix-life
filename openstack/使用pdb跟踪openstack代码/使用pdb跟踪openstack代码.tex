% !TeX spellcheck = en_US
%% 字体:方正静蕾简体
%%		 方正粗宋
\documentclass[a4paper,left=1.5cm,right=1.5cm,11pt]{article}

\usepackage[utf8]{inputenc}
\usepackage{fontspec}
\usepackage{cite}
\usepackage{xeCJK}
\usepackage{indentfirst}
\usepackage{titlesec}
\usepackage{etoolbox}%
\makeatletter
\patchcmd{\ttlh@hang}{\parindent\z@}{\parindent\z@\leavevmode}{}{}%
\patchcmd{\ttlh@hang}{\noindent}{}{}{}%
\makeatother

\usepackage{longtable}
\usepackage{empheq}
\usepackage{graphicx}
\usepackage{float}
\usepackage{rotating}
\usepackage{subfigure}
\usepackage{tabu}
\usepackage{amsmath}
\usepackage{setspace}
\usepackage{amsfonts}
\usepackage{appendix}
\usepackage{listings}
\usepackage{xcolor}
\usepackage{geometry}
\setcounter{secnumdepth}{4}
%\titleformat*{\section}{\LARGE}
%\renewcommand\refname{参考文献}
%\titleformat{\chapter}{\centering\bfseries\huge}{}{0.7em}{}{}
\titleformat{\section}{\LARGE\bf}{\thesection}{1em}{}{}
\titleformat{\subsection}{\Large\bfseries}{\thesubsection}{1em}{}{}
\titleformat{\subsubsection}{\large\bfseries}{\thesubsubsection}{1em}{}{}
\renewcommand{\contentsname}{{ \centerline{目{  } 录}}}
\setCJKfamilyfont{cjkhwxk}{STXINGKA.TTF}
%\setCJKfamilyfont{cjkhwxk}{华文行楷}
%\setCJKfamilyfont{cjkfzcs}{方正粗宋简体}
%\newcommand*{\cjkfzcs}{\CJKfamily{cjkfzcs}}
\newcommand*{\cjkhwxk}{\CJKfamily{cjkhwxk}}
%\newfontfamily\wryh{Microsoft YaHei}
%\newfontfamily\hwzs{华文中宋}
%\newfontfamily\hwst{华文宋体}
%\newfontfamily\hwfs{华文仿宋}
%\newfontfamily\jljt{方正静蕾简体}
%\newfontfamily\hwxk{华文行楷}
\newcommand{\verylarge}{\fontsize{60pt}{\baselineskip}\selectfont}  
\newcommand{\chuhao}{\fontsize{44.9pt}{\baselineskip}\selectfont}  
\newcommand{\xiaochu}{\fontsize{38.5pt}{\baselineskip}\selectfont}  
\newcommand{\yihao}{\fontsize{27.8pt}{\baselineskip}\selectfont}  
\newcommand{\xiaoyi}{\fontsize{25.7pt}{\baselineskip}\selectfont}  
\newcommand{\erhao}{\fontsize{23.5pt}{\baselineskip}\selectfont}  
\newcommand{\xiaoerhao}{\fontsize{19.3pt}{\baselineskip}\selectfont} 
\newcommand{\sihao}{\fontsize{14pt}{\baselineskip}\selectfont}      % 字号设置  
\newcommand{\xiaosihao}{\fontsize{12pt}{\baselineskip}\selectfont}  % 字号设置  
\newcommand{\wuhao}{\fontsize{10.5pt}{\baselineskip}\selectfont}    % 字号设置  
\newcommand{\xiaowuhao}{\fontsize{9pt}{\baselineskip}\selectfont}   % 字号设置  
\newcommand{\liuhao}{\fontsize{7.875pt}{\baselineskip}\selectfont}  % 字号设置  
\newcommand{\qihao}{\fontsize{5.25pt}{\baselineskip}\selectfont}    % 字号设置 

\usepackage{diagbox}
\usepackage{multirow}
\boldmath
\XeTeXlinebreaklocale "zh"
\XeTeXlinebreakskip = 0pt plus 1pt minus 0.1pt
\definecolor{cred}{rgb}{0.8,0.8,0.8}
\definecolor{cgreen}{rgb}{0,0.3,0}
\definecolor{cpurple}{rgb}{0.5,0,0.35}
\definecolor{cdocblue}{rgb}{0,0,0.3}
\definecolor{cdark}{rgb}{0.95,1.0,1.0}
\lstset{
	language=python,
	numbers=left,
	numberstyle=\tiny\color{black},
	showspaces=false,
	showstringspaces=false,
	basicstyle=\scriptsize,
	keywordstyle=\color{purple},
	commentstyle=\itshape\color{cgreen},
	stringstyle=\color{blue},
	frame=lines,
	% escapeinside=``,
	extendedchars=true, 
	xleftmargin=1em,
	xrightmargin=1em, 
	backgroundcolor=\color{cred},
	aboveskip=1em,
	breaklines=true,
	tabsize=4
} 

%\newfontfamily{\consolas}{Consolas}
%\newfontfamily{\monaco}{Monaco}
%\setmonofont[Mapping={}]{Consolas}	%英文引号之类的正常显示,相当于设置英文字体
%\setsansfont{Consolas} %设置英文字体 Monaco, Consolas,  Fantasque Sans Mono
%\setmainfont{Times New Roman}
%\setCJKmainfont{STZHONGS.TTF}
%\setmonofont{Consolas}
% \newfontfamily{\consolas}{YaHeiConsolas.ttf}
\newfontfamily{\monaco}{MONACO.TTF}
\setCJKmainfont{STZHONGS.TTF}
%\setmainfont{MONACO.TTF}
%\setsansfont{MONACO.TTF}

\newcommand{\fic}[1]{\begin{figure}[H]
		\center
		\includegraphics[width=0.8\textwidth]{#1}
	\end{figure}}
	
\newcommand{\sizedfic}[2]{\begin{figure}[H]
		\center
		\includegraphics[width=#1\textwidth]{#2}
	\end{figure}}

\newcommand{\codefile}[1]{\lstinputlisting{#1}}

\newcommand{\interval}{\vspace{0.5em}}

\newcommand{\tablestart}{
	\interval
	\begin{longtable}{p{2cm}p{10cm}}
	\hline}
\newcommand{\tableend}{
	\hline
	\end{longtable}
	\interval}

% 改变段间隔
\setlength{\parskip}{0.2em}
\linespread{1.1}

\usepackage{lastpage}
\usepackage{fancyhdr}
\pagestyle{fancy}
\lhead{\space \qquad \space}
\chead{使用pdb跟踪openstack代码 \qquad}
\rhead{\qquad\thepage/\pageref{LastPage}}
\begin{document}

\tableofcontents

\clearpage

\section{使用pdb跟踪openstack代码}
	这部分内容比较适合调试使用devstack进行安装的openstack。\par

\subsection{devstack screen使用技巧}
	devstack环境中,openstack运行在一个screen中,每个service运行在一个窗口中,如下图所示:
	\sizedfic{1}{1.png}

	可以看到图中底下一排窗口:
	\sizedfic{1}{2.png}

\subsubsection{screen -list}
	这个命令可以查看devstack中有哪些screen,如下图所示:
	\fic{3.png}

\subsubsection{screen -d <screen-id>}
	这个命令用于断开与screen的连接。
	% \fic{4.png}

\subsubsection{screen -r <screen-id>}
	这个命令用于断开并重新进入screen。

\subsubsection{screen -x <screen-id>}
	这个命令相当于在新的终端中查看这个screen。

\subsubsection{screen -h}
	通过这个命令可以查看关于screen的更多信息。

\subsubsection{进入screen以后的操作}
	使用screen -x <screen-id>进入某个screen以后,界面如下:
	\sizedfic{1}{1.png}

	图中带*号的是当前窗口。\par

	可以使用“ctrl+a+n”移动到下一个窗口,可以使用“ctrl+a+p”移动到上一个窗口。
	可以使用“ctrl+a+"”进入选择窗口的模式。\par

	选择窗口还有一个更快的方式,使用使用“ctrl+a+shift+"”,就可以进入如下界面:
	\sizedfic{0.9}{5.png}

	使用“ctrl+a+[”以后就可以滚动窗口了,使用“ctrl+]”重新固定窗口。\par

\subsubsection{借助screen重启openstack服务}
	首先进入想重启的服务的窗口,比如“nova-api”,如下图所示:
	\sizedfic{1}{6.png}

	然后按“ctrl+c”,杀死进程。随后按向上键,出现启动进程的命令,输入即重启“nova-api”服务。

\subsection{使用pdb}
	python中有一个pdb模块,使python代码也可以像gdb一样调试。\par

	只要在需要开始调试的地方加上以下语句即可:
	\begin{lstlisting}
	import pdb; pdb.set_trace()
	\end{lstlisting}

	如下图所示:
	\fic{7.png}

	随后重启代码处相应的服务,比如这里想调试的是“nova-api”,就借助screen重启“nova-api”服务。\par

	随后使用相应的命令触发这个断电,进入“nova-api”的窗口,就可以开始调试了,如下图所示:
	\fic{8.png}

\clearpage

\subsubsection{pdb常用调试命令}
	pdb常用调试命令如下所示:
	\begin{lstlisting}
	q  退出debug
	h  打印可用的调试命令
	b  设置断点,b 5 在第五行设置断点
	h command  打印command的命令含义
	disable codenum  使某一行断点失效
	enable codenum   使某一行的断点有效
	condition codenum xxx  针对断点设置条件
	c    继续执行程序,直到下一个断点
	n    执行下一行代码,如果当前语句有函数调用,则不会进入函数体中
	s    执行下一行代码,但是s会进入函数
	w    打印当前执行点的位置
	j    codenum  让程序跳转到指定的行
	l    列出附近的源码
	p    打印一个参数的值
	a    打印当前函数及参数的值
	u	 跳到上一层代码,也就是调用当前函数的函数
	d	 跳到下一层代码,也就是进入当前所处的函数中
	回车  重复执行上一行的命令
	\end{lstlisting}

	pdb下有一点非常好用,就是它相当于python shell,支持python中的各种语句。
	
\end{document}
