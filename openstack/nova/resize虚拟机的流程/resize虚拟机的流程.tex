% !TeX spellcheck = en_US
%% 字体:方正静蕾简体
%%		 方正粗宋
\documentclass[a4paper,left=1.5cm,right=1.5cm,11pt]{article}

\usepackage[utf8]{inputenc}
\usepackage{fontspec}
\usepackage{cite}
\usepackage{xeCJK}
\usepackage{indentfirst}
\usepackage{titlesec}
\usepackage{etoolbox}%
\makeatletter
\patchcmd{\ttlh@hang}{\parindent\z@}{\parindent\z@\leavevmode}{}{}%
\patchcmd{\ttlh@hang}{\noindent}{}{}{}%
\makeatother

\usepackage{longtable}
\usepackage{empheq}
\usepackage{graphicx}
\usepackage{float}
\usepackage{rotating}
\usepackage{subfigure}
\usepackage{tabu}
\usepackage{amsmath}
\usepackage{setspace}
\usepackage{amsfonts}
\usepackage{appendix}
\usepackage{listings}
\usepackage{xcolor}
\usepackage{geometry}
\setcounter{secnumdepth}{4}
%\titleformat*{\section}{\LARGE}
%\renewcommand\refname{参考文献}
%\titleformat{\chapter}{\centering\bfseries\huge}{}{0.7em}{}{}
\titleformat{\section}{\LARGE\bf}{\thesection}{1em}{}{}
\titleformat{\subsection}{\Large\bfseries}{\thesubsection}{1em}{}{}
\titleformat{\subsubsection}{\large\bfseries}{\thesubsubsection}{1em}{}{}
\renewcommand{\contentsname}{{ \centerline{目{  } 录}}}
\setCJKfamilyfont{cjkhwxk}{STXINGKA.TTF}
%\setCJKfamilyfont{cjkhwxk}{华文行楷}
%\setCJKfamilyfont{cjkfzcs}{方正粗宋简体}
%\newcommand*{\cjkfzcs}{\CJKfamily{cjkfzcs}}
\newcommand*{\cjkhwxk}{\CJKfamily{cjkhwxk}}
%\newfontfamily\wryh{Microsoft YaHei}
%\newfontfamily\hwzs{华文中宋}
%\newfontfamily\hwst{华文宋体}
%\newfontfamily\hwfs{华文仿宋}
%\newfontfamily\jljt{方正静蕾简体}
%\newfontfamily\hwxk{华文行楷}
\newcommand{\verylarge}{\fontsize{60pt}{\baselineskip}\selectfont}  
\newcommand{\chuhao}{\fontsize{44.9pt}{\baselineskip}\selectfont}  
\newcommand{\xiaochu}{\fontsize{38.5pt}{\baselineskip}\selectfont}  
\newcommand{\yihao}{\fontsize{27.8pt}{\baselineskip}\selectfont}  
\newcommand{\xiaoyi}{\fontsize{25.7pt}{\baselineskip}\selectfont}  
\newcommand{\erhao}{\fontsize{23.5pt}{\baselineskip}\selectfont}  
\newcommand{\xiaoerhao}{\fontsize{19.3pt}{\baselineskip}\selectfont} 
\newcommand{\sihao}{\fontsize{14pt}{\baselineskip}\selectfont}      % 字号设置  
\newcommand{\xiaosihao}{\fontsize{12pt}{\baselineskip}\selectfont}  % 字号设置  
\newcommand{\wuhao}{\fontsize{10.5pt}{\baselineskip}\selectfont}    % 字号设置  
\newcommand{\xiaowuhao}{\fontsize{9pt}{\baselineskip}\selectfont}   % 字号设置  
\newcommand{\liuhao}{\fontsize{7.875pt}{\baselineskip}\selectfont}  % 字号设置  
\newcommand{\qihao}{\fontsize{5.25pt}{\baselineskip}\selectfont}    % 字号设置 

\usepackage{diagbox}
\usepackage{multirow}
\boldmath
\XeTeXlinebreaklocale "zh"
\XeTeXlinebreakskip = 0pt plus 1pt minus 0.1pt
\definecolor{cred}{rgb}{0.8,0.8,0.8}
\definecolor{cgreen}{rgb}{0,0.3,0}
\definecolor{cpurple}{rgb}{0.5,0,0.35}
\definecolor{cdocblue}{rgb}{0,0,0.3}
\definecolor{cdark}{rgb}{0.95,1.0,1.0}
\lstset{
	language=bash,
	numbers=left,
	numberstyle=\tiny\color{black},
	showspaces=false,
	showstringspaces=false,
	basicstyle=\scriptsize,
	keywordstyle=\color{purple},
	commentstyle=\itshape\color{cgreen},
	stringstyle=\color{blue},
	frame=lines,
	% escapeinside=``,
	extendedchars=true, 
	xleftmargin=1em,
	xrightmargin=1em, 
	backgroundcolor=\color{cred},
	aboveskip=1em,
	breaklines=true,
	tabsize=4
} 

%\newfontfamily{\consolas}{Consolas}
%\newfontfamily{\monaco}{Monaco}
%\setmonofont[Mapping={}]{Consolas}	%英文引号之类的正常显示,相当于设置英文字体
%\setsansfont{Consolas} %设置英文字体 Monaco, Consolas,  Fantasque Sans Mono
%\setmainfont{Times New Roman}
%\setCJKmainfont{STZHONGS.TTF}
%\setmonofont{Consolas}
% \newfontfamily{\consolas}{YaHeiConsolas.ttf}
\newfontfamily{\monaco}{MONACO.TTF}
\setCJKmainfont{STZHONGS.TTF}
%\setmainfont{MONACO.TTF}
%\setsansfont{MONACO.TTF}

\newcommand{\fic}[1]{\begin{figure}[H]
		\center
		\includegraphics[width=0.8\textwidth]{#1}
	\end{figure}}
	
\newcommand{\sizedfic}[2]{\begin{figure}[H]
		\center
		\includegraphics[width=#1\textwidth]{#2}
	\end{figure}}

\newcommand{\codefile}[1]{\lstinputlisting{#1}}

\newcommand{\interval}{\vspace{0.5em}}

\newcommand{\tablestart}{
	\interval
	\begin{longtable}{p{2cm}p{10cm}}
	\hline}
\newcommand{\tableend}{
	\hline
	\end{longtable}
	\interval}

% 改变段间隔
\setlength{\parskip}{0.2em}
\linespread{1.1}

\usepackage{lastpage}
\usepackage{fancyhdr}
\pagestyle{fancy}
\lhead{\space \qquad \space}
\chead{resize虚拟机的流程 \qquad}
\rhead{\qquad\thepage/\pageref{LastPage}}

\begin{document}

\tableofcontents

\clearpage

\section{分析resize流程前的必要知识}
\subsection{nova中的RPC机制}
\subsection{重要的数据类型}
\subsubsection{req}
\subsubsection{context}
	\begin{lstlisting}
	# 根据req创建环境上下文context
	# context是nova/context.py中的RequestContext类
	context = req.environ["nova.context"]
	\end{lstlisting}

\subsubsection{instance}
	\begin{lstlisting}
	# 根据req和instance_id创建instance
	# instance是nova/context/instance.py中的Instance类
	instance = self._get_server(context, req, instance_id)	


	instance_type
	flavor_id
	deltas
	quotas
	vm_state
	\end{lstlisting}

\section{nova-api阶段}
	入口函数为:
	\begin{lstlisting}
	# 这个函数在nova/api/openstack/compute/servers.py
	def _resize(self, req, instance_id, flavor_id, **kwargs):
		...
        try:
			# compute_api是nova/compute/api.py中的API类
            self.compute_api.resize(context, instance, flavor_id, **kwargs)
        ...
	\end{lstlisting}

	进一步看API.resize()函数:
	\begin{lstlisting}
	# nova/compute/api.py API.resize()
	def resize(self, context, instance, flavor_id=None,
               **extra_instance_updates):
		...

		# filter_properties与选择本地扩容或选择异地扩容有关
        filter_properties = {'ignore_hosts': []}

        if not CONF.allow_resize_to_same_host:
            filter_properties['ignore_hosts'].append(instance['host'])

        if (not flavor_id and not CONF.allow_migrate_to_same_host):
            filter_properties['ignore_hosts'].append(instance['host'])
		
		...

		# scheduler_hint挺重要的,是nova-scheduler的参数
        scheduler_hint = {'filter_properties': filter_properties}
        self.compute_task_api.resize_instance(context, instance,
                extra_instance_updates, scheduler_hint=scheduler_hint,
                flavor=new_instance_type,
                reservations=quotas.reservations or [])
	\end{lstlisting}

	\begin{lstlisting}
	# nova/conductor/api.py ComputeTaskAPI.resize_instance()
	def resize_instance(self, context, instance, extra_instance_updates,
                        scheduler_hint, flavor, reservations):
        self.conductor_compute_rpcapi.migrate_server(
            context, instance, scheduler_hint, False, False, flavor,
            None, None, reservations)
	\end{lstlisting}

	\begin{lstlisting}
	# nova/conductor/rpcapi.py ComputeTaskAPI.migrate_server()
	def migrate_server(self, context, instance, scheduler_hint, live, rebuild,
                  flavor, block_migration, disk_over_commit,
                  reservations=None):
		...
		cctxt = self.client.prepare(version=version)
        return cctxt.call(context, 'migrate_server',
                          instance=instance, scheduler_hint=scheduler_hint,
                          live=live, rebuild=rebuild, flavor=flavor_p,
                          block_migration=block_migration,
                          disk_over_commit=disk_over_commit,
                          reservations=reservations)
	\end{lstlisting}

\section{nova-conductor部分}

	\begin{lstlisting}
	# nova/conductor/manager.py ComputeTaskManager.migrate_server()
	def migrate_server(self, context, instance, scheduler_hint, live, rebuild,
            flavor, block_migration, disk_over_commit, reservations=None):
        ...
        if live and not rebuild and not flavor:
            self._live_migrate(context, instance, scheduler_hint,
                               block_migration, disk_over_commit)
        elif not live and not rebuild and flavor:
            ...
            with compute_utils.EventReporter(context, 'cold_migrate',
                                             instance_uuid):
                self._cold_migrate(context, instance, flavor,
                                   scheduler_hint['filter_properties'],
                                   reservations)
        ...
	\end{lstlisting}

\section{冷迁移}

\subsection{冷迁移中的nova-conductor部分}

	\begin{lstlisting}
	# nova/conductor/manager.py ComputeTaskManager._cold_migrate()
	def _cold_migrate(self, context, instance, flavor, filter_properties,
                      reservations):
        ...
        try:
			...
			# 选择目的主机
            hosts = self.scheduler_client.select_destinations(
                    context, request_spec, filter_properties)
            host_state = hosts[0]
        ...

        try:
            ...
            (host, node) = (host_state['host'], host_state['nodename'])
            self.compute_rpcapi.prep_resize(
                context, image, instance,
                flavor, host,
                reservations, request_spec=request_spec,
                filter_properties=filter_properties, node=node)
        ...
	\end{lstlisting}

	\begin{lstlisting}
	# nova/compute/rpcapi.py ComputeAPI.prep_resize()
	def prep_resize(self, ctxt, image, instance, instance_type, host,
                    reservations=None, request_spec=None,
                    filter_properties=None, node=None):
        ...
		cctxt = self.client.prepare(server=host, version=version)
        cctxt.cast(ctxt, 'prep_resize',
                   instance=instance,
                   instance_type=instance_type_p,
                   image=image_p, reservations=reservations,
                   request_spec=request_spec,
                   filter_properties=filter_properties,
                   node=node)
	\end{lstlisting}

\subsection{冷迁移中的nova-compute部分}
\subsubsection{目的主机上的操作:prep\_resize}
	\begin{lstlisting}
	# nova/compute/manager.py ComputeManager.prep_resize()
	def prep_resize(self, context, image, instance, instance_type,
                    reservations, request_spec, filter_properties, node):
        ...
        with self._error_out_instance_on_exception(context, instance,
                                                   quotas=quotas):
            ...
            try:
                self._prep_resize(context, image, instance,
                                  instance_type, quotas,
                                  request_spec, filter_properties,
                                  node)
			...
	\end{lstlisting}

	\begin{lstlisting}
	# nova/compute/manager.py ComputeManager._prep_resize()
	def _prep_resize(self, context, image, instance, instance_type,
            quotas, request_spec, filter_properties, node):

        ...
        with rt.resize_claim(context, instance, instance_type,
                             image_meta=image, limits=limits) as claim:
			...
            self.compute_rpcapi.resize_instance(
                    context, instance, claim.migration, image,
                    instance_type, quotas.reservations)
	\end{lstlisting}

	\begin{lstlisting}
	# nova/compute/rpcapi.py ComputeAPI.resize_instance()
	def resize_instance(self, ctxt, instance, migration, image, instance_type,
                        reservations=None):
        ...
        cctxt = self.client.prepare(server=_compute_host(None, instance),
                version=version)
        cctxt.cast(ctxt, 'resize_instance',
                   instance=instance, migration=migration,
                   image=image, reservations=reservations,
                   instance_type=instance_type_p)
	\end{lstlisting}

\subsubsection{源主机的操作:resize\_instance}
	\begin{lstlisting}
	# nova/compute/manager.py ComputeManager.resize_instance()
	def resize_instance(self, context, instance, image,
                        reservations, migration, instance_type,
                        clean_shutdown=True):
		...
        with self._error_out_instance_on_exception(context, instance,
                                                   quotas=quotas):
			...
			# 获得虚拟机块设备的信息
			block_device_info = self._get_instance_block_device_info(
                                context, instance, bdms=bdms)
			
			# 关闭虚拟机并迁移虚拟机的增量文件
            disk_info = self.driver.migrate_disk_and_power_off(
                    context, instance, migration.dest_host,
                    instance_type, network_info,
                    block_device_info,
                    timeout, retry_interval)
			...
            self.compute_rpcapi.finish_resize(context, instance,
                    migration, image, disk_info,
                    migration.dest_compute, reservations=quotas.reservations)
            ...
	\end{lstlisting}

	migrate\_disk\_and\_power\_off()是源主机上将虚拟机迁移给目的主机的实现函数,主要利用了libvirt API。
	这个函数的分析在《nova调用libvirt》中的“nova扩容时对libvirt的调用”一节。

	\begin{lstlisting}
	# nova/compute/rpcapi.py(690) ComputeAPI.finish_resize()
	def finish_resize(self, ctxt, instance, migration, image, disk_info,
            host, reservations=None):
		...
        cctxt = self.client.prepare(server=host, version=version)
        cctxt.cast(ctxt, 'finish_resize',
                   instance=instance, migration=migration,
                   image=image, disk_info=disk_info, reservations=reservations)
	\end{lstlisting}

\subsubsection{目的主机上的操作:finish\_resize}
	\begin{lstlisting}
	# nova/compute/manager.py ComputeManager.finish_resize()
	def finish_resize(self, context, disk_info, image, instance,
                      reservations, migration):
        quotas = quotas_obj.Quotas.from_reservations(context,
                                                     reservations,
                                                     instance=instance)
        try:
            self._finish_resize(context, instance, migration,
                                disk_info, image)
        ...
	\end{lstlisting}

	\begin{lstlisting}
	# nova/compute/manager.py ComputeManager._finish_resize()
	def _finish_resize(self, context, instance, migration, disk_info,
                       image):
        ...
        try:
            self.driver.finish_migration(context, migration, instance,
                                         disk_info,
                                         network_info,
                                         image, resize_instance,
                                         block_device_info, power_on)
        ...
	\end{lstlisting}

\section{在线迁移}
\subsection{在线迁移中的nova-conductor部分}
    \begin{lstlisting}
    # nova/conductor/manager.py ComputeTaskManager._live_migrate()
    def _live_migrate(self, context, instance, scheduler_hint,
                      block_migration, disk_over_commit):
        destination = scheduler_hint.get("host")
        try:
            live_migrate.execute(context, instance, destination,
                             block_migration, disk_over_commit)
        ...
    \end{lstlisting}

    \begin{lstlisting}
    # nova/conductor/tasks/live_migrate.py execute()
    def execute(context, instance, destination,
            block_migration, disk_over_commit):
    task = LiveMigrationTask(context, instance,
                             destination,
                             block_migration,
                             disk_over_commit)
    # TODO(johngarbutt) create a superclass that contains a safe_execute call
    return task.execute()
    \end{lstlisting}

    \begin{lstlisting}
    # nova/conductor/tasks/live_migrate.py LiveMigrationTask.execute()
    def execute(self):
        self._check_instance_is_running()
        self._check_host_is_up(self.source)

        if not self.destination:
            self.destination = self._find_destination()
        else:
            self._check_requested_destination()

        # TODO(johngarbutt) need to move complexity out of compute manager
        # TODO(johngarbutt) disk_over_commit?
        return self.compute_rpcapi.live_migration(self.context,
                host=self.source,
                instance=self.instance,
                dest=self.destination,
                block_migration=self.block_migration,
                migrate_data=self.migrate_data)
    \end{lstlisting}

    \begin{lstlisting}
    # nova/compute/rpcapi.py ComputeAPI.live_migration()
    def live_migration(self, ctxt, instance, dest, block_migration, host,
                       migrate_data=None):
        ...
        cctxt = self.client.prepare(server=host, version=version)
        cctxt.cast(ctxt, 'live_migration', instance=instance,
                   dest=dest, block_migration=block_migration,
                   migrate_data=migrate_data)
    \end{lstlisting}

\subsection{在线迁移中的nova-compute部分}
    \begin{lstlisting}
    # nova/compute/manager.py ComputeManager.live_migration()
    def live_migration(self, context, dest, instance, block_migration,
                       migrate_data):
        """Executing live migration.

        :param context: security context
        :param instance: a nova.objects.instance.Instance object
        :param dest: destination host
        :param block_migration: if true, prepare for block migration
        :param migrate_data: implementation specific params

        """

        # NOTE(danms): since instance is not the first parameter, we can't
        # use @object_compat on this method. Since this is the only example,
        # we do this manually instead of complicating the decorator
        if not isinstance(instance, obj_base.NovaObject):
            expected = ['metadata', 'system_metadata',
                        'security_groups', 'info_cache']
            instance = objects.Instance._from_db_object(
                context, objects.Instance(), instance,
                expected_attrs=expected)

        # Create a local copy since we'll be modifying the dictionary
        migrate_data = dict(migrate_data or {})
        try:
            if block_migration:
                block_device_info = self._get_instance_block_device_info(
                    context, instance)
                disk = self.driver.get_instance_disk_info(
                    instance.name, block_device_info=block_device_info)
            else:
                disk = None

            pre_migration_data = self.compute_rpcapi.pre_live_migration(
                context, instance,
                block_migration, disk, dest, migrate_data)
            migrate_data['pre_live_migration_result'] = pre_migration_data

        except Exception:
            with excutils.save_and_reraise_exception():
                LOG.exception(_LE('Pre live migration failed at %s'),
                              dest, instance=instance)
                self._rollback_live_migration(context, instance, dest,
                                              block_migration, migrate_data)

        # Executing live migration
        # live_migration might raises exceptions, but
        # nothing must be recovered in this version.
        self.driver.live_migration(context, instance, dest,
                                   self._post_live_migration,
                                   self._rollback_live_migration,
                                   block_migration, migrate_data)
    \end{lstlisting}

\subsubsection{目的主机上的操作:pre\_live\_migration}
    \begin{lstlisting}
    # nova/compute/manager.py ComputeManager.pre_live_migration()
    def pre_live_migration(self, context, instance, block_migration, disk,
                           migrate_data):
        """Preparations for live migration at dest host.

        :param context: security context
        :param instance: dict of instance data
        :param block_migration: if true, prepare for block migration
        :param migrate_data: if not None, it is a dict which holds data
                             required for live migration without shared
                             storage.

        """
        block_device_info = self._get_instance_block_device_info(
                            context, instance, refresh_conn_info=True)

        network_info = self._get_instance_nw_info(context, instance)
        self._notify_about_instance_usage(
                     context, instance, "live_migration.pre.start",
                     network_info=network_info)

        pre_live_migration_data = self.driver.pre_live_migration(context,
                                       instance,
                                       block_device_info,
                                       network_info,
                                       disk,
                                       migrate_data)

        # NOTE(tr3buchet): setup networks on destination host
        self.network_api.setup_networks_on_host(context, instance,
                                                         self.host)

        # Creating filters to hypervisors and firewalls.
        # An example is that nova-instance-instance-xxx,
        # which is written to libvirt.xml(Check "virsh nwfilter-list")
        # This nwfilter is necessary on the destination host.
        # In addition, this method is creating filtering rule
        # onto destination host.
        self.driver.ensure_filtering_rules_for_instance(instance,
                                            network_info)

        self._notify_about_instance_usage(
                     context, instance, "live_migration.pre.end",
                     network_info=network_info)

        return pre_live_migration_data
    \end{lstlisting}

    \begin{lstlisting}
    # nova/virt/libvirt/driver.py LibvirtDriver.pre_live_migration()
    def pre_live_migration(self, context, instance, block_device_info,
                           network_info, disk_info, migrate_data=None):
        """Preparation live migration."""
        # Steps for volume backed instance live migration w/o shared storage.
        is_shared_block_storage = True
        is_shared_instance_path = True
        is_block_migration = True
        instance_relative_path = None
        if migrate_data:
            is_shared_block_storage = migrate_data.get(
                    'is_shared_block_storage', True)
            is_shared_instance_path = migrate_data.get(
                    'is_shared_instance_path', True)
            is_block_migration = migrate_data.get('block_migration', True)
            instance_relative_path = migrate_data.get('instance_relative_path')

        if not (is_shared_instance_path and is_shared_block_storage):
            # NOTE(dims): Using config drive with iso format does not work
            # because of a bug in libvirt with read only devices. However
            # one can use vfat as config_drive_format which works fine.
            # Please see bug/1246201 for details on the libvirt bug.
            if CONF.config_drive_format != 'vfat':
                if configdrive.required_by(instance):
                    raise exception.NoLiveMigrationForConfigDriveInLibVirt()

        if not is_shared_instance_path:
            # NOTE(mikal): this doesn't use libvirt_utils.get_instance_path
            # because we are ensuring that the same instance directory name
            # is used as was at the source
            if instance_relative_path:
                instance_dir = os.path.join(CONF.instances_path,
                                            instance_relative_path)
            else:
                instance_dir = libvirt_utils.get_instance_path(instance)

            if os.path.exists(instance_dir):
                raise exception.DestinationDiskExists(path=instance_dir)
            os.mkdir(instance_dir)

            if not is_shared_block_storage:
                # Ensure images and backing files are present.
                self._create_images_and_backing(context, instance,
                                                instance_dir, disk_info)

        if not (is_block_migration or is_shared_instance_path):
            # NOTE(angdraug): when block storage is shared between source and
            # destination and instance path isn't (e.g. volume backed or rbd
            # backed instance), instance path on destination has to be prepared

            # Touch the console.log file, required by libvirt.
            console_file = self._get_console_log_path(instance)
            libvirt_utils.file_open(console_file, 'a').close()

            # if image has kernel and ramdisk, just download
            # following normal way.
            self._fetch_instance_kernel_ramdisk(context, instance)

        # Establishing connection to volume server.
        block_device_mapping = driver.block_device_info_get_mapping(
            block_device_info)
        for vol in block_device_mapping:
            connection_info = vol['connection_info']
            disk_info = blockinfo.get_info_from_bdm(
                CONF.libvirt.virt_type, vol)
            self._connect_volume(connection_info, disk_info)

        if is_block_migration and len(block_device_mapping):
            # NOTE(stpierre): if this instance has mapped volumes,
            # we can't do a block migration, since that will
            # result in volumes being copied from themselves to
            # themselves, which is a recipe for disaster.
            LOG.error(
                _LE('Cannot block migrate instance %s with mapped volumes') %
                instance.uuid)
            raise exception.MigrationError(
                _('Cannot block migrate instance %s with mapped volumes') %
                instance.uuid)

        # We call plug_vifs before the compute manager calls
        # ensure_filtering_rules_for_instance, to ensure bridge is set up
        # Retry operation is necessary because continuously request comes,
        # concurrent request occurs to iptables, then it complains.
        max_retry = CONF.live_migration_retry_count
        for cnt in range(max_retry):
            try:
                self.plug_vifs(instance, network_info)
                break
            except processutils.ProcessExecutionError:
                if cnt == max_retry - 1:
                    raise
                else:
                    LOG.warn(_LW('plug_vifs() failed %(cnt)d. Retry up to '
                                 '%(max_retry)d.'),
                             {'cnt': cnt,
                              'max_retry': max_retry},
                             instance=instance)
                    greenthread.sleep(1)

        res_data = {'graphics_listen_addrs': {}}
        res_data['graphics_listen_addrs']['vnc'] = CONF.vncserver_listen
        res_data['graphics_listen_addrs']['spice'] = CONF.spice.server_listen

        return res_data
    \end{lstlisting}

\subsubsection{源主机上的操作:live\_migration}
    \begin{lstlisting}
    # nova/compute/manager.py ComputeManager.live_migration()
    def live_migration(self, context, dest, instance, block_migration,
                       migrate_data):
        """Executing live migration.

        :param context: security context
        :param instance: a nova.objects.instance.Instance object
        :param dest: destination host
        :param block_migration: if true, prepare for block migration
        :param migrate_data: implementation specific params

        """

        # NOTE(danms): since instance is not the first parameter, we can't
        # use @object_compat on this method. Since this is the only example,
        # we do this manually instead of complicating the decorator
        if not isinstance(instance, obj_base.NovaObject):
            expected = ['metadata', 'system_metadata',
                        'security_groups', 'info_cache']
            instance = objects.Instance._from_db_object(
                context, objects.Instance(), instance,
                expected_attrs=expected)

        # Create a local copy since we'll be modifying the dictionary
        migrate_data = dict(migrate_data or {})
        try:
            if block_migration:
                block_device_info = self._get_instance_block_device_info(
                    context, instance)
                disk = self.driver.get_instance_disk_info(
                    instance.name, block_device_info=block_device_info)
            else:
                disk = None

            pre_migration_data = self.compute_rpcapi.pre_live_migration(
                context, instance,
                block_migration, disk, dest, migrate_data)
            migrate_data['pre_live_migration_result'] = pre_migration_data

        except Exception:
            with excutils.save_and_reraise_exception():
                LOG.exception(_LE('Pre live migration failed at %s'),
                              dest, instance=instance)
                self._rollback_live_migration(context, instance, dest,
                                              block_migration, migrate_data)

        # Executing live migration
        # live_migration might raises exceptions, but
        # nothing must be recovered in this version.
        self.driver.live_migration(context, instance, dest,
                                   self._post_live_migration,
                                   self._rollback_live_migration,
                                   block_migration, migrate_data)
    \end{lstlisting}

    \begin{lstlisting}
    # nova/virt/libvirt/driver.py LibvirtDriver.live_migration()
    def live_migration(self, context, instance, dest,
                       post_method, recover_method, block_migration=False,
                       migrate_data=None):
        """Spawning live_migration operation for distributing high-load.

        :param context: security context
        :param instance:
            nova.db.sqlalchemy.models.Instance object
            instance object that is migrated.
        :param dest: destination host
        :param post_method:
            post operation method.
            expected nova.compute.manager._post_live_migration.
        :param recover_method:
            recovery method when any exception occurs.
            expected nova.compute.manager._rollback_live_migration.
        :param block_migration: if true, do block migration.
        :param migrate_data: implementation specific params

        """

        # 'dest' will be substituted into 'migration_uri' so ensure
        # it does't contain any characters that could be used to
        # exploit the URI accepted by libivrt
        if not libvirt_utils.is_valid_hostname(dest):
            raise exception.InvalidHostname(hostname=dest)

        greenthread.spawn(self._live_migration, context, instance, dest,
                          post_method, recover_method, block_migration,
                          migrate_data)
    \end{lstlisting}

    \begin{lstlisting}
    # nova/virt/libvirt/driver.py LibvirtDriver._live_migration()
    def _live_migration(self, context, instance, dest, post_method,
                        recover_method, block_migration=False,
                        migrate_data=None):
        """Do live migration.

        :param context: security context
        :param instance:
            nova.db.sqlalchemy.models.Instance object
            instance object that is migrated.
        :param dest: destination host
        :param post_method:
            post operation method.
            expected nova.compute.manager._post_live_migration.
        :param recover_method:
            recovery method when any exception occurs.
            expected nova.compute.manager._rollback_live_migration.
        :param block_migration: if true, do block migration.
        :param migrate_data: implementation specific params
        """

        # Do live migration.
        try:
            if block_migration:
                flaglist = CONF.libvirt.block_migration_flag.split(',')
            else:
                flaglist = CONF.libvirt.live_migration_flag.split(',')
            flagvals = [getattr(libvirt, x.strip()) for x in flaglist]
            logical_sum = reduce(lambda x, y: x | y, flagvals)

            dom = self._lookup_by_name(instance["name"])

            pre_live_migrate_data = (migrate_data or {}).get(
                                        'pre_live_migration_result', {})
            listen_addrs = pre_live_migrate_data.get('graphics_listen_addrs')

            migratable_flag = getattr(libvirt, 'VIR_DOMAIN_XML_MIGRATABLE',
                                      None)

            if migratable_flag is None or listen_addrs is None:
                self._check_graphics_addresses_can_live_migrate(listen_addrs)
                dom.migrateToURI(CONF.libvirt.live_migration_uri % dest,
                                 logical_sum,
                                 None,
                                 CONF.libvirt.live_migration_bandwidth)
            else:
                old_xml_str = dom.XMLDesc(migratable_flag)
                new_xml_str = self._correct_listen_addr(old_xml_str,
                                                        listen_addrs)
                try:
                    dom.migrateToURI2(CONF.libvirt.live_migration_uri % dest,
                                      None,
                                      new_xml_str,
                                      logical_sum,
                                      None,
                                      CONF.libvirt.live_migration_bandwidth)
                except libvirt.libvirtError as ex:
                    # NOTE(mriedem): There is a bug in older versions of
                    # libvirt where the VIR_DOMAIN_XML_MIGRATABLE flag causes
                    # virDomainDefCheckABIStability to not compare the source
                    # and target domain xml's correctly for the CPU model.
                    # We try to handle that error here and attempt the legacy
                    # migrateToURI path, which could fail if the console
                    # addresses are not correct, but in that case we have the
                    # _check_graphics_addresses_can_live_migrate check in place
                    # to catch it.
                    # TODO(mriedem): Remove this workaround when
                    # Red Hat BZ #1141838 is closed.
                    error_code = ex.get_error_code()
                    if error_code == libvirt.VIR_ERR_CONFIG_UNSUPPORTED:
                        LOG.warn(_LW('An error occurred trying to live '
                                     'migrate. Falling back to legacy live '
                                     'migrate flow. Error: %s'), ex,
                                 instance=instance)
                        self._check_graphics_addresses_can_live_migrate(
                            listen_addrs)
                        dom.migrateToURI(
                            CONF.libvirt.live_migration_uri % dest,
                            logical_sum,
                            None,
                            CONF.libvirt.live_migration_bandwidth)
                    else:
                        raise

        except Exception as e:
            with excutils.save_and_reraise_exception():
                LOG.error(_LE("Live Migration failure: %s"), e,
                          instance=instance)
                recover_method(context, instance, dest, block_migration)

        post_method(context, instance, dest, block_migration,
                    migrate_data)
    \end{lstlisting}

\end{document}