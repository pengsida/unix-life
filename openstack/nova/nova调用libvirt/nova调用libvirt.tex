% !TeX spellcheck = en_US
%% 字体:方正静蕾简体
%%		 方正粗宋
\documentclass[a4paper,left=1.5cm,right=1.5cm,11pt]{article}

\usepackage[utf8]{inputenc}
\usepackage{fontspec}
\usepackage{cite}
\usepackage{xeCJK}
\usepackage{indentfirst}
\usepackage{titlesec}
\usepackage{etoolbox}%
\makeatletter
\patchcmd{\ttlh@hang}{\parindent\z@}{\parindent\z@\leavevmode}{}{}%
\patchcmd{\ttlh@hang}{\noindent}{}{}{}%
\makeatother

\usepackage{longtable}
\usepackage{empheq}
\usepackage{graphicx}
\usepackage{float}
\usepackage{rotating}
\usepackage{subfigure}
\usepackage{tabu}
\usepackage{amsmath}
\usepackage{setspace}
\usepackage{amsfonts}
\usepackage{appendix}
\usepackage{listings}
\usepackage{xcolor}
\usepackage{geometry}
\setcounter{secnumdepth}{4}
%\titleformat*{\section}{\LARGE}
%\renewcommand\refname{参考文献}
%\titleformat{\chapter}{\centering\bfseries\huge}{}{0.7em}{}{}
\titleformat{\section}{\LARGE\bf}{\thesection}{1em}{}{}
\titleformat{\subsection}{\Large\bfseries}{\thesubsection}{1em}{}{}
\titleformat{\subsubsection}{\large\bfseries}{\thesubsubsection}{1em}{}{}
\renewcommand{\contentsname}{{ \centerline{目{  } 录}}}
\setCJKfamilyfont{cjkhwxk}{STXINGKA.TTF}
%\setCJKfamilyfont{cjkhwxk}{华文行楷}
%\setCJKfamilyfont{cjkfzcs}{方正粗宋简体}
%\newcommand*{\cjkfzcs}{\CJKfamily{cjkfzcs}}
\newcommand*{\cjkhwxk}{\CJKfamily{cjkhwxk}}
%\newfontfamily\wryh{Microsoft YaHei}
%\newfontfamily\hwzs{华文中宋}
%\newfontfamily\hwst{华文宋体}
%\newfontfamily\hwfs{华文仿宋}
%\newfontfamily\jljt{方正静蕾简体}
%\newfontfamily\hwxk{华文行楷}
\newcommand{\verylarge}{\fontsize{60pt}{\baselineskip}\selectfont}  
\newcommand{\chuhao}{\fontsize{44.9pt}{\baselineskip}\selectfont}  
\newcommand{\xiaochu}{\fontsize{38.5pt}{\baselineskip}\selectfont}  
\newcommand{\yihao}{\fontsize{27.8pt}{\baselineskip}\selectfont}  
\newcommand{\xiaoyi}{\fontsize{25.7pt}{\baselineskip}\selectfont}  
\newcommand{\erhao}{\fontsize{23.5pt}{\baselineskip}\selectfont}  
\newcommand{\xiaoerhao}{\fontsize{19.3pt}{\baselineskip}\selectfont} 
\newcommand{\sihao}{\fontsize{14pt}{\baselineskip}\selectfont}      % 字号设置  
\newcommand{\xiaosihao}{\fontsize{12pt}{\baselineskip}\selectfont}  % 字号设置  
\newcommand{\wuhao}{\fontsize{10.5pt}{\baselineskip}\selectfont}    % 字号设置  
\newcommand{\xiaowuhao}{\fontsize{9pt}{\baselineskip}\selectfont}   % 字号设置  
\newcommand{\liuhao}{\fontsize{7.875pt}{\baselineskip}\selectfont}  % 字号设置  
\newcommand{\qihao}{\fontsize{5.25pt}{\baselineskip}\selectfont}    % 字号设置 

\usepackage{diagbox}
\usepackage{multirow}
\boldmath
\XeTeXlinebreaklocale "zh"
\XeTeXlinebreakskip = 0pt plus 1pt minus 0.1pt
\definecolor{cred}{rgb}{0.8,0.8,0.8}
\definecolor{cgreen}{rgb}{0,0.3,0}
\definecolor{cpurple}{rgb}{0.5,0,0.35}
\definecolor{cdocblue}{rgb}{0,0,0.3}
\definecolor{cdark}{rgb}{0.95,1.0,1.0}
\lstset{
	language=python,
	numbers=left,
	numberstyle=\tiny\color{black},
	showspaces=false,
	showstringspaces=false,
	basicstyle=\scriptsize,
	keywordstyle=\color{purple},
	commentstyle=\itshape\color{cgreen},
	stringstyle=\color{blue},
	frame=lines,
	% escapeinside=``,
	extendedchars=true, 
	xleftmargin=1em,
	xrightmargin=1em, 
	backgroundcolor=\color{cred},
	aboveskip=1em,
	breaklines=true,
	tabsize=4
} 

%\newfontfamily{\consolas}{Consolas}
%\newfontfamily{\monaco}{Monaco}
%\setmonofont[Mapping={}]{Consolas}	%英文引号之类的正常显示,相当于设置英文字体
%\setsansfont{Consolas} %设置英文字体 Monaco, Consolas,  Fantasque Sans Mono
%\setmainfont{Times New Roman}
%\setCJKmainfont{STZHONGS.TTF}
%\setmonofont{Consolas}
% \newfontfamily{\consolas}{YaHeiConsolas.ttf}
\newfontfamily{\monaco}{MONACO.TTF}
\setCJKmainfont{STZHONGS.TTF}
%\setmainfont{MONACO.TTF}
%\setsansfont{MONACO.TTF}

\newcommand{\fic}[1]{\begin{figure}[H]
		\center
		\includegraphics[width=0.8\textwidth]{#1}
	\end{figure}}
	
\newcommand{\sizedfic}[2]{\begin{figure}[H]
		\center
		\includegraphics[width=#1\textwidth]{#2}
	\end{figure}}

\newcommand{\codefile}[1]{\lstinputlisting{#1}}

\newcommand{\interval}{\vspace{0.5em}}

\newcommand{\tablestart}{
	\interval
	\begin{longtable}{p{2cm}p{10cm}}
	\hline}
\newcommand{\tableend}{
	\hline
	\end{longtable}
	\interval}

% 改变段间隔
\setlength{\parskip}{0.2em}
\linespread{1.1}

\usepackage{lastpage}
\usepackage{fancyhdr}
\pagestyle{fancy}
\lhead{\space \qquad \space}
\chead{nova调用libvirt \qquad}
\rhead{\qquad\thepage/\pageref{LastPage}}

\begin{document}

\tableofcontents

\clearpage

\section{前言}
	在nova/virt/libvirt/driver.py中有LibvirtDriver类,用于调用libvirt。

\section{nova创建虚拟机时对libvirt的调用}
	LibvirtDriver类中创建虚拟机的函数代码如下:
	\begin{lstlisting}
	def spawn(self, context, instance, image_meta, injected_files,
              admin_password, network_info=None, block_device_info=None):
		
		# 获取disk配置信息 
        disk_info = blockinfo.get_disk_info(CONF.libvirt.virt_type,
                                            instance,
                                            image_meta,
                                            block_device_info)

        ...

		# 创建镜像
        self._create_image(context, instance, disk_info['mapping'],
                           injection_info=injection_info,
                           block_device_info=block_device_info)

        ...

		# 创建xml配置文件
        xml = self._get_guest_xml(context, instance, network_info,
                                  disk_info, image_meta,
                                  block_device_info=block_device_info)

		...

		# 创建虚拟机和网络
        self._create_domain_and_network(
            context, xml, instance, network_info, disk_info,
            block_device_info=block_device_info,
            post_xml_callback=gen_confdrive,
            destroy_disks_on_failure=True)
	\end{lstlisting}

\clearpage

\subsection{创建虚拟机的函数}
	LibvirtDriver类中调用libvirt创建虚拟机和网络的函数代码如下,因为我更关注创建虚拟机的流程,所以把与创建网络有关的代码略去:
	\begin{lstlisting}
	def _create_domain_and_network(self, context, xml, instance, network_info,
                                   disk_info, block_device_info=None,
                                   power_on=True, reboot=False,
                                   vifs_already_plugged=False,
                                   post_xml_callback=None,
                                   destroy_disks_on_failure=False):

        ...

		# 创建虚拟机
        guest = None
        try:
            with self.virtapi.wait_for_instance_event(
                    instance, events, deadline=timeout,
                    error_callback=self._neutron_failed_callback):
                ...
                with self._lxc_disk_handler(instance, instance.image_meta,
                                            block_device_info, disk_info):
                    guest = self._create_domain(
                        xml, pause=pause, power_on=power_on,
                        post_xml_callback=post_xml_callback)
					...
        ...

        return guest
	\end{lstlisting}

	创建虚拟机的函数代码如下:
	\begin{lstlisting}
	def _create_domain(self, xml=None, domain=None,
                       power_on=True, pause=False, post_xml_callback=None):
		
		# 创建虚拟机
		# 这里的libvirt_guest在driver.py中有引用:from nova.virt.libvirt import guest as libvirt_guest
        if xml:
            guest = libvirt_guest.Guest.create(xml, self._host)
        else:
            guest = libvirt_guest.Guest(domain)

        ...

        return guest
	\end{lstlisting}

	从上面的代码可以看出,nova/virt/libvirt/guest.py中的Guest类就是对虚拟机instance的抽象。\par

	因为我们这里是新建虚拟机,所以domain应该是None,是调用Guest.create()函数来创建虚拟机。
	下面来看Guest类的create()函数:
	\begin{lstlisting}
class Guest(object):
	@classmethod
    def create(cls, xml, host):
        
        try:
            ...
			# 这里使用host创建虚拟机,host是nova/virt/libvirt/host.py中的Host类
            guest = host.write_instance_config(xml)
        
		...

        return guest
	\end{lstlisting}

	这里的Host类的write\_instance\_config()的代码如下所示:
	\begin{lstlisting}
	def write_instance_config(self, xml):
		# 直接告诉我们,get_connection()函数返回的是与libvirt相关的对象
		# 下面这行代码可以让我们想起:virsh define demo.xml这个命令
		# 实际上self.get_connection()返回的是一个virConnect对象
        domain = self.get_connection().defineXML(xml)

        return libvirt_guest.Guest(domain)
	\end{lstlisting}

	如果想知道openstack怎么创建一个virConnect对象,就继续看get\_connection()函数:
	\begin{lstlisting}
	def get_connection(self):
        
        try:
			# 很清楚地看到conn对象是由_get_connection()创建的
            conn = self._get_connection()
        
		...

        return conn
	\end{lstlisting}

	进一步看\_get\_connection()函数:
	\begin{lstlisting}
	def _get_connection(self):
		# _wrapped_conn_lock是一个互斥锁
        with self._wrapped_conn_lock:
			...
			# _wrapped_conn是一个连接到libvirt的对象
            # 当前服务没有调用libvirt是,_wrapped_conn是None
            if self._wrapped_conn is None:
                try:
                    # 连接到libvirt并返回一个对象
                    self._wrapped_conn = self._get_new_connection()
                
				...

        return self._wrapped_conn
	\end{lstlisting}

	再来看\_get\_new\_connection()函数:
	\begin{lstlisting}
	def _get_new_connection(self):
        ...
		# 这里的uri是'qemu:///system',根据之前的代码追踪过程可以轻易得知,在此不再详述如何追踪到它的值
		# _read_only为False
		# 使用_connect()函数创建wrapped_conn对象
		# 如果熟悉libvirt API,那么看到这条语句会想到下面C语言中的这条语句:
		#	conn = virConnectPtr("qemu:///system")
		#	conn是virConnectPtr对象,python中是virConnect类
        wrapped_conn = self._connect(self._uri, self._read_only)

        ...

        return wrapped_conn
	\end{lstlisting}

	分析到这,应该算是比较清楚了。我们如果是熟悉virConnect对象,那么就能使用libvirt的API在openstack中呼风唤雨了。\par

\clearpage

\section{nova扩容时对libvirt的调用}
	LibvirtDriver类中创建虚拟机快照的函数是migrate\_disk\_and\_power\_off()。

	\begin{lstlisting}
	# nova/virt/libvirt/driver.py LibvirtDriver.migrate_disk_and_power_off()
	def migrate_disk_and_power_off(self, context, instance, dest,
                                   flavor, network_info,
                                   block_device_info=None,
                                   timeout=0, retry_interval=0):
		# 获取disk配置信息
		# block_device_info是一个字典,如下所示:
		# {'swap': swap,
		# 'root_device_name': root_device_name,
		# 'ephemerals': ephemerals,
		# 'block_device_mapping': block_device_mapping}
		# 
		# disk_info_text数据类型是一个list,里面有多个字典,字典格式如下
		# {'type': disk_type,
		# 'path': path,
		# 'virt_disk_size': virt_size,
		# 'backing_file': backing_file,
		# 'disk_size': dk_size,
		# 'over_committed_disk_size': over_commit_size}
        disk_info_text = self.get_instance_disk_info(instance['name'],
                block_device_info=block_device_info)

		# 调用json.loads()
		disk_info = jsonutils.loads(disk_info_text)

		# 调用os.path.join(CONF.instances_path, instance['name'])获得inst_base
        inst_base = libvirt_utils.get_instance_path(instance)
		# 创建resize以后instance后备文件的存放路径
        inst_base_resize = inst_base + "_resize"
		# 判断是否共用storage
        shared_storage = self._is_storage_shared_with(dest, inst_base)

		# 在目的主机下根据inst_base创建增量文件的文件夹
        if not shared_storage:
            utils.execute('ssh', dest, 'mkdir', '-p', inst_base)

		# 关闭虚拟机
        self.power_off(instance, timeout, retry_interval)

		# block_device_info是一个字典
		# 相当于调用block_device_info.get('block_device_mapping')
        block_device_mapping = driver.block_device_info_get_mapping(
            block_device_info)

		# 迁移前,通过特定类型的卷驱动卸载卷设备
        for vol in block_device_mapping:
            connection_info = vol['connection_info']
            disk_dev = vol['mount_device'].rpartition("/")[2]
            self._disconnect_volume(connection_info, disk_dev)

        try:
			# 重命名inst_base为inst_base_resize
            utils.execute('mv', inst_base, inst_base_resize)
            if shared_storage:
                dest = None
                utils.execute('mkdir', '-p', inst_base)

            active_flavor = flavors.extract_flavor(instance)

			# 迁移虚拟机磁盘内容
            for info in disk_info:
                # assume inst_base == dirname(info['path'])
                img_path = info['path']
                fname = os.path.basename(img_path)
                from_path = os.path.join(inst_base_resize, fname)

                if (fname == 'disk.swap' and
                    active_flavor.get('swap', 0) != flavor.get('swap', 0)):
                    # To properly resize the swap partition, it must be
                    # re-created with the proper size.  This is acceptable
                    # because when an OS is shut down, the contents of the
                    # swap space are just garbage, the OS doesn't bother about
                    # what is in it.

                    # We will not copy over the swap disk here, and rely on
                    # finish_migration/_create_image to re-create it for us.
                    continue

                on_execute = lambda process: self.job_tracker.add_job(
                    instance, process.pid)
                on_completion = lambda process: self.job_tracker.remove_job(
                    instance, process.pid)

				# 如果类型是qcow2并且有后备文件
                if info['type'] == 'qcow2' and info['backing_file']:
                    tmp_path = from_path + "_rbase"
                    # merge backing file
                    utils.execute('qemu-img', 'convert', '-f', 'qcow2',
                                  '-O', 'qcow2', from_path, tmp_path)

                    if shared_storage:
                        utils.execute('mv', tmp_path, img_path)
                    else:
                        libvirt_utils.copy_image(tmp_path, img_path, host=dest,
                                                 on_execute=on_execute,
                                                 on_completion=on_completion)
                        utils.execute('rm', '-f', tmp_path)
				# 如果类型是raw或者是没有后备文件的qcow2
                else:
                    libvirt_utils.copy_image(from_path, img_path, host=dest,
                                             on_execute=on_execute,
                                             on_completion=on_completion)
        ...
	\end{lstlisting}

\subsection{卸载卷设备}

\subsection{关闭虚拟机}
	相关语句如下:
	\begin{lstlisting}
	self.power_off(instance, timeout, retry_interval)
	\end{lstlisting}

\clearpage

\section{nova创建快照时对libvirt的调用}
	LibvirtDriver类中创建虚拟机快照的函数是snapshot()。\par

	这个函数主要功能的实现都与libvirt有关,我留取了一些主要的代码,并在之后逐个分析:
	\begin{lstlisting}
	def snapshot(self, context, instance, image_id, update_task_state):

        try:
			# 得到虚拟机的抽象对象virt_dom
            virt_dom = self._lookup_by_name(instance['name'])
        ...

		# 得到虚拟机的磁盘路径
        disk_path = libvirt_utils.find_disk(virt_dom)

		# 得到虚拟机的磁盘类型
        source_format = libvirt_utils.get_disk_type(disk_path)

		# 如果是静态创建快照,将会执行下面这段代码,稍后会在“静态创建快照”一节分析
		if CONF.libvirt.virt_type != 'lxc' and not live_snapshot:
            if state == power_state.RUNNING or state == power_state.PAUSED:
                ...
                virt_dom.managedSave(0)

		# snapshot_backend也将用于静态创建快照,稍后分析
        snapshot_backend = self.image_backend.snapshot(instance,
                disk_path,
                image_type=source_format)

        with utils.tempdir(dir=snapshot_directory) as tmpdir:
            try:
				...
                if live_snapshot:
					# 动态创建快照
                    self._live_snapshot(virt_dom, disk_path, out_path,
                                        image_format)
                else:
					# 静态创建快照
                    snapshot_backend.snapshot_extract(out_path, image_format)
            ...
	\end{lstlisting}

\clearpage

\subsection{获得虚拟机抽象对象}
	函数中的语句如下:
	\begin{lstlisting}
	# instance是Intance类,Instance类定义在nova/objects/instance.py中
	# 返回了virDomain对象
	virt_dom = self._lookup_by_name(instance['name'])
	\end{lstlisting}

	\_lookup\_by\_name()函数如下:
	\begin{lstlisting}
	def _lookup_by_name(self, instance_name):
        try:
			# 这里的_conn就是libvirt API中的virConnect对象
            return self._conn.lookupByName(instance_name)
        ...
	\end{lstlisting}

	lookupByName是libvirt的一个API函数,可以根据域的名字返回virDomain对象。

\subsection{获得虚拟机磁盘路径}
	函数中的语句如下:
	\begin{lstlisting}
	# libvirt_utils在driver.py中有引用:from nova.virt.libvirt import utils as libvirt_utils
	disk_path = libvirt_utils.find_disk(virt_dom)
	\end{lstlisting}

	find\_disk函数如下:
	\begin{lstlisting}
	def find_disk(virt_dom):
		# 在这里,我们只关心下面这条语句
		# virt_dom是libvirt的virDomain对象
		xml_desc = virt_dom.XMLDesc(0)
		...
	\end{lstlisting}

	这里的XMLDesc()是virDomain类的一个成员函数,用于提供域的XML配置文件。
	也就是说,获得虚拟机磁盘路径的核心函数是virDomain.XMLDesc()。\par

	虽然我这里省略了find\_disk()函数接下来的代码,但是大家应该也能知道,
	接下来就是解析这个xml文件,得到域的磁盘路径。

\subsection{获得虚拟机磁盘类型}
	函数中的语句如下:
	\begin{lstlisting}
	# libvirt_utils在driver.py中有引用:from nova.virt.libvirt import utils as libvirt_utils
	# 返回了虚拟机磁盘的类型
	source_format = libvirt_utils.get_disk_type(disk_path)
	\end{lstlisting}

	get\_disk\_type()函数如下:
	\begin{lstlisting}
	def get_disk_type(path):
    ...
	# 这个函数用于得到磁盘的类型
	# qemu_img_info()返回了是QemuImgInfo类
	# QemuImgInfo类定义于nova/openstack/common/imageutils.py中
	# QemuImgInfo.file_format存放着磁盘类型的信息
    return images.qemu_img_info(path).file_format
	\end{lstlisting}

	继续追踪qemu\_img\_info()函数:
	\begin{lstlisting}
	def qemu_img_info(path):
    ...
	# 这条语句相当于执行命令行:qemu-img info disk_path,返回镜像文件的信息
	# utils是nova/utils.py文件
    out, err = utils.execute('env', 'LC_ALL=C', 'LANG=C',
                             'qemu-img', 'info', path)
    ...
	# 返回QemuImgInfo类,这个类定义于nova/openstack/common/imageutils.py中
    return imageutils.QemuImgInfo(out)
	\end{lstlisting}

	所以说,openstack获得虚拟机磁盘信息,最终是调用了“qemu-img info <filename>”这条命令。

\subsection{静态创建快照}
	可以从snapshot()函数知道静态创建快照的相关语句如下:
	\begin{lstlisting}
	...
	if CONF.libvirt.virt_type != 'lxc' and not live_snapshot:
            if state == power_state.RUNNING or state == power_state.PAUSED:
                ...
				# 
                virt_dom.managedSave(0)
	# 获得与静态创建快照有关的数据结构
	snapshot_backend = self.image_backend.snapshot(instance,
                disk_path,
                image_type=source_format)
	...
	with utils.tempdir(dir=snapshot_directory) as tmpdir:
            try:
				...
                else:
					# 静态创建快照
                    snapshot_backend.snapshot_extract(out_path, image_format)
	\end{lstlisting}

\clearpage

	我们逐句分析这三句注释了的代码:
	\begin{itemize}
		\item[1.] 第一句注释代码:
		\begin{lstlisting}
	# virt_dom是libvirt的virDomain对象
	# virDomain.managedSave()函数用于将虚拟机的域挂起,然后将它的内存信息保存到虚拟机镜像的一个文件中
	virt_dom.managedSave(0)
		\end{lstlisting}

		其实第一步已经完成了对虚拟机创建快照,只是现在这个快照保存在镜像文件中,还需要将它提取出来。

		\item[2.] 第二句注释代码:
		\begin{lstlisting}
	# image_backend是nova/libvirt/imagebackend.py文件中定义的Backend类
	# snapshot()函数根据image_type返回相应的类
	snapshot_backend = self.image_backend.snapshot(instance,
                disk_path,
                image_type=source_format)
		\end{lstlisting}

		进一步看snapshot()函数:
		\begin{lstlisting}
	def snapshot(self, instance, disk_path, image_type=None):
		# self.backend()根据image_type返回相应的类
		# 具体细节在此不详述,有兴趣的可以看Backend类
        backend = self.backend(image_type)
        return backend(instance=instance, path=disk_path)
		\end{lstlisting}

		\item[3.] 在分析第三句注释代码前,我先说一些准备信息。\par
		我分析代码的时候,使用的磁盘类型是qcow2类型,所以现在snapshot\_backend是Qcow2类。
		而且现在我们已经通过managedSave()函数将它的快照保存到虚拟机镜像的一个文件中。
		所以使用snapshot\_extract()函数就很好理解了,就是提取快照文件:
		\begin{lstlisting}
	snapshot_backend.snapshot_extract(out_path, image_format)

	# snapshot_extract()函数如下
class Qcow2(Image):
	...
	def snapshot_extract(self, target, out_format):
        libvirt_utils.extract_snapshot(self.path, 'qcow2',
                                       target,
                                       out_format)
		\end{lstlisting}

\clearpage

		代码到了这一步,已经是即将完成快照的提取。
		进一步看extract\_snapshot()函数:
		\begin{lstlisting}
	def extract_snapshot(disk_path, source_fmt, out_path, dest_fmt):
		# 函数相当于执行了“qemu-img convert -f source_fmt -O dest_fmt disk_path out_path”
		# qemu-img convert就是用来提取快照的命令
		qemu_img_cmd = ('qemu-img', 'convert', '-f', source_fmt, '-O', dest_fmt)
		...
		qemu_img_cmd += (disk_path, out_path)
		execute(*qemu_img_cmd)
		\end{lstlisting}
	\end{itemize}

	总结一下,openstack静态创建快照步骤如下:
	\begin{itemize}
		\item[1.] 调用virDomain.managedSave()创建快照。
		\item[2.] 用“qemu-img convert”命令提取快照。
	\end{itemize}

\subsection{动态创建快照}
	snapshot()函数中动态创建快照的相关语句如下:
	\begin{lstlisting}
	...
	with utils.tempdir(dir=snapshot_directory) as tmpdir:
            try:
				...
                if live_snapshot:
					# 动态创建快照
                    self._live_snapshot(virt_dom, disk_path, out_path,
                                        image_format)
				...
	\end{lstlisting}

	进一步看\_live\_snapshot()函数:
	\begin{lstlisting}
	def _live_snapshot(self, domain, disk_path, out_path, image_format):
		# domain是虚拟机的virDomain对象

		# 传入VIR_DOMAIN_XML_INACTIVE和VIR_DOMAIN_XML_SECURE参数
		# 获得domain的xml配置文件
		# 这个xml用于之后的undefine()和define()函数
        xml = domain.XMLDesc(
            libvirt.VIR_DOMAIN_XML_INACTIVE |
            libvirt.VIR_DOMAIN_XML_SECURE)

        try:
			# 调用virDomain.blockJobAbort()停止活动块操作
            domain.blockJobAbort(disk_path, 0)
        ...
		# 获得虚拟机backing_file的路径
        src_back_path = libvirt_utils.get_disk_backing_file(disk_path,
                                                            basename=False)
        disk_delta = out_path + '.delta'
		# 这个函数位于nova/virt/libvirt/utils.py中
		# 函数中调用“qemu-img create -f qcow2 -o src_disk_size,src_back_path src_back_path”创建镜像文件
        libvirt_utils.create_cow_image(src_back_path, disk_delta,
                                       src_disk_size)

        try:
			# 因为virDomain.blockRebase()不能对persistent的domain操作
			# 所以调用virDomain.undefine()将domain变为transient
            if domain.isPersistent():
                domain.undefine()

            # 将虚拟机镜像文件的内容拷贝到disk_delta中的镜像文件中
            domain.blockRebase(disk_path, disk_delta, 0,
                               libvirt.VIR_DOMAIN_BLOCK_REBASE_COPY |
                               libvirt.VIR_DOMAIN_BLOCK_REBASE_REUSE_EXT |
                               libvirt.VIR_DOMAIN_BLOCK_REBASE_SHALLOW)

            ...

			# 调用virtDomain.blockJobAbort()终止数据拷贝
            domain.blockJobAbort(disk_path, 0)
            ...
        finally:
			# 将domain从transient变为persistent
            self._conn.defineXML(xml)

        # 这个函数调用“qemu-img convert”命令将disk_delta中的文件变为一个qcow2文件
        libvirt_utils.extract_snapshot(disk_delta, 'qcow2',
                                       out_path, image_format)
	\end{lstlisting}

	总结一下,openstack动态创建快照步骤如下:
	\begin{itemize}
		\item[1.] 使用“qemu-img create”命令创建用于备份的镜像。
		\item[2.] 调用virtDomain.blockRebase()函数将虚拟机的磁盘内容拷贝到备份镜像中。
		\item[3.] 使用"qemu-img convert"命令将备份镜像转换为一个qcow2文件。
	\end{itemize}

\clearpage

\section{openstack中对libvirt调用的框架}

\clearpage

\section{查看libvirt的python API}

	可以通过python的help()函数来查看virConnect类:
	\begin{itemize}
		\item[1.] 首先进入python的help界面:
		\sizedfic{0.6}{1.png}

		\item[2.] 查看libvirt的帮助文档:
		\sizedfic{0.6}{2.png}

		\item[3.] 通过“virConnect”找到对virConnect各个函数的介绍:
		\sizedfic{0.6}{3.png}

	\end{itemize}

	% 进一步看\_connect()函数:
	% \begin{lstlisting}
	% @staticmethod
    % def _connect(uri, read_only):
	% 	...
	% 	# 下面这个函数使用Evenlet机制会创建一个线程
    %     return tpool.proxy_call(
    %         (libvirt.virDomain, libvirt.virConnect),
    %         libvirt.openAuth, uri, auth, flags)
	% \end{lstlisting}

\end{document}