% !TeX spellcheck = en_US
%% 字体:方正静蕾简体
%%		 方正粗宋
\documentclass[a4paper,left=1.5cm,right=1.5cm,11pt]{article}

\usepackage[utf8]{inputenc}
\usepackage{fontspec}
\usepackage{cite}
\usepackage{xeCJK}
\usepackage{indentfirst}
\usepackage{titlesec}
\usepackage{etoolbox}%
\makeatletter
\patchcmd{\ttlh@hang}{\parindent\z@}{\parindent\z@\leavevmode}{}{}%
\patchcmd{\ttlh@hang}{\noindent}{}{}{}%
\makeatother
\usepackage{hyperref}
\usepackage{longtable}
\usepackage{empheq}
\usepackage{graphicx}
\usepackage{float}
\usepackage{rotating}
\usepackage{subfigure}
\usepackage{tabu}
\usepackage{amsmath}
\usepackage{setspace}
\usepackage{amsfonts}
\usepackage{appendix}
\usepackage{listings}
\usepackage{xcolor}
\usepackage{geometry}
\setcounter{secnumdepth}{4}
%\titleformat*{\section}{\LARGE}
%\renewcommand\refname{参考文献}
%\titleformat{\chapter}{\centering\bfseries\huge}{}{0.7em}{}{}
\titleformat{\section}{\LARGE\bf}{\thesection}{1em}{}{}
\titleformat{\subsection}{\Large\bfseries}{\thesubsection}{1em}{}{}
\titleformat{\subsubsection}{\large\bfseries}{\thesubsubsection}{1em}{}{}
\renewcommand{\contentsname}{{ \centerline{目{  } 录}}}
\setCJKfamilyfont{cjkhwxk}{STXINGKA.TTF}
%\setCJKfamilyfont{cjkhwxk}{华文行楷}
%\setCJKfamilyfont{cjkfzcs}{方正粗宋简体}
%\newcommand*{\cjkfzcs}{\CJKfamily{cjkfzcs}}
\newcommand*{\cjkhwxk}{\CJKfamily{cjkhwxk}}
%\newfontfamily\wryh{Microsoft YaHei}
%\newfontfamily\hwzs{华文中宋}
%\newfontfamily\hwst{华文宋体}
%\newfontfamily\hwfs{华文仿宋}
%\newfontfamily\jljt{方正静蕾简体}
%\newfontfamily\hwxk{华文行楷}
\newcommand{\verylarge}{\fontsize{60pt}{\baselineskip}\selectfont}  
\newcommand{\chuhao}{\fontsize{44.9pt}{\baselineskip}\selectfont}  
\newcommand{\xiaochu}{\fontsize{38.5pt}{\baselineskip}\selectfont}  
\newcommand{\yihao}{\fontsize{27.8pt}{\baselineskip}\selectfont}  
\newcommand{\xiaoyi}{\fontsize{25.7pt}{\baselineskip}\selectfont}  
\newcommand{\erhao}{\fontsize{23.5pt}{\baselineskip}\selectfont}  
\newcommand{\xiaoerhao}{\fontsize{19.3pt}{\baselineskip}\selectfont} 
\newcommand{\sihao}{\fontsize{14pt}{\baselineskip}\selectfont}      % 字号设置  
\newcommand{\xiaosihao}{\fontsize{12pt}{\baselineskip}\selectfont}  % 字号设置  
\newcommand{\wuhao}{\fontsize{10.5pt}{\baselineskip}\selectfont}    % 字号设置  
\newcommand{\xiaowuhao}{\fontsize{9pt}{\baselineskip}\selectfont}   % 字号设置  
\newcommand{\liuhao}{\fontsize{7.875pt}{\baselineskip}\selectfont}  % 字号设置  
\newcommand{\qihao}{\fontsize{5.25pt}{\baselineskip}\selectfont}    % 字号设置 

\usepackage{diagbox}
\usepackage{multirow}
\boldmath
\XeTeXlinebreaklocale "zh"
\XeTeXlinebreakskip = 0pt plus 1pt minus 0.1pt
\definecolor{cred}{rgb}{0.8,0.8,0.8}
\definecolor{cgreen}{rgb}{0,0.3,0}
\definecolor{cpurple}{rgb}{0.5,0,0.35}
\definecolor{cdocblue}{rgb}{0,0,0.3}
\definecolor{cdark}{rgb}{0.95,1.0,1.0}
\lstset{
	language=python,
	numbers=left,
	numberstyle=\tiny\color{black},
	showspaces=false,
	showstringspaces=false,
	basicstyle=\scriptsize,
	keywordstyle=\color{purple},
	commentstyle=\color{cgreen},
	stringstyle=\color{blue},
	frame=lines,
	% escapeinside=``,
	extendedchars=true, 
	xleftmargin=1em,
	xrightmargin=1em, 
	backgroundcolor=\color{cred},
	aboveskip=1em,
	breaklines=true,
	tabsize=4
} 

%\newfontfamily{\consolas}{Consolas}
%\newfontfamily{\monaco}{Monaco}
%\setmonofont[Mapping={}]{Consolas}	%英文引号之类的正常显示,相当于设置英文字体
%\setsansfont{Consolas} %设置英文字体 Monaco, Consolas,  Fantasque Sans Mono
%\setmainfont{Times New Roman}
%\setCJKmainfont{STZHONGS.TTF}
%\setmonofont{Consolas}
% \newfontfamily{\consolas}{YaHeiConsolas.ttf}
\newfontfamily{\monaco}{MONACO.TTF}
\setCJKmainfont{STZHONGS.TTF}
%\setmainfont{MONACO.TTF}
%\setsansfont{MONACO.TTF}

\newcommand{\fic}[1]{\begin{figure}[H]
		\center
		\includegraphics[width=0.8\textwidth]{#1}
	\end{figure}}
	
\newcommand{\sizedfic}[2]{\begin{figure}[H]
		\center
		\includegraphics[width=#1\textwidth]{#2}
	\end{figure}}

\newcommand{\codefile}[1]{\lstinputlisting{#1}}

\newcommand{\interval}{\vspace{0.5em}}

\newcommand{\tablestart}{
	\interval
	\begin{longtable}{p{2cm}p{10cm}}
	\hline}
\newcommand{\tableend}{
	\hline
	\end{longtable}
	\interval}

% 改变段间隔
\setlength{\parskip}{0.2em}
\linespread{1.1}

\usepackage{lastpage}
\usepackage{fancyhdr}
\pagestyle{fancy}
\lhead{\space \qquad \space}
\chead{openstack中REST\_API的使用 \qquad}
\rhead{\qquad\thepage/\pageref{LastPage}}

\begin{document}

\tableofcontents

\clearpage

\section{openstack中REST\_API的使用}
	如果我们想要使用REST API,首先需要认证,随后服务器将返回一个认证令牌。\par

	有了认证令牌后,我们就可以使用API请求了。发送API请求时,令牌信息包含在“X-Auth-Token”的包头中,使用该令牌发送请求,直到请求的服务完成或者Unauthorized (401)错误出现。

\subsection{认证}
\subsubsection{设置环境变量}
	我这里是使用devstack安装的openstack,我们可以直接在devstack目录下运行如下命令:
	\begin{lstlisting}
	source openrc
	\end{lstlisting}

	运行这个命令后,我们相当于设置了以下几个环境变量:
	\begin{lstlisting}
	$OS_AUTH_URL	$OS_IDENTITY_API_VERSION	$OS_PROJECT_NAME$OS_TOKEN	
	$OS_VOLUME_API_VERSION	$OS_CACERT	
	$OS_PASSWORD	$OS_REGION_NAME	$OS_USER_DOMAIN_ID        
	$OS_PROJECT_DOMAIN_ID	$OS_TENANT_NAME	$OS_USERNAME
	\end{lstlisting}

\subsubsection{申请认证令牌}
	运行curl命令去请求一个token:
	\begin{lstlisting}
	curl -s -X POST $OS_AUTH_URL/tokens \
	-H "Content-Type: application/json" \
	-d '{"auth": {"tenantName": "'"$OS_PROJECT_NAME"'", "passwordCredentials": {"username": "'"$OS_USERNAME"'", "password": "'"$OS_PASSWORD"'"}}}' \
	| python -m json.tool
	\end{lstlisting}

	如果发生错误,需要修改OS\_AUTH\_URL环境变量:
	\begin{lstlisting}
	# 首先查看OS_AUTH_URL值
	echo $OS_AUTH_URL
	# 设置OS_AUTH_URL的值,如果它原先的值为http://10.250.1.3:5000/v3
	export OS_AUTH_URL=http://10.250.1.3:5000/v2.0
	\end{lstlisting}

	再次运行上述的curl命令来申请一个token,结果如下:
	\begin{lstlisting}
pengsida@pogba:~/devstack$ curl -s -X POST http://10.250.1.3:5000/v2.0/tokens -H "Content-Type: application/json" -d '{"auth": {"tenantName": "'"$OS_PROJECT_NAME"'", "passwordCredentials": {"username": "'"$OS_USERNAME"'", "password": "'"$OS_PASSWORD"'"}}}' | python -m json.tool

{
    "access": {
        "metadata": {
            "is_admin": 0,
            "roles": [
                "4d20a1d8a1d64f2b8d521ccbccc9db1a",
                "306b85cfe88e4df8b2477eda61a91b68",
                "8abbd485a94d46c494ea63f37a2ad09e"
            ]
        },
        "serviceCatalog": [
            {
                "endpoints": [
                    {
                        "id": "23496304dc7842639f92fc75faf31782",
                        "publicURL": "http://10.250.1.3:8774/v2.1",
                        "region": "RegionOne"
                    }
                ],
                "endpoints_links": [],
                "name": "nova",
                "type": "compute"
            },
            {
                "endpoints": [
                    {
                        "id": "750d704ed5474307b63a077b1cbeedd9",
                        "publicURL": "http://10.250.1.3:9696/",
                        "region": "RegionOne"
                    }
                ],
                "endpoints_links": [],
                "name": "neutron",
                "type": "network"
            },
            {
                "endpoints": [
                    {
                        "id": "e9a2852060eb4a329980d75c5baa55f2",
                        "publicURL": "http://10.250.1.3:9292",
                        "region": "RegionOne"
                    }
                ],
                "endpoints_links": [],
                "name": "glance",
                "type": "image"
            },
            {
                "endpoints": [
                    {
                        "id": "558f71f4a83a4729873f3ab350f642b2",
                        "publicURL": "http://10.250.1.3:8774/v2/6c0fd7ee21d0497e8be5b4b2aacbab0e",
                        "region": "RegionOne"
                    }
                ],
                "endpoints_links": [],
                "name": "nova_legacy",
                "type": "compute_legacy"
            },
            {
                "endpoints": [
                    {
                        "adminURL": "http://10.250.1.3:6385",
                        "id": "4d4df5e62dc44b7a9baf6b33b5cb3e22",
                        "internalURL": "http://10.250.1.3:6385",
                        "publicURL": "http://10.250.1.3:6385",
                        "region": "RegionOne"
                    }
                ],
                "endpoints_links": [],
                "name": "ironic",
                "type": "baremetal"
            },
            {
                "endpoints": [
                    {
                        "adminURL": "http://10.250.1.3:8080",
                        "id": "2315c80d82f048488bc6b3d4b412b4ff",
                        "publicURL": "http://10.250.1.3:8080/v1/AUTH_6c0fd7ee21d0497e8be5b4b2aacbab0e",
                        "region": "RegionOne"
                    }
                ],
                "endpoints_links": [],
                "name": "swift",
                "type": "object-store"
            },
            {
                "endpoints": [
                    {
                        "id": "a1cef78d5a864105a2741daa7cf01c53",
                        "publicURL": "http://10.250.1.3/placement",
                        "region": "RegionOne"
                    }
                ],
                "endpoints_links": [],
                "name": "placement",
                "type": "placement"
            },
            {
                "endpoints": [
                    {
                        "adminURL": "http://10.250.1.3/identity_admin",
                        "id": "0ddc9f95a90d4b6a8b01554c9e1edac0",
                        "publicURL": "http://10.250.1.3/identity",
                        "region": "RegionOne"
                    }
                ],
                "endpoints_links": [],
                "name": "keystone",
                "type": "identity"
            }
        ],
        "token": {
            "audit_ids": [
                "T4sMOKXkRWSiSDu_i6l2pA"
            ],
            "expires": "2017-03-24T07:55:24.000000Z",
            "id": "gAAAAABY1MLceZlkbRVBXuCYU4pwXO9qUeJ6P1MXVpZpIpVPe0783hHgwXlFU_haDBkcFHln0qa4RMhvjxrQz-BOmPsW7WrB0qGd9WOEcDwEtCHTU3SzDKh6e_sThIwFqlDlXRgZaKZAJMaOEWusFnD4q85IbhhjGc_zRAYjkmuP53Sllkh3sUI",
            "issued_at": "2017-03-24T06:55:24.000000Z",
            "tenant": {
                "description": "",
                "enabled": true,
                "id": "6c0fd7ee21d0497e8be5b4b2aacbab0e",
                "name": "demo"
            }
        },
        "user": {
            "id": "9c3316565d4c41a8b272273763537e69",
            "name": "demo",
            "roles": [
                {
                    "name": "Member"
                },
                {
                    "name": "baremetal_observer"
                },
                {
                    "name": "anotherrole"
                }
            ],
            "roles_links": [],
            "username": "demo"
        }
    }
}
	\end{lstlisting}

\subsection{使用API请求}
\subsubsection{设置各个服务的环境变量}
	我们从返回的结果中可以看到,“token”这个字典中的“id”词条的值为“gAAAAABY1MLceZlkbRVBXuCYU4pwXO9qUeJ6P1MXVpZpIpVPe0783hHgwXlFU_haDBkcFHln0qa4RMhvjxrQz-BOmPsW7WrB0qGd9WOEcDwEtCHTU3SzDKh6e_sThIwFqlDlXRgZaKZAJMaOEWusFnD4q85IbhhjGc_zRAYjkmuP53Sllkh3sUI”。\par

	我们这里再设置一个OS\_TOKEN环境变量:
	\begin{lstlisting}
	export OS_TOKEN=gAAAAABY1MLceZlkbRVBXuCYU4pwXO9qUeJ6P1MXVpZpIpVPe0783hHgwXlFU_haDBkcFHln0qa4RMhvjxrQz-BOmPsW7WrB0qGd9WOEcDwEtCHTU3SzDKh6e_sThIwFqlDlXRgZaKZAJMaOEWusFnD4q85IbhhjGc_zRAYjkmuP53Sllkh3sUI
	\end{lstlisting}

	设置compute服务的环境变量:
	\begin{lstlisting}
	export OS_COMPUTE_API=http://10.250.1.3:8774/v2.1
	\end{lstlisting}

	设置image服务的环境变量:
	\begin{lstlisting}
	export OS_IMAGE_API=http://10.250.1.3:9292
	\end{lstlisting}

	设置baremetal服务的环境变量:
	\begin{lstlisting}
	export OS_BAREMETAL_API=http://10.250.1.3:6385
	\end{lstlisting}

\subsubsection{使用COMPUTE服务的API请求}
	COMPUTE API的网站如下:\par
	\url{https://developer.openstack.org/api-ref/compute/}\par

	举一个使用REST API的例子,下图是与servers有关的API:
	\fic{1.png}

	如果我们想要列出所有server,就使用如下命令:
	\begin{lstlisting}
	curl -s -X GET -H "X-Auth-Token: $OS_TOKEN" $OS_COMPUTE_API/servers | python -m json.tool
	\end{lstlisting}

	如果我们想要创建一个server,就是用如下命令:
	\begin{lstlisting}
	curl -s -X POST  -H "X-Auth-Token: $OS_TOKEN" -H "Content-Type: application/json" $OS_COMPUTE_API/servers -d '{"server":{"name":"CURL","imageRef":"b3365982-4580-4412-96c0-45684e676960","flavorRef":"9f211e26-a3f8-4e6a-b3c3-d93b5f062bd7","adminPass":"p1111111"}}' | python -m json.tool
	\end{lstlisting}

	上述命令中,imageRef和flavorRef分别是image和flavor资源的id,可以使用如下命令查看:
	\begin{lstlisting}
	curl -s -X GET -H "X-Auth-Token: $OS_TOKEN" $OS_COMPUTE_API/images | python -m json.tool
	curl -s -X GET -H "X-Auth-Token: $OS_TOKEN" $OS_COMPUTE_API/flavors | python -m json.tool
	\end{lstlisting}

\subsubsection{使用IRONIC服务的API请求}
	IRONIC API的网站如下:\par
	\url{https://developer.openstack.org/api-ref/baremetal/}

	类似于使用COMPUTE服务的API请求,IRONIC中与nodes有关的API如下图:
	\fic{2.png}

	我们使用如下命令查看所有的物理机:
	\begin{lstlisting}
	curl -s -X GET -H "X-Auth-Token: $OS_TOKEN" $OS_BAREMETAL_API/nodes | python -m json.tool
	\end{lstlisting}

\end{document}