% !TeX spellcheck = en_US
%% 字体:方正静蕾简体
%%		 方正粗宋
\documentclass[a4paper,left=1.5cm,right=1.5cm,11pt]{article}

\usepackage[utf8]{inputenc}
\usepackage{fontspec}
\usepackage{cite}
\usepackage{xeCJK}
\usepackage{indentfirst}
\usepackage{titlesec}
\usepackage{etoolbox}%
\makeatletter
\patchcmd{\ttlh@hang}{\parindent\z@}{\parindent\z@\leavevmode}{}{}%
\patchcmd{\ttlh@hang}{\noindent}{}{}{}%
\makeatother

\usepackage{longtable}
\usepackage{empheq}
\usepackage{graphicx}
\usepackage{float}
\usepackage{rotating}
\usepackage{subfigure}
\usepackage{tabu}
\usepackage{amsmath}
\usepackage{setspace}
\usepackage{amsfonts}
\usepackage{appendix}
\usepackage{listings}
\usepackage{xcolor}
\usepackage{geometry}
\setcounter{secnumdepth}{4}
%\titleformat*{\section}{\LARGE}
%\renewcommand\refname{参考文献}
%\titleformat{\chapter}{\centering\bfseries\huge}{}{0.7em}{}{}
\titleformat{\section}{\LARGE\bf}{\thesection}{1em}{}{}
\titleformat{\subsection}{\Large\bfseries}{\thesubsection}{1em}{}{}
\titleformat{\subsubsection}{\large\bfseries}{\thesubsubsection}{1em}{}{}
\renewcommand{\contentsname}{{ \centerline{目{  } 录}}}
\setCJKfamilyfont{cjkhwxk}{STXINGKA.TTF}
%\setCJKfamilyfont{cjkhwxk}{华文行楷}
%\setCJKfamilyfont{cjkfzcs}{方正粗宋简体}
%\newcommand*{\cjkfzcs}{\CJKfamily{cjkfzcs}}
\newcommand*{\cjkhwxk}{\CJKfamily{cjkhwxk}}
%\newfontfamily\wryh{Microsoft YaHei}
%\newfontfamily\hwzs{华文中宋}
%\newfontfamily\hwst{华文宋体}
%\newfontfamily\hwfs{华文仿宋}
%\newfontfamily\jljt{方正静蕾简体}
%\newfontfamily\hwxk{华文行楷}
\newcommand{\verylarge}{\fontsize{60pt}{\baselineskip}\selectfont}  
\newcommand{\chuhao}{\fontsize{44.9pt}{\baselineskip}\selectfont}  
\newcommand{\xiaochu}{\fontsize{38.5pt}{\baselineskip}\selectfont}  
\newcommand{\yihao}{\fontsize{27.8pt}{\baselineskip}\selectfont}  
\newcommand{\xiaoyi}{\fontsize{25.7pt}{\baselineskip}\selectfont}  
\newcommand{\erhao}{\fontsize{23.5pt}{\baselineskip}\selectfont}  
\newcommand{\xiaoerhao}{\fontsize{19.3pt}{\baselineskip}\selectfont} 
\newcommand{\sihao}{\fontsize{14pt}{\baselineskip}\selectfont}      % 字号设置  
\newcommand{\xiaosihao}{\fontsize{12pt}{\baselineskip}\selectfont}  % 字号设置  
\newcommand{\wuhao}{\fontsize{10.5pt}{\baselineskip}\selectfont}    % 字号设置  
\newcommand{\xiaowuhao}{\fontsize{9pt}{\baselineskip}\selectfont}   % 字号设置  
\newcommand{\liuhao}{\fontsize{7.875pt}{\baselineskip}\selectfont}  % 字号设置  
\newcommand{\qihao}{\fontsize{5.25pt}{\baselineskip}\selectfont}    % 字号设置 

\usepackage{diagbox}
\usepackage{multirow}
\boldmath
\XeTeXlinebreaklocale "zh"
\XeTeXlinebreakskip = 0pt plus 1pt minus 0.1pt
\definecolor{cred}{rgb}{0.8,0.8,0.8}
\definecolor{cgreen}{rgb}{0,0.3,0}
\definecolor{cpurple}{rgb}{0.5,0,0.35}
\definecolor{cdocblue}{rgb}{0,0,0.3}
\definecolor{cdark}{rgb}{0.95,1.0,1.0}
\lstset{
	language=bash,
	numbers=left,
	numberstyle=\tiny\color{black},
	showspaces=false,
	showstringspaces=false,
	basicstyle=\scriptsize,
	keywordstyle=\color{purple},
	commentstyle=\itshape\color{cgreen},
	stringstyle=\color{blue},
	frame=lines,
	% escapeinside=``,
	extendedchars=true, 
	xleftmargin=1em,
	xrightmargin=1em, 
	backgroundcolor=\color{cred},
	aboveskip=1em,
	breaklines=true,
	tabsize=4
} 

%\newfontfamily{\consolas}{Consolas}
%\newfontfamily{\monaco}{Monaco}
%\setmonofont[Mapping={}]{Consolas}	%英文引号之类的正常显示,相当于设置英文字体
%\setsansfont{Consolas} %设置英文字体 Monaco, Consolas,  Fantasque Sans Mono
%\setmainfont{Times New Roman}
%\setCJKmainfont{STZHONGS.TTF}
%\setmonofont{Consolas}
% \newfontfamily{\consolas}{YaHeiConsolas.ttf}
\newfontfamily{\monaco}{MONACO.TTF}
\setCJKmainfont{STZHONGS.TTF}
%\setmainfont{MONACO.TTF}
%\setsansfont{MONACO.TTF}

\newcommand{\fic}[1]{\begin{figure}[H]
		\center
		\includegraphics[width=0.8\textwidth]{#1}
	\end{figure}}
	
\newcommand{\sizedfic}[2]{\begin{figure}[H]
		\center
		\includegraphics[width=#1\textwidth]{#2}
	\end{figure}}

\newcommand{\codefile}[1]{\lstinputlisting{#1}}

\newcommand{\interval}{\vspace{0.5em}}

\newcommand{\tablestart}{
	\interval
	\begin{longtable}{p{2cm}p{10cm}}
	\hline}
\newcommand{\tableend}{
	\hline
	\end{longtable}
	\interval}

% 改变段间隔
\setlength{\parskip}{0.2em}
\linespread{1.1}

\usepackage{lastpage}
\usepackage{fancyhdr}
\pagestyle{fancy}
\lhead{\space \qquad \space}
\chead{openstack中的RPC \qquad}
\rhead{\qquad\thepage/\pageref{LastPage}}

\begin{document}

\tableofcontents

\clearpage

\section{openstack中的RPC}
	openstack中的RPC机制的实现基于AMQP作为通讯模型。

\subsection{AMQP}
	AMQP是用于异步消息通讯的消息中间件协议,它的模型如下所示:
	\sizedfic{0.7}{1.png}
	
	它有四个重要的角色:
	\begin{itemize}
		\item[1.] Exchange,根据Routing key转发消息到对应的Message Queue中。
		\item[2.] Routing key,Exchange根据Routing key来决定把消息发送到哪些Message Queue中。
		\item[3.] Publisher,消息发送者,指明消息的Routing key,然后把消息传给Exchange。
		\item[4.] Consumer,消息接受者,从Message Queue获取消息。
	\end{itemize}

	Exchange转发消息的原理如下:
	\begin{lstlisting}
	每一个发送的消息上都有一个routing key,而每一个Queue也有一个binding key。
	Exchange进行消息路由时,会查询每一个Queue,如果某个Queue的binding key与某个消息的routing key相匹配,这个消息就会被转发到那个Queue里。
	\end{lstlisting}

	AMQP有三种类型的Exchange:
	\begin{itemize}
		\item[1.] Direct,binding key和routing key必须完全一致,不支持通配符。
		\item[2.] Topic,同Direct类型,但支持通配符。
		\item[3.] Fanout,忽略binding key和routing key,消息会被传递到所有绑定的队列上。
	\end{itemize}

\clearpage

\subsection{RPC中的client和server}
	在RPC通信中有两个角色:client和server。client发起RPC请求,server端接受RPC请求,然后调用本地的程序执行,
	最后将执行结果返回给client。\par

	openstack中的RPC利用了publisher和consumer实现了client和server之间的RPC通信,
	其中publisher相当于client,consumer相当于server。\par

\subsubsection{四种publisher}
	openstack中四种publisher的介绍如下:
	\begin{itemize}
		\item[1.] Direct Publisher:用于点对点的消息通信,在 OpenStack RPC 通信中用于建立 RPC 消息应答通路。创建或声明 Direct Exchange,用于 RPC 的消息返回,Exchange 的名字以消息 id 命名。
		\item[2.] Topic Publisher:用于 Publish-Subscribe(Pub-Sub)模式的通信,Topic Publisher 创建 Topic Exchange,用于 RPC 消息发送,并设置消息的 Routing Key 转发消息。
		\item[3.] Fanout Publisher:创建 Fanout Exchange,用于广播消息的发送,所有绑定到该 Exchange 的 Message Queue 都会接收到消息。
		\item[4.] Notify Publisher:同 Topic Publisher,在 Openstack 系统中用于发送 Notification 相关的消息。
	\end{itemize}

\subsubsection{三种consumer}
	openstack中三种consumer的介绍如下:
	\begin{itemize}
		\item[1.] Direct Consumer:接收 Direct Exchange 发送的消息,创建专属的 Message Queue,即 Message Queue 的消息只能由创建 Message Queue 的 Consumer 获取。
		\item[2.] Topic Consumer:接收 topic Exchange 根据 Routing Key 转发的消息,创建的 Message Queue 是可共享的,这就意味着多个 Consumer 可以订阅同一个 Message Queue,并从其中读取消息。
		\item[3.] Fanout Consumer:接收 Fanout Exchange 广播的消息,并创建专属的 Message Queue。
	\end{itemize}

\subsubsection{RPC发送请求}
	RPC中的client端由publisher实现,publisher发送消息并声明消息地址。consumer接收消息并进行消息处理,
	如果需要消息应答则返回处理请求的结果消息。\par

	OpenStack提供了三种发送消息请求的方式:
	\begin{itemize}
		\item[1.] rpc.call 发送 RPC 请求并返回请求处理结果,
		由 Topic Publisher 发送消息,Topic Exchange 根据消息地址进行消息转发至对应的 Message Queue 中,
		Topic Consumer 监听 Message Queue,发现需要处理的消息则进行消息处理,并由 Direct Publisher 将请求处理结果消息,
		请求发送方创建 Direct Consumer 监听消息的返回结果。\par
		如下图所示:
		\sizedfic{0.7}{2.jpg}

		\item[2.] rpc.cast 发送 RPC 请求无返回,与 rpc.call 不同之处在于,不需要请求处理结果的返回,因此没有 Direct Publisher 和 Direct Consumer 处理。\par
		如下图所示:
		\sizedfic{0.7}{3.jpg}

		\item[3.] rpc.fanout\_cast 用于发送 RPC 广播信息无返回结果,如下图所示:
		\sizedfic{0.6}{4.jpg}

	\end{itemize}

\subsection{openstack中使用RPC的例子}
	我们分析 cinder 创建 volume 的 RPC 消息的发送过程,从而更加深刻的理解 RPC 消息的处理流程。\par

	Create volume 的过程可以看做是两个阶段的 RPC 请求处理。\par

	第一阶段:
	\begin{itemize}
		\item[1.] Client 即 cinder-api 调用 RPC.cast 发送消息,RPC.cast 创建 Topic Publisher 对象,topic 为 cinder-scheduler,也就是消息的 Routing Key,Exchange name 为 OpenStack,消息体中标记了消息接收者即 Server 所需要调用的方法“create\_volume”。
		\item[2.] Exchange 接收到消息,根据 Routing Key 将消息发送至 Message Queue cinder-scheduler 上,这个 Message Queue 是由 cinder-scheduler 服务的 Topic Consumer 订阅的,因此,cinder-scheduler 服务将接收到 create\_volume 的消息。
		\item[3.] Consumer 调用注册的 RPC\_dispatcher,即 cinder-scheduler manager,然后由 cinder-scheduler manager 调用 RPC 处理方法 create\_volume。
	\end{itemize}

	第一阶段的流程如下图所示:
	\sizedfic{0.7}{5.jpg}

	第一个阶段的最后是由 cinder-scheduler manager 调用 RPC 处理方法 create\_volume,我们第二阶段就是要分析这个过程:
	\begin{itemize}
		\item[1.] Cinder-scheduler 调用 RPC.cast,Exchange name 为 Openstack, routing-key 为 cinder-volume:host。
		\item[2.] Exchange 接收到消息,根据 Routing Key 将消息发送到队列 cinder-volume:host 上。
		\item[3.] 在主机 host 上的 Cinder-volume consumer 从 Queue 中接收到 message,回调 RPC\_dispatcher 即 cinder-volume manager。
		\item[4.] Cinder-volume manger 根据消息体中 method 信息,调用 create\_volume 方法,创建 volume。
	\end{itemize}

	第二阶段的流程如下图所示:
	\fic{6.jpg}

\end{document}