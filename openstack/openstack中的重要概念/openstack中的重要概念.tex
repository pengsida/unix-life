% !TeX spellcheck = en_US
%% 字体:方正静蕾简体
%%		 方正粗宋
\documentclass[a4paper,left=1.5cm,right=1.5cm,11pt]{article}

\usepackage[utf8]{inputenc}
\usepackage{fontspec}
\usepackage{cite}
\usepackage{xeCJK}
\usepackage{indentfirst}
\usepackage{titlesec}
\usepackage{etoolbox}%
\makeatletter
\patchcmd{\ttlh@hang}{\parindent\z@}{\parindent\z@\leavevmode}{}{}%
\patchcmd{\ttlh@hang}{\noindent}{}{}{}%
\makeatother

\usepackage{longtable}
\usepackage{empheq}
\usepackage{graphicx}
\usepackage{float}
\usepackage{rotating}
\usepackage{subfigure}
\usepackage{tabu}
\usepackage{amsmath}
\usepackage{setspace}
\usepackage{amsfonts}
\usepackage{appendix}
\usepackage{listings}
\usepackage{xcolor}
\usepackage{geometry}
\setcounter{secnumdepth}{4}
%\titleformat*{\section}{\LARGE}
%\renewcommand\refname{参考文献}
%\titleformat{\chapter}{\centering\bfseries\huge}{}{0.7em}{}{}
\titleformat{\section}{\LARGE\bf}{\thesection}{1em}{}{}
\titleformat{\subsection}{\Large\bfseries}{\thesubsection}{1em}{}{}
\titleformat{\subsubsection}{\large\bfseries}{\thesubsubsection}{1em}{}{}
\renewcommand{\contentsname}{{ \centerline{目{  } 录}}}
\setCJKfamilyfont{cjkhwxk}{STXINGKA.TTF}
%\setCJKfamilyfont{cjkhwxk}{华文行楷}
%\setCJKfamilyfont{cjkfzcs}{方正粗宋简体}
%\newcommand*{\cjkfzcs}{\CJKfamily{cjkfzcs}}
\newcommand*{\cjkhwxk}{\CJKfamily{cjkhwxk}}
%\newfontfamily\wryh{Microsoft YaHei}
%\newfontfamily\hwzs{华文中宋}
%\newfontfamily\hwst{华文宋体}
%\newfontfamily\hwfs{华文仿宋}
%\newfontfamily\jljt{方正静蕾简体}
%\newfontfamily\hwxk{华文行楷}
\newcommand{\verylarge}{\fontsize{60pt}{\baselineskip}\selectfont}  
\newcommand{\chuhao}{\fontsize{44.9pt}{\baselineskip}\selectfont}  
\newcommand{\xiaochu}{\fontsize{38.5pt}{\baselineskip}\selectfont}  
\newcommand{\yihao}{\fontsize{27.8pt}{\baselineskip}\selectfont}  
\newcommand{\xiaoyi}{\fontsize{25.7pt}{\baselineskip}\selectfont}  
\newcommand{\erhao}{\fontsize{23.5pt}{\baselineskip}\selectfont}  
\newcommand{\xiaoerhao}{\fontsize{19.3pt}{\baselineskip}\selectfont} 
\newcommand{\sihao}{\fontsize{14pt}{\baselineskip}\selectfont}      % 字号设置  
\newcommand{\xiaosihao}{\fontsize{12pt}{\baselineskip}\selectfont}  % 字号设置  
\newcommand{\wuhao}{\fontsize{10.5pt}{\baselineskip}\selectfont}    % 字号设置  
\newcommand{\xiaowuhao}{\fontsize{9pt}{\baselineskip}\selectfont}   % 字号设置  
\newcommand{\liuhao}{\fontsize{7.875pt}{\baselineskip}\selectfont}  % 字号设置  
\newcommand{\qihao}{\fontsize{5.25pt}{\baselineskip}\selectfont}    % 字号设置 

\usepackage{diagbox}
\usepackage{multirow}
\boldmath
\XeTeXlinebreaklocale "zh"
\XeTeXlinebreakskip = 0pt plus 1pt minus 0.1pt
\definecolor{cred}{rgb}{0.8,0.8,0.8}
\definecolor{cgreen}{rgb}{0,0.3,0}
\definecolor{cpurple}{rgb}{0.5,0,0.35}
\definecolor{cdocblue}{rgb}{0,0,0.3}
\definecolor{cdark}{rgb}{0.95,1.0,1.0}
\lstset{
	language=bash,
	numbers=left,
	numberstyle=\tiny\color{black},
	showspaces=false,
	showstringspaces=false,
	basicstyle=\scriptsize,
	keywordstyle=\color{purple},
	commentstyle=\itshape\color{cgreen},
	stringstyle=\color{blue},
	frame=lines,
	% escapeinside=``,
	extendedchars=true, 
	xleftmargin=1em,
	xrightmargin=1em, 
	backgroundcolor=\color{cred},
	aboveskip=1em,
	breaklines=true,
	tabsize=4
} 

%\newfontfamily{\consolas}{Consolas}
%\newfontfamily{\monaco}{Monaco}
%\setmonofont[Mapping={}]{Consolas}	%英文引号之类的正常显示,相当于设置英文字体
%\setsansfont{Consolas} %设置英文字体 Monaco, Consolas,  Fantasque Sans Mono
%\setmainfont{Times New Roman}
%\setCJKmainfont{STZHONGS.TTF}
%\setmonofont{Consolas}
% \newfontfamily{\consolas}{YaHeiConsolas.ttf}
\newfontfamily{\monaco}{MONACO.TTF}
\setCJKmainfont{STZHONGS.TTF}
%\setmainfont{MONACO.TTF}
%\setsansfont{MONACO.TTF}

\newcommand{\fic}[1]{\begin{figure}[H]
		\center
		\includegraphics[width=0.8\textwidth]{#1}
	\end{figure}}
	
\newcommand{\sizedfic}[2]{\begin{figure}[H]
		\center
		\includegraphics[width=#1\textwidth]{#2}
	\end{figure}}

\newcommand{\codefile}[1]{\lstinputlisting{#1}}

\newcommand{\interval}{\vspace{0.5em}}

\newcommand{\tablestart}{
	\interval
	\begin{longtable}{p{2cm}p{10cm}}
	\hline}
\newcommand{\tableend}{
	\hline
	\end{longtable}
	\interval}

% 改变段间隔
\setlength{\parskip}{0.2em}
\linespread{1.1}

\usepackage{lastpage}
\usepackage{fancyhdr}
\pagestyle{fancy}
\lhead{\space \qquad \space}
\chead{openstack中的重要概念 \qquad}
\rhead{\qquad\thepage/\pageref{LastPage}}

\begin{document}
    
\section{openstack中的重要概念}
\subsection{flavor}
    在OpenStack中,flavors定义了nova实例的CPU、内存、存储容量等数值。
    简单地说,一个flavor相当于一个实例的硬件配置。\par

    一个flavor包含了如下参数:
    \begin{itemize}
        \item[1.] Flavor ID,这是一个flavor的uuid。uuid一般是自动产生的。
        \item[2.] Name,一个flavor的名称。
        \item[3.] VCPUS,虚拟CPU的数量。
        \item[4.] Memory MB,RAM的大小,单位为MB。
        \item[5.] Root Disk GB,root分区的磁盘空间的大小,单位为GB。
        \item[6.] Ephemeral Disk GB,临时分区的磁盘空间的大小,单位为GB,
                  默认值为0。需要知道的是,当虚拟机关闭时,临时分区所有数据将丢失。
                  而且制作快照时,不会考虑临时分区中的数据。
        \item[7.] Swap,交换空间的大小,单位为MB,默认值为0。
        \item[8.] RXTX Factor,这是一个可选属性,用于创建不同带宽的server,默认值为1.0。
                  RXTX Factor仅适用于基于Xen或NSX的系统。
        \item[9.] Is Public,用于决定是否任何用户都可以使用这个flavor,默认值为True。
        \item[10.] Extra Specs,键和值的pair,用于定义flavor可以在哪些compute node上运行。
    \end{itemize}

    对于Newton而言,openstack没有默认的flavor,而Mitaka和更早的版本有如下的默认flavor:
    \fic{1.png}

\subsubsection{管理flavor}
    在openstack中,可以使用openstack flavor命令行工具来管理flavor。\par

    常用的openstack flavor命令如下所示:
    \begin{itemize}
        \item[1.] 列出flavors,并显示出flavor的属性,命令如下:
        \begin{lstlisting}
    openstack flavor list
        \end{lstlisting}

        \item[2.] 创建flavor,命令如下:
        \begin{lstlisting}
    openstack flavor create FLAVOR_NAME --id FLAVOR_ID --ram RAM_IN_MB --disk ROOT_DISK_IN_GB --vcpus NUMBER_OF_VCPUS
        \end{lstlisting}

        可以通过如下命令查看create更多的选项:
        \begin{lstlisting}
    openstack help flavor create
        \end{lstlisting}

        \item[3.] 将flavor分配给一个project,命令如下:
        \begin{lstlisting}
    # FLAVOR是flavor的名称或ID
    # TENANT_ID是project的ID
    nova flavor-access-add FLAVOR TENANT_ID
        \end{lstlisting}

        \item[4.] 删除flavor,命令如下:
        \begin{lstlisting}
    openstack flavor delete FLAVOR_ID
        \end{lstlisting}

        \item[5.] 查看flavor命令的帮助手册,命令如下:
        \begin{lstlisting}
    openstack flavor --help
        \end{lstlisting}
    \end{itemize}

\subsubsection{Extra Specs}
    这个网页可以查看Extra Specs上的值:
    \begin{lstlisting}
    https://docs.openstack.org/admin-guide/compute-flavors.html#extra-specs
    \end{lstlisting}

    % flavor中的Extra Specs包含如下几个方面:
    % \begin{itemize}
    %     \item[1.] CPU limits,有如下几个选项:
    %     \begin{itemize}
    %         \item cpu\_shares,指定域的比例加权共享(propotional weighted share),这个值没有单位。
    %         \item cpu\_shares\_level,指定分配等级(allocation level),有custon,high,normal和low。
    %         \item cpu\_period,指定qemu和lxc虚拟机管理程序的enforcement interval。
    %               在一段时间内,一个域的vCPU不允许消耗超过运行时间的配额值。
    %               这个值应该在[1000, 1000000]的范围内。
    %               值为0的cpu\_period表示没有值。
    %     \end{itemize}
    % \end{itemize}

    % \begin{itemize}
    %     \item[1.] 指定CPU、Memory、Disk和Bandwidth的配额quota,命令如下:
    %     \begin{lstlisting}
    % openstack flavor set FLAVOR_NAME \
    %     --property quota:cpu_quota=10000 \
    %     --property quota:cpu_period=20000
    %     \end{lstlisting}
    % \end{itemize}

\end{document}