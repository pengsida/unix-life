% !TeX spellcheck = en_US
%% 字体:方正静蕾简体
%%		 方正粗宋
\documentclass[a4paper,left=2.5cm,right=2.5cm,11pt]{article}

\usepackage[utf8]{inputenc}
\usepackage{fontspec}
\usepackage{cite}
\usepackage{xeCJK}
\usepackage{indentfirst}
\usepackage{titlesec}
\usepackage{longtable}
\usepackage{graphicx}
\usepackage{float}
\usepackage{rotating}
\usepackage{subfigure}
\usepackage{tabu}
\usepackage{amsmath}
\usepackage{setspace}
\usepackage{amsfonts}
\usepackage{appendix}
\usepackage{listings}
\usepackage{xcolor}
\usepackage{geometry}
\setcounter{secnumdepth}{4}
\usepackage{mhchem}
\usepackage{multirow}
\usepackage{extarrows}
\usepackage{hyperref}
\titleformat*{\section}{\LARGE}
\renewcommand\refname{参考文献}
\renewcommand{\abstractname}{\sihao \cjkfzcs 摘{  }要}
%\titleformat{\chapter}{\centering\bfseries\huge\wryh}{}{0.7em}{}{}
%\titleformat{\section}{\LARGE\bf}{\thesection}{1em}{}{}
\titleformat{\subsection}{\Large\bfseries}{\thesubsection}{1em}{}{}
\titleformat{\subsubsection}{\large\bfseries}{\thesubsubsection}{1em}{}{}
\renewcommand{\contentsname}{{\cjkfzcs \centerline{目{  } 录}}}
\setCJKfamilyfont{cjkhwxk}{STXingkai}
\setCJKfamilyfont{cjkfzcs}{STSongti-SC-Regular}
% \setCJKfamilyfont{cjkhwxk}{华文行楷}
% \setCJKfamilyfont{cjkfzcs}{方正粗宋简体}
\newcommand*{\cjkfzcs}{\CJKfamily{cjkfzcs}}
\newcommand*{\cjkhwxk}{\CJKfamily{cjkhwxk}}
\newfontfamily\wryh{Microsoft YaHei}
\newfontfamily\hwzs{STZhongsong}
\newfontfamily\hwst{STSong}
\newfontfamily\hwfs{STFangsong}
\newfontfamily\jljt{MicrosoftYaHei}
\newfontfamily\hwxk{STXingkai}
% \newfontfamily\hwzs{华文中宋}
% \newfontfamily\hwst{华文宋体}
% \newfontfamily\hwfs{华文仿宋}
% \newfontfamily\jljt{方正静蕾简体}
% \newfontfamily\hwxk{华文行楷}
\newcommand{\verylarge}{\fontsize{60pt}{\baselineskip}\selectfont}  
\newcommand{\chuhao}{\fontsize{44.9pt}{\baselineskip}\selectfont}  
\newcommand{\xiaochu}{\fontsize{38.5pt}{\baselineskip}\selectfont}  
\newcommand{\yihao}{\fontsize{27.8pt}{\baselineskip}\selectfont}  
\newcommand{\xiaoyi}{\fontsize{25.7pt}{\baselineskip}\selectfont}  
\newcommand{\erhao}{\fontsize{23.5pt}{\baselineskip}\selectfont}  
\newcommand{\xiaoerhao}{\fontsize{19.3pt}{\baselineskip}\selectfont} 
\newcommand{\sihao}{\fontsize{14pt}{\baselineskip}\selectfont}      % 字号设置  
\newcommand{\xiaosihao}{\fontsize{12pt}{\baselineskip}\selectfont}  % 字号设置  
\newcommand{\wuhao}{\fontsize{10.5pt}{\baselineskip}\selectfont}    % 字号设置  
\newcommand{\xiaowuhao}{\fontsize{9pt}{\baselineskip}\selectfont}   % 字号设置  
\newcommand{\liuhao}{\fontsize{7.875pt}{\baselineskip}\selectfont}  % 字号设置  
\newcommand{\qihao}{\fontsize{5.25pt}{\baselineskip}\selectfont}    % 字号设置 

\usepackage{diagbox}
\usepackage{multirow}
\boldmath
\XeTeXlinebreaklocale "zh"
\XeTeXlinebreakskip = 0pt plus 1pt minus 0.1pt
\definecolor{cred}{rgb}{0.8,0.8,0.8}
\definecolor{cgreen}{rgb}{0,0.3,0}
\definecolor{cpurple}{rgb}{0.5,0,0.35}
\definecolor{cdocblue}{rgb}{0,0,0.3}
\definecolor{cdark}{rgb}{0.95,1.0,1.0}
\lstset{
	language=[x86masm]Assembler,
	numbers=left,
	numberstyle=\tiny\color{black},
	showspaces=false,
	showstringspaces=false,
	basicstyle=\scriptsize,
	keywordstyle=\color{purple},
	commentstyle=\itshape\color{cgreen},
	stringstyle=\color{blue},
	frame=lines,
	% escapeinside=``,
	extendedchars=true, 
	xleftmargin=1em,
	xrightmargin=1em, 
	backgroundcolor=\color{cred},
	aboveskip=1em,
	breaklines=true,
	tabsize=4
} 

\newfontfamily{\consolas}{Consolas}
\newfontfamily{\monaco}{Monaco}
\setmonofont[Mapping={}]{Consolas}	%英文引号之类的正常显示,相当于设置英文字体
\setsansfont{Consolas} %设置英文字体 Monaco, Consolas,  Fantasque Sans Mono
\setmainfont{Times New Roman}

\setCJKmainfont{华文中宋}


\newcommand{\fic}[1]{\begin{figure}[H]
		\center
		\includegraphics[width=0.8\textwidth]{#1}
	\end{figure}}
	
\newcommand{\sizedfic}[2]{\begin{figure}[H]
		\center
		\includegraphics[width=#1\textwidth]{#2}
	\end{figure}}

\newcommand{\codefile}[1]{\lstinputlisting{#1}}

\newcommand{\interval}{\vspace{0.5em}}

\newcommand{\tablestart}{
	\interval
	\begin{longtable}{p{2cm}p{10cm}}
	\hline}
\newcommand{\tableend}{
	\hline
	\end{longtable}
	\interval}

% 改变段间隔
\setlength{\parskip}{0.2em}
\linespread{1.1}

\usepackage{lastpage}
\usepackage{fancyhdr}
\pagestyle{fancy}
\lhead{\space \qquad \space}
\chead{寄存器 \qquad}
\rhead{\qquad\thepage/\pageref{LastPage}}
\begin{document}

\tableofcontents

\clearpage

\section{通用寄存器}
	8086CPU中有AX、BX、CX和DX四个寄存器,用于存放一般性的数据。\par

	通过mov指令可以修改AX、BX、CX和DX的值,例子如下:
	\begin{lstlisting}
	mov ax,123
	mov bx,123
	mov cx,123
	mov dx,123
	\end{lstlisting}

\section{8086CPU给出物理地址的方法}
	8086CPU有20位地址总线,可是8086CPU是16位结构,
	所以地址加法器采用物理地址=段地址x16+偏移地址的方法将段地址和偏移地址合成物理地址。\par

	需要注意的是,虽然“段地址”这个名称中包含段,但是内存没有分段,只是CPU用分段的方式来管理内存。

\section{段寄存器}
	8086CPU中有CS、DS、SS和ES四个寄存器,用于存放段地址。\par

	不同于通用寄存器,8086CPU不支持将数据直接送入段寄存器。可以将数据先送入通用寄存器,然后再将通用寄存器的内容送入段寄存器。
	也可以将内存单元中的数据送入段寄存器。

\subsection{CS和IP}
	CS是代码段寄存器,IP为指令指针寄存器。当CS中的内容为M,IP中的内容为N,8086CPU将从内存M$\times$16+N开始执行指令。\par

	8086CPU读取指令的步骤如下:
	\begin{itemize}
		\item[1.] 从CS:IP指向的内存单元读取指令,读取的指令进入指令缓冲器。
		\item[2.] IP=IP+所读取指令的长度,从而指向下一条指令。
		\item[3.] 执行指令,转到步骤1,重复这个过程。
	\end{itemize}

\subsection{修改CS和IP的值}
	mov指令不能用于修改CS、IP的值,只有转移指令可以修改CS、IP的内容,比如说jmp指令。\par

	如果想同时修改CS、IP的内容,可以用指令“jmp 段地址:偏移地址”完成,如下例所示:
	\begin{lstlisting}
	; 执行后,CS=2AE3H,IP=0003H,CPU从2AE33H处读取指令
	jmp 2AE3:3
	\end{lstlisting}

	如果只想修改IP的内容,可以用指令“jmp 某一合法寄存器”,如下例所示:
	\begin{lstlisting}
	; jmp指令用寄存器中的值修改IP,CS的内容不变
	jmp ax
	jmp bx
	\end{lstlisting}

\subsection{代码段}
	在编程时,可以根据需要将一组内存单元定义为一个段。代码段就是用于存放代码的一组内存单元。\par

	需要知道的是,虽然我们编程时安排了代码段,但CPU不会由于这种安排自动地执行代码段中的指令,CPU只认可CS:IP指向的内存单元中的内容为指令。
	所以要让CPU执行我们放在代码段中的指令,必须讲CS:IP指向所定义的代码段中的第一条指令的首地址。

\section{DS和[address]}
	DS寄存器用于访问数据的段地址,而[address]中的address是偏移地址,例子如下:
	\begin{lstlisting}
	; 将1000:0处的数据读入al中
	mov bx,1000H
	mov ds,bx
	mov al,[0]
	\end{lstlisting}

	mov指令支持将一个内存单元中的内容送入一个寄存器中,指令格式为:
	\begin{lstlisting}
	mov 寄存器名,内存单元地址
	\end{lstlisting}

	其中内存单元地址就是用ds和“[...]”表示,ds表示内存单元的段地址,[...]表示内存单元的偏移地址。

\section{mov、add、sub指令}
	mov指令有如下几种形式:
	\begin{lstlisting}
	mov 寄存器, 数据
	mov 寄存器, 寄存器
	mov 寄存器, 内存单元
	mov 内存单元, 寄存器
	mov 段寄存器, 寄存器
	mov 寄存器, 段寄存器
	mov 段寄存器, 内存单元
	mov 内存单元, 段寄存器
	\end{lstlisting}

	add指令有如下几种形式:
	\begin{lstlisting}
	add 寄存器, 数据
	add 寄存器, 寄存器
	add 寄存器, 内存单元
	add 内存单元, 寄存器
	\end{lstlisting}

	sub指令有如下几种形式:
	\begin{lstlisting}
	sub 寄存器, 数据
	sub 寄存器, 寄存器
	sub 寄存器, 内存单元
	sub 内存单元, 寄存器
	\end{lstlisting}

\section{数据段}
	数据段就是用于存储数据的一组内存单元。在访问数据段中的数据时,只要用ds存放数据段的段地址,然后根据需要,用相关指令访问数据段中的具体单元。

\section{CPU提供的栈机制}
	8086CPU提供入栈和出栈指令,也就是push和pop。\par

\subsection{SS和SP寄存器}

	8086CPU提供了段寄存器SS和寄存器SP,SS用于存放栈顶的段地址,SP用于存放偏移地址,SS:SP指向栈顶元素。\par

	8086CPU对push指令的执行如下图所示:
	\fic{1.png}

	可以看到,入栈时,栈顶从高地址向低地址增长,不过段基址是低地址,而SP初始值是栈的长度。\par

	需要注意的是,8086CPU不能保证我们对栈的操作不会越界,我们在编程的时候需要自己防止栈顶越界的问题。

\subsection{push、pop指令}
	push的形式如下:
	\begin{lstlisting}
	push 寄存器
	push 段寄存器
	push 内存单元
	\end{lstlisting}

	pop的形式如下:
	\begin{lstlisting}
	pop 寄存器
	pop 段寄存器
	pop 内存单元
	\end{lstlisting}

\subsection{栈段}
	我们可以将一组内存空间当作栈来使用,以栈的方式进行访问,那么这段空间就可以称为一个栈。\par

	可以用SS存放栈段的基地址,将SP的初始值设为栈段的长度,就可以将SS:SP指向我们定义的栈段。

\end{document}
