% !TeX spellcheck = en_US
%% 字体:方正静蕾简体
%%		 方正粗宋
\documentclass[a4paper,left=2.5cm,right=2.5cm,11pt]{article}

\usepackage[utf8]{inputenc}
\usepackage{fontspec}
\usepackage{cite}
\usepackage{xeCJK}
\usepackage{indentfirst}
\usepackage{titlesec}
\usepackage{longtable}
\usepackage{graphicx}
\usepackage{float}
\usepackage{rotating}
\usepackage{subfigure}
\usepackage{tabu}
\usepackage{amsmath}
\usepackage{setspace}
\usepackage{amsfonts}
\usepackage{appendix}
\usepackage{listings}
\usepackage{xcolor}
\usepackage{geometry}
\setcounter{secnumdepth}{4}
\usepackage{mhchem}
\usepackage{multirow}
\usepackage{extarrows}
\usepackage{hyperref}
\titleformat*{\section}{\LARGE}
\renewcommand\refname{参考文献}
\renewcommand{\abstractname}{\sihao \cjkfzcs 摘{  }要}
%\titleformat{\chapter}{\centering\bfseries\huge\wryh}{}{0.7em}{}{}
%\titleformat{\section}{\LARGE\bf}{\thesection}{1em}{}{}
\titleformat{\subsection}{\Large\bfseries}{\thesubsection}{1em}{}{}
\titleformat{\subsubsection}{\large\bfseries}{\thesubsubsection}{1em}{}{}
\renewcommand{\contentsname}{{\cjkfzcs \centerline{目{  } 录}}}
\setCJKfamilyfont{cjkhwxk}{STXingkai}
\setCJKfamilyfont{cjkfzcs}{STSongti-SC-Regular}
% \setCJKfamilyfont{cjkhwxk}{华文行楷}
% \setCJKfamilyfont{cjkfzcs}{方正粗宋简体}
\newcommand*{\cjkfzcs}{\CJKfamily{cjkfzcs}}
\newcommand*{\cjkhwxk}{\CJKfamily{cjkhwxk}}
\newfontfamily\wryh{Microsoft YaHei}
\newfontfamily\hwzs{STZhongsong}
\newfontfamily\hwst{STSong}
\newfontfamily\hwfs{STFangsong}
\newfontfamily\jljt{MicrosoftYaHei}
\newfontfamily\hwxk{STXingkai}
% \newfontfamily\hwzs{华文中宋}
% \newfontfamily\hwst{华文宋体}
% \newfontfamily\hwfs{华文仿宋}
% \newfontfamily\jljt{方正静蕾简体}
% \newfontfamily\hwxk{华文行楷}
\newcommand{\verylarge}{\fontsize{60pt}{\baselineskip}\selectfont}  
\newcommand{\chuhao}{\fontsize{44.9pt}{\baselineskip}\selectfont}  
\newcommand{\xiaochu}{\fontsize{38.5pt}{\baselineskip}\selectfont}  
\newcommand{\yihao}{\fontsize{27.8pt}{\baselineskip}\selectfont}  
\newcommand{\xiaoyi}{\fontsize{25.7pt}{\baselineskip}\selectfont}  
\newcommand{\erhao}{\fontsize{23.5pt}{\baselineskip}\selectfont}  
\newcommand{\xiaoerhao}{\fontsize{19.3pt}{\baselineskip}\selectfont} 
\newcommand{\sihao}{\fontsize{14pt}{\baselineskip}\selectfont}      % 字号设置  
\newcommand{\xiaosihao}{\fontsize{12pt}{\baselineskip}\selectfont}  % 字号设置  
\newcommand{\wuhao}{\fontsize{10.5pt}{\baselineskip}\selectfont}    % 字号设置  
\newcommand{\xiaowuhao}{\fontsize{9pt}{\baselineskip}\selectfont}   % 字号设置  
\newcommand{\liuhao}{\fontsize{7.875pt}{\baselineskip}\selectfont}  % 字号设置  
\newcommand{\qihao}{\fontsize{5.25pt}{\baselineskip}\selectfont}    % 字号设置 

\usepackage{diagbox}
\usepackage{multirow}
\boldmath
\XeTeXlinebreaklocale "zh"
\XeTeXlinebreakskip = 0pt plus 1pt minus 0.1pt
\definecolor{cred}{rgb}{0.8,0.8,0.8}
\definecolor{cgreen}{rgb}{0,0.3,0}
\definecolor{cpurple}{rgb}{0.5,0,0.35}
\definecolor{cdocblue}{rgb}{0,0,0.3}
\definecolor{cdark}{rgb}{0.95,1.0,1.0}
\lstset{
	language=[x86masm]Assembler,
	numbers=left,
	numberstyle=\tiny\color{black},
	showspaces=false,
	showstringspaces=false,
	basicstyle=\scriptsize,
	keywordstyle=\color{purple},
	commentstyle=\itshape\color{cgreen},
	stringstyle=\color{blue},
	frame=lines,
	% escapeinside=``,
	extendedchars=true, 
	xleftmargin=1em,
	xrightmargin=1em, 
	backgroundcolor=\color{cred},
	aboveskip=1em,
	breaklines=true,
	tabsize=4
} 

\newfontfamily{\consolas}{Consolas}
\newfontfamily{\monaco}{Monaco}
\setmonofont[Mapping={}]{Consolas}	%英文引号之类的正常显示,相当于设置英文字体
\setsansfont{Consolas} %设置英文字体 Monaco, Consolas,  Fantasque Sans Mono
\setmainfont{Times New Roman}

\setCJKmainfont{华文中宋}


\newcommand{\fic}[1]{\begin{figure}[H]
		\center
		\includegraphics[width=0.8\textwidth]{#1}
	\end{figure}}
	
\newcommand{\sizedfic}[2]{\begin{figure}[H]
		\center
		\includegraphics[width=#1\textwidth]{#2}
	\end{figure}}

\newcommand{\codefile}[1]{\lstinputlisting{#1}}

\newcommand{\interval}{\vspace{0.5em}}

\newcommand{\tablestart}{
	\interval
	\begin{longtable}{p{2cm}p{10cm}}
	\hline}
\newcommand{\tableend}{
	\hline
	\end{longtable}
	\interval}

% 改变段间隔
\setlength{\parskip}{0.2em}
\linespread{1.1}

\usepackage{lastpage}
\usepackage{fancyhdr}
\pagestyle{fancy}
\lhead{\space \qquad \space}
\chead{数据处理的两个基本问题 \qquad}
\rhead{\qquad\thepage/\pageref{LastPage}}
\begin{document}

\tableofcontents

\clearpage

\section{两个基本问题}
	两个基本问题是:
	\begin{itemize}
		\item[1.] 处理的数据在什么地方?
		\item[2.] 要处理的数据有多长?
	\end{itemize}

\subsection{bx、si、di和bp}
	bx、si、di和bp这4个寄存器可以单个出现,或者只能以四种组合出现:bx和si、bx和di、bp和si、bp和di。\par

	需要注意的是,如果使用[bp],那么默认段地址存放在ss中,而不是存放在ds中。

\subsection{汇编语言中数据位置的表达}
	关于数据处理的指令可以分为三类:读取、写入和运算。
	对于机器指令,关心的是处理的数据在什么地方。指令在执行前,所要处理的数据可以在三个地方:CPU内部、内存和端口。\par

	汇编语言用如下三个概念来表达数据的位置:
	\begin{itemize}
		\item[1.] 立即数idata,直接包含在机器指令中的数据,也就是在CPU的指令缓冲器中,如下所示:
		\begin{lstlisting}
	mov ax, 1
		\end{lstlisting}

		\item[2.] 寄存器,指令要处理的数据在寄存器中,在汇编指令中给出相应的寄存器名,如下所示:
		\begin{lstlisting}
	mov ax, bx
		\end{lstlisting}

		\item[3.] 段地址和偏移地址,指令要处理的数据在内存中,在汇编指令中以[X]的格式给出,如下所示:
		\begin{lstlisting}
	mov ax, [di]
	mov ax, [bp]
		\end{lstlisting}
	\end{itemize}

\subsubsection{寻址方式}
	寻址方式总结如下图所示:
	\fic{1.png}

\subsection{要处理的数据有多长}
	8086CPU的指令可以处理两种长度的数据:byte和word。
	所以需要在机器指令中指令指令进行的是字操作还是字节操作,有如下几种方式:
	\begin{itemize}
		\item[1.] 通过寄存器名指明要处理的数据的长度。如果用ax、bx、cx、dx,就是字操作。如果用al、bl、cl、dl等,就是字节操作。如下所示:
		\begin{lstlisting}
	mov ax, 1
	mov bx, ds:[0]

	mov al, 1
	mov al, bl
	mov ds:[0], al
		\end{lstlisting}

		\item[2.] 通过X ptr操作符指明内存单元的长度。如果用word ptr,就是字操作。如果用byte ptr,就是字节操作。如下所示:
		\begin{lstlisting}
	mov word ptr ds:[0], 1
	mov byte ptr ds:[0], 1
		\end{lstlisting}

		\item[3.] 部分指令默认了访问的是字单元还是字节单元,比如“push [address]”就是默认字操作。
	\end{itemize}

\section{div指令}
	div指令是除法指令,对它的介绍如下:
	\begin{itemize}
		\item[1.] 除数:有8位和16位两种,在一个寄存器或内存单元中。
		\item[2.] 被除数:如果除数为8位,那么被除数为16位,默认放在AX中。如果除数为16位,那么被除数为32位,默认在DX和AX中存放,DX存放高16位,AX存放低16位。
		\item[3.] 结果:如果除数为8位,那么AL存储商,AH存储余数。如果除数为16位,那么AX存储商,DX存储余数。
		\item[4.] div指令的格式如下:
		\begin{lstlisting}
	div reg
	div 内存单元
		\end{lstlisting}

		可以看出,div指令的操作对象不可以是段寄存器sreg,只能是寄存器reg。
	\end{itemize}

\section{dup}
	dup是一个操作符,和db、dw、dd这些指令配合使用。dup的使用格式如下:
	\begin{itemize}
		\item db 重复的次数 dup (重复的字节型数据)
		\item dw 重复的次数 dup (重复的字型数据)
		\item dd 重复的次数 dup (重复的双字型数据)
	\end{itemize}

	dup的使用如下例所示:
	\begin{lstlisting}
	db 3 dup (0)
	db 3 dup (0, 1, 2)
	db 3 dup ('abc', 'ABC')
	\end{lstlisting}

\end{document}
