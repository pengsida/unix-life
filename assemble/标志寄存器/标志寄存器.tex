% !TeX spellcheck = en_US
%% 字体:方正静蕾简体
%%		 方正粗宋
\documentclass[a4paper,left=2.5cm,right=2.5cm,11pt]{article}

\usepackage[utf8]{inputenc}
\usepackage{fontspec}
\usepackage{cite}
\usepackage{xeCJK}
\usepackage{indentfirst}
\usepackage{titlesec}
\usepackage{longtable}
\usepackage{graphicx}
\usepackage{float}
\usepackage{rotating}
\usepackage{subfigure}
\usepackage{tabu}
\usepackage{amsmath}
\usepackage{setspace}
\usepackage{amsfonts}
\usepackage{appendix}
\usepackage{listings}
\usepackage{xcolor}
\usepackage{geometry}
\setcounter{secnumdepth}{4}
\usepackage{mhchem}
\usepackage{multirow}
\usepackage{extarrows}
\usepackage{hyperref}
\titleformat*{\section}{\LARGE}
\renewcommand\refname{参考文献}
\renewcommand{\abstractname}{\sihao \cjkfzcs 摘{  }要}
%\titleformat{\chapter}{\centering\bfseries\huge\wryh}{}{0.7em}{}{}
%\titleformat{\section}{\LARGE\bf}{\thesection}{1em}{}{}
\titleformat{\subsection}{\Large\bfseries}{\thesubsection}{1em}{}{}
\titleformat{\subsubsection}{\large\bfseries}{\thesubsubsection}{1em}{}{}
\renewcommand{\contentsname}{{\cjkfzcs \centerline{目{  } 录}}}
\setCJKfamilyfont{cjkhwxk}{STXingkai}
\setCJKfamilyfont{cjkfzcs}{STSongti-SC-Regular}
% \setCJKfamilyfont{cjkhwxk}{华文行楷}
% \setCJKfamilyfont{cjkfzcs}{方正粗宋简体}
\newcommand*{\cjkfzcs}{\CJKfamily{cjkfzcs}}
\newcommand*{\cjkhwxk}{\CJKfamily{cjkhwxk}}
\newfontfamily\wryh{Microsoft YaHei}
\newfontfamily\hwzs{STZhongsong}
\newfontfamily\hwst{STSong}
\newfontfamily\hwfs{STFangsong}
\newfontfamily\jljt{MicrosoftYaHei}
\newfontfamily\hwxk{STXingkai}
% \newfontfamily\hwzs{华文中宋}
% \newfontfamily\hwst{华文宋体}
% \newfontfamily\hwfs{华文仿宋}
% \newfontfamily\jljt{方正静蕾简体}
% \newfontfamily\hwxk{华文行楷}
\newcommand{\verylarge}{\fontsize{60pt}{\baselineskip}\selectfont}  
\newcommand{\chuhao}{\fontsize{44.9pt}{\baselineskip}\selectfont}  
\newcommand{\xiaochu}{\fontsize{38.5pt}{\baselineskip}\selectfont}  
\newcommand{\yihao}{\fontsize{27.8pt}{\baselineskip}\selectfont}  
\newcommand{\xiaoyi}{\fontsize{25.7pt}{\baselineskip}\selectfont}  
\newcommand{\erhao}{\fontsize{23.5pt}{\baselineskip}\selectfont}  
\newcommand{\xiaoerhao}{\fontsize{19.3pt}{\baselineskip}\selectfont} 
\newcommand{\sihao}{\fontsize{14pt}{\baselineskip}\selectfont}      % 字号设置  
\newcommand{\xiaosihao}{\fontsize{12pt}{\baselineskip}\selectfont}  % 字号设置  
\newcommand{\wuhao}{\fontsize{10.5pt}{\baselineskip}\selectfont}    % 字号设置  
\newcommand{\xiaowuhao}{\fontsize{9pt}{\baselineskip}\selectfont}   % 字号设置  
\newcommand{\liuhao}{\fontsize{7.875pt}{\baselineskip}\selectfont}  % 字号设置  
\newcommand{\qihao}{\fontsize{5.25pt}{\baselineskip}\selectfont}    % 字号设置 

\usepackage{diagbox}
\usepackage{multirow}
\boldmath
\XeTeXlinebreaklocale "zh"
\XeTeXlinebreakskip = 0pt plus 1pt minus 0.1pt
\definecolor{cred}{rgb}{0.8,0.8,0.8}
\definecolor{cgreen}{rgb}{0,0.3,0}
\definecolor{cpurple}{rgb}{0.5,0,0.35}
\definecolor{cdocblue}{rgb}{0,0,0.3}
\definecolor{cdark}{rgb}{0.95,1.0,1.0}
\lstset{
	language=[x86masm]Assembler,
	numbers=left,
	numberstyle=\tiny\color{black},
	showspaces=false,
	showstringspaces=false,
	basicstyle=\scriptsize,
	keywordstyle=\color{purple},
	commentstyle=\itshape\color{cgreen},
	stringstyle=\color{blue},
	frame=lines,
	% escapeinside=``,
	extendedchars=true, 
	xleftmargin=1em,
	xrightmargin=1em, 
	backgroundcolor=\color{cred},
	aboveskip=1em,
	breaklines=true,
	tabsize=4
} 

\newfontfamily{\consolas}{Consolas}
\newfontfamily{\monaco}{Monaco}
\setmonofont[Mapping={}]{Consolas}	%英文引号之类的正常显示,相当于设置英文字体
\setsansfont{Consolas} %设置英文字体 Monaco, Consolas,  Fantasque Sans Mono
\setmainfont{Times New Roman}

\setCJKmainfont{华文中宋}


\newcommand{\fic}[1]{\begin{figure}[H]
		\center
		\includegraphics[width=0.8\textwidth]{#1}
	\end{figure}}
	
\newcommand{\sizedfic}[2]{\begin{figure}[H]
		\center
		\includegraphics[width=#1\textwidth]{#2}
	\end{figure}}

\newcommand{\codefile}[1]{\lstinputlisting{#1}}

\newcommand{\interval}{\vspace{0.5em}}

\newcommand{\tablestart}{
	\interval
	\begin{longtable}{p{2cm}p{10cm}}
	\hline}
\newcommand{\tableend}{
	\hline
	\end{longtable}
	\interval}

% 改变段间隔
\setlength{\parskip}{0.2em}
\linespread{1.1}

\usepackage{lastpage}
\usepackage{fancyhdr}
\pagestyle{fancy}
\lhead{\space \qquad \space}
\chead{标志寄存器 \qquad}
\rhead{\qquad\thepage/\pageref{LastPage}}
\begin{document}

\tableofcontents

\clearpage

\section{标志寄存器}
	标志寄存器存储的信息通常被称为程序状态字。8086CPU的flag寄存器如下图所示:
	\fic{1.png}

	需要知道的是,一般只有运算指令会影响标志寄存器,而传送指令pop、push、mov等对标志寄存器就没有影响。

\subsection{ZF标志}
	ZF标志是零标志位。如果指令执行后结果为0,那么ZF=1;如果结果不为0,那么ZF=0。如下例所示:
	\begin{lstlisting}
	mov ax, 1
	sub ax, 1
	\end{lstlisting}

\subsection{PF标志}
	PF标志是奇偶标志位。如果指令执行后结果的所有二进制中1的个数为偶数,那么PF=1;如果个数为奇数,那么PF=0。如下例所示:
	\begin{lstlisting}
	mov al, 1
	add al, 10
	; al为00001011B,PF=0
	\end{lstlisting}

\subsection{SF标志}
	SF标志为符号标志位。如果指令执行后结果为负,那么SF=1;如果结果为非负,那么SF=0。
	如果我们将数据当作有符号数来运算的时候,就可以通过SF标志来得知结果的正负。
	如下例所示:
	\begin{lstlisting}
	mov al, 100000001B
	add al, 1
	; al结果为100000010B,SF=1
	\end{lstlisting}

\subsection{CF标志}
	CF标志是进位标志位。在进行无符号数运算的时候,它用于记录运算结果的最高有效位向更高位的进位,或从更高位的借位值。如下例所示:
	\begin{lstlisting}
	mov al, 98H
	add al, al ; CF=1
	mov al, al ; CF=0

	mov al, 97H
	sub al, 98H ; CF=1
	sub al, al  ; CF=0
	\end{lstlisting}

\subsection{OF标志}
	OF标志是溢出标志位。在进行有符号数运算的时候,如果结果超过机器所能表示的范围称为溢出,此时OF=1,如果没有发生溢出,OF=0。

\section{部分运算指令}
\subsection{adc指令}
	adc指令是带进位加法指令,指令格式为:“adc 操作对象1, 操作对象2”,它的功能是:操作对象1=操作对象1+操作对象2+CF。

\subsection{sbb指令}
	sbb指令是带借位减法指令,指令格式为:“sbb 操作对象1, 操作对象2”,它的功能是:操作对象1=操作对象1-操作对象2-CF。

\subsection{cmp指令}
	cmp指令是比较指令,指令格式为:“cmp 操作对象1, 操作对象2”,cmp的功能相当于减法指令,只是不保存结果,然后根据计算结果对标志寄存器进行设置。

\subsubsection{检测比较结果的条件转移指令}
	cmp指令对两个操作数进行比较后,一些条件转移指令根据标志寄存器的相关标志位进行转移。如下例:
	\begin{lstlisting}
	cmp ah, bh
	je s
	add ah, bh
s:	add ah, ah
	\end{lstlisting}

	根据比较结果进行转移的条件转移指令如下图所示:
	\fic{2.png}

\subsection{DF标志}
	DF标志是方向标志位。当DF=0时,串处理指令操作后si,di递增。当DF=1时,串处理指令操作后si,di递减。

\subsubsection{串传送指令}
	串操作指令有movsb,相应的功能如下:
	\begin{lstlisting}
	((es)*16+(di))=((ds)*16+(si))
	如果DF=0,(si)=(si)+1,(di)=(di)+1
	如果DF=1,(si)=(si)-1,(di)=(di)-1
	\end{lstlisting}

	串操作指令也可以传送一个字,指令为movsw,相应的功能为:
	\begin{lstlisting}
	((es)*16+(di))=((ds)*16+(si))
	((es)*16+(di)+1)=((ds)*16+(si)+1)
	如果DF=0,(si)=(si)+2,(di)=(di)+2
	如果DF=1,(si)=(si)-2,(di)=(di)-2
	\end{lstlisting}

\subsubsection{rep指令}
	rep指令一般和movsb、movsw配合使用,格式如下:
	\begin{lstlisting}
	rep movsb
	rep movsw
	\end{lstlisting}

	rep指令根据cx的值重复执行后面的串传送指令,“rep movsw”相当于:
	\begin{lstlisting}
	s: movsw
	   loop s
	\end{lstlisting}

\subsubsection{cld指令和std指令}
	cld指令的功能是将DF标志置0,std指令的功能是将DF标志置1。

\subsection{pushf和popf}
	pushf的功能是将标志寄存器的值压栈,而popf是从栈中弹出数据,送入标志寄存器中。

\end{document}
