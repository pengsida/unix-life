% !TeX spellcheck = en_US
%% 字体:方正静蕾简体
%%		 方正粗宋
\documentclass[a4paper,left=2.5cm,right=2.5cm,11pt]{article}

\usepackage[utf8]{inputenc}
\usepackage{fontspec}
\usepackage{cite}
\usepackage{xeCJK}
\usepackage{indentfirst}
\usepackage{titlesec}
\usepackage{longtable}
\usepackage{graphicx}
\usepackage{float}
\usepackage{rotating}
\usepackage{subfigure}
\usepackage{tabu}
\usepackage{amsmath}
\usepackage{setspace}
\usepackage{amsfonts}
\usepackage{appendix}
\usepackage{listings}
\usepackage{xcolor}
\usepackage{geometry}
\setcounter{secnumdepth}{4}
\usepackage{mhchem}
\usepackage{multirow}
\usepackage{extarrows}
\usepackage{hyperref}
\titleformat*{\section}{\LARGE}
\renewcommand\refname{参考文献}
\renewcommand{\abstractname}{\sihao \cjkfzcs 摘{  }要}
%\titleformat{\chapter}{\centering\bfseries\huge\wryh}{}{0.7em}{}{}
%\titleformat{\section}{\LARGE\bf}{\thesection}{1em}{}{}
\titleformat{\subsection}{\Large\bfseries}{\thesubsection}{1em}{}{}
\titleformat{\subsubsection}{\large\bfseries}{\thesubsubsection}{1em}{}{}
\renewcommand{\contentsname}{{\cjkfzcs \centerline{目{  } 录}}}
\setCJKfamilyfont{cjkhwxk}{STXingkai}
\setCJKfamilyfont{cjkfzcs}{STSongti-SC-Regular}
% \setCJKfamilyfont{cjkhwxk}{华文行楷}
% \setCJKfamilyfont{cjkfzcs}{方正粗宋简体}
\newcommand*{\cjkfzcs}{\CJKfamily{cjkfzcs}}
\newcommand*{\cjkhwxk}{\CJKfamily{cjkhwxk}}
\newfontfamily\wryh{Microsoft YaHei}
\newfontfamily\hwzs{STZhongsong}
\newfontfamily\hwst{STSong}
\newfontfamily\hwfs{STFangsong}
\newfontfamily\jljt{MicrosoftYaHei}
\newfontfamily\hwxk{STXingkai}
% \newfontfamily\hwzs{华文中宋}
% \newfontfamily\hwst{华文宋体}
% \newfontfamily\hwfs{华文仿宋}
% \newfontfamily\jljt{方正静蕾简体}
% \newfontfamily\hwxk{华文行楷}
\newcommand{\verylarge}{\fontsize{60pt}{\baselineskip}\selectfont}  
\newcommand{\chuhao}{\fontsize{44.9pt}{\baselineskip}\selectfont}  
\newcommand{\xiaochu}{\fontsize{38.5pt}{\baselineskip}\selectfont}  
\newcommand{\yihao}{\fontsize{27.8pt}{\baselineskip}\selectfont}  
\newcommand{\xiaoyi}{\fontsize{25.7pt}{\baselineskip}\selectfont}  
\newcommand{\erhao}{\fontsize{23.5pt}{\baselineskip}\selectfont}  
\newcommand{\xiaoerhao}{\fontsize{19.3pt}{\baselineskip}\selectfont} 
\newcommand{\sihao}{\fontsize{14pt}{\baselineskip}\selectfont}      % 字号设置  
\newcommand{\xiaosihao}{\fontsize{12pt}{\baselineskip}\selectfont}  % 字号设置  
\newcommand{\wuhao}{\fontsize{10.5pt}{\baselineskip}\selectfont}    % 字号设置  
\newcommand{\xiaowuhao}{\fontsize{9pt}{\baselineskip}\selectfont}   % 字号设置  
\newcommand{\liuhao}{\fontsize{7.875pt}{\baselineskip}\selectfont}  % 字号设置  
\newcommand{\qihao}{\fontsize{5.25pt}{\baselineskip}\selectfont}    % 字号设置 

\usepackage{diagbox}
\usepackage{multirow}
\boldmath
\XeTeXlinebreaklocale "zh"
\XeTeXlinebreakskip = 0pt plus 1pt minus 0.1pt
\definecolor{cred}{rgb}{0.8,0.8,0.8}
\definecolor{cgreen}{rgb}{0,0.3,0}
\definecolor{cpurple}{rgb}{0.5,0,0.35}
\definecolor{cdocblue}{rgb}{0,0,0.3}
\definecolor{cdark}{rgb}{0.95,1.0,1.0}
\lstset{
	language=[x86masm]Assembler,
	numbers=left,
	numberstyle=\tiny\color{black},
	showspaces=false,
	showstringspaces=false,
	basicstyle=\scriptsize,
	keywordstyle=\color{purple},
	commentstyle=\itshape\color{cgreen},
	stringstyle=\color{blue},
	frame=lines,
	% escapeinside=``,
	extendedchars=true, 
	xleftmargin=1em,
	xrightmargin=1em, 
	backgroundcolor=\color{cred},
	aboveskip=1em,
	breaklines=true,
	tabsize=4
} 

\newfontfamily{\consolas}{Consolas}
\newfontfamily{\monaco}{Monaco}
\setmonofont[Mapping={}]{Consolas}	%英文引号之类的正常显示,相当于设置英文字体
\setsansfont{Consolas} %设置英文字体 Monaco, Consolas,  Fantasque Sans Mono
\setmainfont{Times New Roman}

\setCJKmainfont{华文中宋}


\newcommand{\fic}[1]{\begin{figure}[H]
		\center
		\includegraphics[width=0.8\textwidth]{#1}
	\end{figure}}
	
\newcommand{\sizedfic}[2]{\begin{figure}[H]
		\center
		\includegraphics[width=#1\textwidth]{#2}
	\end{figure}}

\newcommand{\codefile}[1]{\lstinputlisting{#1}}

% 改变段间隔
\setlength{\parskip}{0.2em}
\linespread{1.1}

\usepackage{lastpage}
\usepackage{fancyhdr}
\pagestyle{fancy}
\lhead{\space \qquad \space}
\chead{实现特权级转移 \qquad}
\rhead{\qquad\thepage/\pageref{LastPage}}
\begin{document}

\tableofcontents

\clearpage

\section{实现特权级转移}
\subsection{理论知识}
	特权级转移需要借助堆栈切换。当调用门用于把程序控制转移到一个更高级别的非一致性代码段时,处理器会自动切换到目的代码段特权级的堆栈。
	此时处理器会按照以下步骤切换堆栈:
	\begin{itemize}
		\item 当前任务的TSS段存放着特权级0、1和2的堆栈的初始指针值。处理器会将目的代码段的DPL作为新任务的CPL,并从TSS中选择新栈的SS和ESP。
		\item 将SS和ESP寄存器的当前值压入新栈,并将新栈的段选择符和栈指针加载到SS和ESP。
		\item 将调用门描述符中指定的参数从当前栈压入新栈。参数数目由调用门描述符中的PARAM COUNT字段决定。
		\item 将CS和EIP寄存器的当前值压入新栈,并将目的代码段选择符加载到CS,将调用门选择符中的偏移值加载到EIP中。
	\end{itemize}

	调用过程如下图所示:
	\fic{1.png}

	当调用过程结束后,处理器使用RET执行远返回到一个调用过程。此时CPU会执行以下步骤:
	\begin{itemize}
		\item 检查保存的CS寄存器中的RPL字段值,以确定返回时特权级是否需要改变。
		\item 弹出新栈中的CS和EIP值,并且检查代码段描述符的DPL和代码段选择符的RPL。
		\item 如果返回过程会改变特权级,而且此时RET指令包含一个参数个数操作数,那么就需要在弹出CS和EIP之后,将参数个数值加载到ESP寄存器中,用于丢弃新栈中的参数。
		\item 弹出SS和ESP,从而切换回调用者的堆栈。
		\item 检查DS、ES、FS和GS,如果其中的段选择符指向的段描述符的DPL小于新CPL(仅适用于一致代码段),处理器将用空选择符来加载这个段寄存器。
	\end{itemize}

	综上,使用调用门实现不同特权级之间的调用可以分为两个过程,一个是通过调用门和call指令实现从低特权级转移到高特权级,
	另一个是通过ret指令实现从高特权级到低特权级。

\subsection{代码实现}
\subsubsection{高特权级到低特权级}
	根据上一节可知,处理器通过ret指令实现从高特权级到低特权级。在ret指令执行之前,堆栈中应该已经有了ss、esp、cs和eip。如下图所示:
	\fic{2.png}

	首先添加一个特权级为3的代码段,为了实现代码段转移,我们需要添加一个代码段和相应的堆栈段。
	先是在GDT表中添加该代码段和堆栈段的描述符,然后定义堆栈段ring3和代码段ring3。随后,我们在32位代码段中通过执行retf指令跳转到ring3代码段中。
	\begin{lstlisting}
	; 在GDT中添加相应的代码段和堆栈段
	[SECTION .gdt]
	; ...
	; 特权级为3
	LABEL_DESC_CODE_RING3: Descriptor 0, SegCodeRing3Len-1, DA_C+DA_32+DA_DPL3
	LABEL_DESC_STACK3: Descriptor 0, TopOfStack3, DA_DRWA+DA_32+DA_DPL3
	; ...
	; 请求特权级为3
	SelectorCodeRing3 equ LABEL_DESC_CODE_RING3-LABEL_GDT+SA_RPL3
	SelectorStack3 equ LABEL_DESC_STACK3-LABEL_GDT+SA_RPL3
	; ...
	; 定义堆栈段ring3
	[SECTION .s3]
	ALIGN 32
	[BITS 32]
	LABEL_STACK3:
		; 该堆栈段有512个字节大小
		times 512 db 0
	TopOfStack3 equ $-LABEL_STACK3-1
	; ...
	; 定义代码段ring3
	[SECTION .ring3]
	ALIGN 32
	[BITS 32]
	LABEL_CODE_RING3:
		mov ax, SelectorVideo
		mov gs, ax
		mov edi, (80*14 + 0) * 2
		mov ah, 0Ch
		mov al, '3'
		mov [gs:edi], ax
		jmp $
	SegCodeRing3Len equ $-LABEL_CODE_RING3
	; ...
	[SECTION .s32]
	[BITS 32]
	LABEL_SEG_CODE32:
		; ...
		; 压入ss
		push SelectorStack3
		; 压入esp
		push TopOfStack3
		; 压入cs
		push SelectorCodeRing3
		; 压入eip
		push 0
		; 执行ret指令
		retf
	\end{lstlisting}
\subsubsection{低特权级到高特权级}
	从低特权级到高特权级转移的时候,需要用到TSS,所以要添加TSS段。
	需要根据TSS的结构定义TSS,TSS的结构如下图:
	\fic{3.png}

	\begin{lstlisting}
	[SECTION .gdt]
	LABEL_DESC_TSS: Descriptor 0, TSSLen-1, DA_386TSS
	; ...
	SelectorTSS equ LABEL_DESC_TSS-LABEL_GDT
	; ...
	[SECTION .tss]
	ALIGN
	[BITS 32]
	LABEL_TSS:
		DD 0 ; 前一任务链接
		DD TopOfStack   ; 0级堆栈段基址
		DD SelectorStack ; 0级堆栈选择符
		DD 0 ; 1级堆栈段基址
		DD 0 ; 1级堆栈选择符
		DD 0 ; 2级堆栈段基址
		DD 0 ; 2级堆栈选择符
		DD 0 ; CR3
		DD 0 ; EIP
		DD 0 ; EFLAGS
		DD 0 ; EIP
		DD 0 ; EAX
		DD 0 ; ECX
		DD 0 ; EDX
		DD 0 ; EBX
		DD 0 ; ESP
		DD 0 ; EBP
		DD 0 ; ESI
		DD 0 ; EDI
		DD 0 ; ES
		DD 0 ; CS
		DD 0 ; SS
		DD 0 ; DS
		DD 0 ; FS
		DD 0 ; GS
		DD 0 ; LDT段选择符
		DW 0 ; 调试陷阱T标志位
		DW $-LABEL_TSS+2 ; I/O位图基址
		DB 0ffh ; I/O位图结束标志
	TSSLen equ $-LABEL_TSS
	\end{lstlisting}

	接着添加调用门,用于不同特权级的转移。调用门的添加步骤在第四次中已经讲到了,
	在这里就不再详细论述。添加调用门成功后,就可以在ring3代码段中通过调用门转移到特权级为0的代码段中。
	\begin{lstlisting}
	[SECTION .gdt]
	; ...
	; 定义一个特权级为0的代码段
	LABEL_DESC_CODE_TEST: Descriptor 0, SegCodeDestLen-1, DA_C+DA_32
	; 定义一个能够跳转到0特权级代码段的门描述符
	LABEL_CALL_GATE_TEST: Gate SelectorCodeDest, 0, 0, DA_386Gate+DA_DPL3
	; ...
	SelectorCallGateTest equ LABEL_CALL_GATE_TEST - LABEL_GDT

	; 在32位代码段中加载TSS描述符
	; 需要在特权级变换之前加载TSS描述符
	[SECTION .s32]
	; ...
	mov ax, SelectorTSS
	ltr ax
	; ...

	; 在ring3代码段中通过调用门跳转到特权级为0的代码段中
	[SECTION .ring3]
	ALIGN 32
	[BITS 32]
	LABEL_CODE_RING3:
		mov ax, SelectorVideo
		mov gs, ax
		mov edi, (80 * 14 + 0) * 2
		mov ah, 0Ch
		mov al, '3'
		mov [gs:edi], ax
		call SelectorCallGateTest:0
		jmp $
	SegCodeRing3Len equ $-LABEL_CODE_RING3
	\end{lstlisting}


\end{document}
