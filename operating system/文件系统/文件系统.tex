% !TeX spellcheck = en_US
%% 字体:方正静蕾简体
%%		 方正粗宋
\documentclass[a4paper,left=2.5cm,right=2.5cm,11pt]{article}

\usepackage[utf8]{inputenc}
\usepackage{fontspec}
\usepackage{cite}
\usepackage{xeCJK}
\usepackage{indentfirst}
\usepackage{titlesec}
\usepackage{longtable}
\usepackage{graphicx}
\usepackage{float}
\usepackage{rotating}
\usepackage{subfigure}
\usepackage{tabu}
\usepackage{amsmath}
\usepackage{setspace}
\usepackage{amsfonts}
\usepackage{appendix}
\usepackage{listings}
\usepackage{xcolor}
\usepackage{geometry}
\setcounter{secnumdepth}{4}
\usepackage{mhchem}
\usepackage{multirow}
\usepackage{extarrows}
\usepackage{hyperref}
\titleformat*{\section}{\LARGE}
\renewcommand\refname{参考文献}
\renewcommand{\abstractname}{\sihao \cjkfzcs 摘{  }要}
%\titleformat{\chapter}{\centering\bfseries\huge\wryh}{}{0.7em}{}{}
%\titleformat{\section}{\LARGE\bf}{\thesection}{1em}{}{}
\titleformat{\subsection}{\Large\bfseries}{\thesubsection}{1em}{}{}
\titleformat{\subsubsection}{\large\bfseries}{\thesubsubsection}{1em}{}{}
\renewcommand{\contentsname}{{\cjkfzcs \centerline{目{  } 录}}}
\setCJKfamilyfont{cjkhwxk}{STXingkai}
\setCJKfamilyfont{cjkfzcs}{STSongti-SC-Regular}
% \setCJKfamilyfont{cjkhwxk}{华文行楷}
% \setCJKfamilyfont{cjkfzcs}{方正粗宋简体}
\newcommand*{\cjkfzcs}{\CJKfamily{cjkfzcs}}
\newcommand*{\cjkhwxk}{\CJKfamily{cjkhwxk}}
\newfontfamily\wryh{Microsoft YaHei}
\newfontfamily\hwzs{STZhongsong}
\newfontfamily\hwst{STSong}
\newfontfamily\hwfs{STFangsong}
\newfontfamily\jljt{MicrosoftYaHei}
\newfontfamily\hwxk{STXingkai}
% \newfontfamily\hwzs{华文中宋}
% \newfontfamily\hwst{华文宋体}
% \newfontfamily\hwfs{华文仿宋}
% \newfontfamily\jljt{方正静蕾简体}
% \newfontfamily\hwxk{华文行楷}
\newcommand{\verylarge}{\fontsize{60pt}{\baselineskip}\selectfont}  
\newcommand{\chuhao}{\fontsize{44.9pt}{\baselineskip}\selectfont}  
\newcommand{\xiaochu}{\fontsize{38.5pt}{\baselineskip}\selectfont}  
\newcommand{\yihao}{\fontsize{27.8pt}{\baselineskip}\selectfont}  
\newcommand{\xiaoyi}{\fontsize{25.7pt}{\baselineskip}\selectfont}  
\newcommand{\erhao}{\fontsize{23.5pt}{\baselineskip}\selectfont}  
\newcommand{\xiaoerhao}{\fontsize{19.3pt}{\baselineskip}\selectfont} 
\newcommand{\sihao}{\fontsize{14pt}{\baselineskip}\selectfont}      % 字号设置  
\newcommand{\xiaosihao}{\fontsize{12pt}{\baselineskip}\selectfont}  % 字号设置  
\newcommand{\wuhao}{\fontsize{10.5pt}{\baselineskip}\selectfont}    % 字号设置  
\newcommand{\xiaowuhao}{\fontsize{9pt}{\baselineskip}\selectfont}   % 字号设置  
\newcommand{\liuhao}{\fontsize{7.875pt}{\baselineskip}\selectfont}  % 字号设置  
\newcommand{\qihao}{\fontsize{5.25pt}{\baselineskip}\selectfont}    % 字号设置 

\usepackage{diagbox}
\usepackage{multirow}
\boldmath
\XeTeXlinebreaklocale "zh"
\XeTeXlinebreakskip = 0pt plus 1pt minus 0.1pt
\definecolor{cred}{rgb}{0.8,0.8,0.8}
\definecolor{cgreen}{rgb}{0,0.3,0}
\definecolor{cpurple}{rgb}{0.5,0,0.35}
\definecolor{cdocblue}{rgb}{0,0,0.3}
\definecolor{cdark}{rgb}{0.95,1.0,1.0}
\lstset{
	language=[x86masm]Assembler,
	numbers=left,
	numberstyle=\tiny\color{black},
	showspaces=false,
	showstringspaces=false,
	basicstyle=\scriptsize,
	keywordstyle=\color{purple},
	commentstyle=\itshape\color{cgreen},
	stringstyle=\color{blue},
	frame=lines,
	% escapeinside=``,
	extendedchars=true, 
	xleftmargin=1em,
	xrightmargin=1em, 
	backgroundcolor=\color{cred},
	aboveskip=1em,
	breaklines=true,
	tabsize=4
} 

\newfontfamily{\consolas}{Consolas}
\newfontfamily{\monaco}{Monaco}
\setmonofont[Mapping={}]{Consolas}	%英文引号之类的正常显示,相当于设置英文字体
\setsansfont{Consolas} %设置英文字体 Monaco, Consolas,  Fantasque Sans Mono
\setmainfont{Times New Roman}

\setCJKmainfont{华文中宋}


\newcommand{\fic}[1]{\begin{figure}[H]
		\center
		\includegraphics[width=0.8\textwidth]{#1}
	\end{figure}}
	
\newcommand{\sizedfic}[2]{\begin{figure}[H]
		\center
		\includegraphics[width=#1\textwidth]{#2}
	\end{figure}}

\newcommand{\codefile}[1]{\lstinputlisting{#1}}

% 改变段间隔
\setlength{\parskip}{0.2em}
\linespread{1.1}

\usepackage{lastpage}
\usepackage{fancyhdr}
\pagestyle{fancy}
\lhead{\space \qquad \space}
\chead{文件系统 \qquad}
\rhead{\qquad\thepage/\pageref{LastPage}}
\begin{document}

\tableofcontents

\clearpage

\section{硬盘驱动程序}
	对硬盘控制器的操作通过I/O端口来进行,分为两组端口,分别是命令块寄存器和控制块寄存器。如下图所示:
	\fic{1.png}

	一块PC主板上有两个IDE口,分为Primary和Secondary。对于IDE口的访问通过I/O端口地址来区分。
	每个IDE口上可以连接两个设备,分别为主设备和从设备。通过Device寄存器可以选择哪个设备,如果Device寄存器第四位为0,就是操作主设备,如果为1就是操作从设备。

\subsection{对命令块寄存器的操作}
\subsubsection{硬盘操作的端口}
	\begin{lstlisting}
	#define REG_DATA	0x1F0
	#define REG_FEATURES	0x1F1
	#define REG_ERROR	REG_FEATURES
	#define REG_NSECTOR	0x1F2
	#define REG_LBA_LOW	0x1F3
	#define REG_LBA_MID	0x1F4
	#define REG_LBA_HIGH	0x1F5
	#define REG_DEVICE	0x1F6
	#define REG_STATUS	0x1F7
	#define REG_CMD		0x1F7
	\end{lstlisting}

	状态寄存器有如下几位:
	\begin{lstlisting}
	#define	STATUS_BSY	0x80
	#define	STATUS_DRDY	0x40
	#define	STATUS_DFSE	0x20
	#define	STATUS_DSC	0x10
	#define	STATUS_DRQ	0x08
	#define	STATUS_CORR	0x04
	#define	STATUS_IDX	0x02
	#define	STATUS_ERR	0x01
	\end{lstlisting}

	知道端口以后,我们当然想着如何操作它。首先我们定义一个结构体,用于存放操作端口的信息:
	\begin{lstlisting}
	struct hd_cmd {
		u8	features;
		u8	count;
		u8	lba_low;
		u8	lba_mid;
		u8	lba_high;
		u8	device;
		u8	command;
	};
	\end{lstlisting}

	随后编写操作端口的函数:
	\begin{lstlisting}
	PRIVATE void hd_cmd_out(struct hd_cmd* cmd)
	{
		if(!waitfor(STATUS_BSY, 0, HD_TIMEOUT))
			panic("hd error");
		
		// 填充命令块寄存器的信息
		out_byte(REG_DEV_CTRL, 0);
		out_byte(REG_FEATURES, cmd->features);
		out_byte(REG_NSECTOR, cmd->count);
		out_byte(REG_LBA_LOW, cmd->lba_low);
		out_byte(REG_LBA_MID, cmd->lba_mid);
		out_byte(REG_LBA_HIGH, cmd->lba_high);
		out_byte(REG_DEVICE, cmd->device);
		
		// 写入命令
		out_byte(REG_CMD, cmd->command);
	}
	\end{lstlisting}

	读端口中的数据的函数如下:
	\begin{lstlisting}
	; void port_read(u16 port, void* buf, int n);
	port_read:
		mov	edx, [esp + 4]		; port
		mov	edi, [esp + 4 + 4]	; buf
		mov	ecx, [esp + 4 + 4 + 4]	; n
		shr	ecx, 1
		cld
		rep	insw ; 从DX指出的外设端口输入一个字节或字到由ES: DI指定的存储器中
		ret
	\end{lstlisting}

\subsubsection{Device寄存器与LBA三个寄存器}
	在操作端口的时候,我们通过Device寄存器指定设备。Device寄存器如下图所示:
	\fic{2.png}

	寄存器有三部分:
	\begin{itemize}
		\item[1.] DRV位,用于指定主设备还是从设备。
		\item[2.] LBA模式位,用于指定操作模式。如果LBA位为0,对磁盘的操作使用CHS模式。如果LBA位为1,对磁盘的操作使用LBA模式。
		\item[3.] 低四位,在CHS模式中表示磁头号,在LBA模式中表示LBA的24~27位。
	\end{itemize}

	LBA模式:用于寻址硬盘,0~7位由LBA Low寄存器指定,8~15位由LBA Mid寄存器指定,16~23位由LBA High指定,24~27位由Device寄存器的低四位指定。\par

	为了更简单地表示Device寄存器,我使用一个宏来表示寄存器:
	\begin{lstlisting}
	#define MAKE_DEVICE_REG(lba, drv, lba_highest) (((lba) << 6) | \
							                        ((drv) << 4) | \
													(lba_highest & 0xF) | 0xA0)
	\end{lstlisting}

	

\subsection{硬盘驱动程序}
	首先写一个硬盘驱动程序的系统进程,这个驱动程序接收消息类型如下:
	\begin{itemize}
		\item[1.] 接收DEV\_IDENTIFY,用于得到硬盘信息。
		\item[2.] 接收DEV\_OPEN,用于遍历所有分区。
		\item[3.] 接收DEV\_CLOSE,用于关闭硬盘。
		\item[4.] 接收DEV\_READ,用于读取硬盘内容。
		\item[5.] 接收DEV\_WRITE,用于写入硬盘。
		\item[6.] 接收DEV\_IOCTL,用于把请求的设备的起始扇区和扇区数目返回给调用者。
	\end{itemize}

	\begin{lstlisting}
	PUBLIC void task_hd()
	{
		MESSAGE msg;

		init_hd();

		while(1)
		{
			send_rec(RECEIVE, ANY, &msg);
			int src = msg.source;
			switch(msg.type)
			{
				case DEV_IDENTIFY:
					hd_identify(0);
					break;
				case DEV_OPEN:
					hd_open(msg.DEVICE);
					break;
				default:
					dump_msg("HD driver::unknown msg", &msg);
					spin("FS::main_loop (invalid msg.type)");
					break;
			}
			send_rec(SEND, src, &msg);
		}
	}
	\end{lstlisting}

\subsubsection{初始化硬盘中断的函数}
	这里的硬盘中断是为了和之后的硬盘驱动程序进行交互。
	硬盘驱动程序操作硬盘,随即进入阻塞等待硬盘工作的完成。
	硬盘工作完成时,将触发硬盘中断处理程序hd\_handler(),随即通知硬盘驱动程序。

	\begin{lstlisting}
	PRIVATE void init_hd()
	{
		u8 * pNrDrives = (u8*)(0x475);
		printl("NrDrives:%d.\n", *pNrDrives);
		assert(*pNrDrives);

		put_irq_handler(AT_WINI_IRQ, hd_handler);
		enable_irq(CASCADE_IRQ);
		enable_irq(AT_WINI_IRQ);
	}
	\end{lstlisting}

	硬盘中断处理程序如下:
	\begin{lstlisting}
	PUBLIC void hd_handler(int irq)
	{
		hd_status = in_byte(REG_STATUS);
		inform_int(TASK_HD);
	}

	PUBLIC void inform_int(int task_nr)
	{
		struct proc* p = proc_table + task_nr;

		// 如果TASK_HD进程正在等待硬盘中断
		if ((p->p_flags & RECEIVING) && 
			((p->p_recvfrom == INTERRUPT) || (p->p_recvfrom == ANY))) {
			p->p_msg->source = INTERRUPT;
			p->p_msg->type = HARD_INT;
			p->p_msg = 0;
			p->has_int_msg = 0;
			
			// 解除TASK_HD进程的阻塞
			p->p_flags &= ~RECEIVING; 
			p->p_recvfrom = NO_TASK;
			assert(p->p_flags == 0);
			unblock(p);

			assert(p->p_flags == 0);
			assert(p->p_msg == 0);
			assert(p->p_recvfrom == NO_TASK);
			assert(p->p_sendto == NO_TASK);
		}
		else {
			p->has_int_msg = 1;
		}
	}
	\end{lstlisting}

\subsubsection{得到硬盘信息}

	编写得到硬盘信息的函数,实现它的想法如下:
	\begin{itemize}
		\item[1.] 首先向Device寄存器的第四位指定驱动器,0表示主设备,1表示从设备。
		\item[2.] 然后填充命令块寄存器的其他寄存器。
		\item[3.] 然后向Command寄存器写入十六进制ECh,从而获得Data数据。
		\item[4.] 等待硬盘中断,原因随后解释。
		\item[5.] 最后我们通过Data寄存器读取数据,总共256个数据。
	\end{itemize}

	\begin{lstlisting}
	PRIVATE void hd_identify(int drive)
	{
		struct hd_cmd cmd;
		cmd.device = MAKE_DEVICE_REG(0, drive, 0);
		cmd.command = ATA_IDENTIFY;
		hd_cmd_out(&cmd);
		interrupt_wait();
		port_read(REG_DATA, hdbuf, SECTOR_SIZE);
		print_identify_info((u16*)hdbuf);
	}
	\end{lstlisting}

	当我们硬盘驱动程序发出指令后,需要等待硬盘工作的完成。我们这里设计硬盘完成工作的时候,会触发中断,执行中断处理程序hd\_handler(),然后调用inform\_int()函数,
	解除硬盘驱动的阻塞。
	等待硬盘中断的函数如下:
	\begin{lstlisting}
	PRIVATE void interrupt_wait()
	{
		MESSAGE msg;
		send_rec(RECEIVE, INTERRUPT, &msg);
	}
	\end{lstlisting}

	现在我们的hdbuf就存放着256字节的硬盘信息,具体参数可以从AT Attachment with Packet Interface文档中茶的,
	我们这里只打印出如下的参数:
	\fic{3.png}

	打印硬盘信息的函数如下:
	\begin{lstlisting}
	PRIVATE void print_identify_info(u16* hdinfo)
	{
		int i, k;
		char s[64];
		struct iden_info_ascii{
			int idx;
			int len;
			char* desc;
		} iinfo[] = {
			{10, 20, "HD SN"}, // 定义序列号
			{27, 40, "HD Model"} // 定义型号
		}

		for (k = 0; k < sizeof(iinfo)/sizeof(iinfo[0]); k++)
		{
			char* p = (char*)&hdinfo[iinfo[k].idx];
			for(i = 0; i < iinfo[k].len/2; i++)
			{
				s[i*2 + 1] = *p++;
				s[i*2] = *p++;
			}
			s[i*2] = 0;
			printl("%s: %s\n", iinfo[k].desc, s);
		}

		// 得到支持的功能
		int capabilities = hdinfo[49];
		printl("LBA supported: %s\n", (capabilities & 0x0200) ? "Yes" : "No");

		// 得到支持的命令集
		int cmd_set_supported = hdinfo[83];
		printl("LBA48 supported: %s\n", (cmd_set_supported & 0x0400) ? "Yes" : "No");

		// 得到用户可用的最大扇区数
		int sectors = ((int)hdinfo[61] << 16) + hdinfo[60];
		printl("HD size: %dMB\n", sectors * 512 / 1000000);
	}
	\end{lstlisting}

\subsubsection{遍历所有分区}
	函数如下:
	\begin{lstlisting}
	PRIVATE void hd_open(int device)
	{
		int drive = DRV_OF_DEV(device);
		assert(drive == 0);	/* only one drive */

		// 读取硬盘驱动器的信息
		hd_identify(drive);

		// 打印分区表
		if (hd_info[drive].open_cnt++ == 0) {
			partition(drive * (NR_PART_PER_DRIVE + 1), P_PRIMARY);
			print_hdinfo(&hd_info[drive]);
		}
	}
	\end{lstlisting}

	定义一个硬盘信息的结构体,每个硬盘都应该有一个hd\_info结构体,
	其中primary成员用来记录所有主分区的起始扇区和扇区数目,
	logical成员用来记录所有逻辑分区的起始扇区和扇区数目。
	\begin{lstlisting}
	struct part_info {
		u32	base;	// 起始扇区
		u32	size;   // 扇区数目
	};

	/* main drive struct, one entry per drive */
	struct hd_info
	{
		int			open_cnt;
		struct part_info	primary[NR_PRIM_PER_DRIVE];
		struct part_info	logical[NR_SUB_PER_DRIVE];
	};
	\end{lstlisting}

	在实现partition()函数之前,先来了解主分区和扩展分区:
	\begin{lstlisting}
	磁盘分区有三种形式:主分区、扩展分区和逻辑分区。

	主分区最多有四个。如果要在硬盘上安装操作系统,那么这个必须有一个主分区。
	主分区中不能再划分其他类型的分区,每个主分区相当于一个逻辑磁盘。

	扩展分区不能直接使用,必须将它划分为若干个逻辑分区才能使用。
	逻辑分区相当于一个逻辑磁盘,逻辑分区必须在扩展分区中划分。
	扩展分区中可能又有一个扩展分区。

	由主分区和逻辑分区构成的逻辑磁盘称为驱动器(Driver)或卷(Volume)。
	\end{lstlisting}
	
	再来看一下硬盘分区表,
	硬盘分区表是一个结构体数组,数组的每个成员是一个16字节的结构体,如下所示:
	\fic{4.png}

	这块分区表位于分区起始扇区的1BEh处,我们可以通过类似于hd\_identify()函数的方法得到它。
	我们可以根据分区表来判断分区类型、起始扇区LBA和扇区数目,其中关键信息是分区类型,
	如果分区类型是5,说明它是扩展分区,需要继续递归,否则直接打印分区信息。
	定义一个分区表的结构体:
	\begin{lstlisting}
	struct part_ent {
		u8 boot_ind;	 // 表示是否可以引导
		u8 start_head;	 // 起始柱头号
		u8 start_sector; // 起始扇区号
		u8 start_cyl;	// 起始柱面号的低八位
		u8 sys_id;	    // 分区类型
		u8 end_head;	// 结束柱头号
		u8 end_sector;	// 结束扇区号
		u8 end_cyl;		// 结束柱面号的低八位
		u32 start_sect;	// 起始扇区的LBA
		u32 nr_sects; 	// 扇区数目
	} PARTITION_ENTRY;
	\end{lstlisting}

	我们现在对遍历一块硬盘的分区有了基本的头绪,但还有一件事需要知道,就是向command寄存器发送命令时,
	需要指定扇区的LBA,这个需要填充LBA low、LBA mid、LBA high和Device寄存器。这样我们就需要分区的设备号。
	在操作系统中,有两种设备号:主设备号和次设备号。主设备号告诉操作系统应该用哪个驱动程序来处理,次设备号告诉驱动程序具体是哪个设备。
	我们这里遍历分区用的当然是次设备号。给定一个次设备号,我们可以很容易地计算出它是主分区还是扩展分区。给定一个分区的名称,我们也很容易计算出其次设备号。\par

	定义一些与设备号相关的宏:
	\begin{lstlisting}
	#define	MAX_DRIVES		2  // 支持硬盘的数目
	#define	NR_PART_PER_DRIVE	4 // 每个硬盘最多有多少主分区
	#define	NR_SUB_PER_PART		16 // 每个扩展分区最多有多少个逻辑分区
	#define	NR_SUB_PER_DRIVE	(NR_SUB_PER_PART * NR_PART_PER_DRIVE)
	#define	NR_PRIM_PER_DRIVE	(NR_PART_PER_DRIVE + 1)

	#define	MAX_PRIM		(MAX_DRIVES * NR_PRIM_PER_DRIVE - 1)
	#define	MAX_SUBPARTITIONS	(NR_SUB_PER_DRIVE * MAX_DRIVES)

	#define	MINOR_hd1a		0x10 // 扩展分区最小设备号
	#define	MINOR_hd2a		(MINOR_hd1a+NR_SUB_PER_PART)
	\end{lstlisting}

	得到一块分区在哪个硬盘上:
	\begin{lstlisting}
	#define	DRV_OF_DEV(dev) (dev <= MAX_PRIM ? \
				dev / NR_PRIM_PER_DRIVE : \
				(dev - MINOR_hd1a) / NR_SUB_PER_DRIVE)
	\end{lstlisting}

	partition()函数用于获取硬盘分区表,实现的想法如下:
	\begin{itemize}
		\item[1.] 首先判断是主分区还是扩展分区。
		\item[2.] 然后用get\_part\_table()函数,也就是读取该分区的信息。
		\item[3.] 因为主分区中可以包含扩展分区,所以继续递归partition()函数。
	\end{itemize}

	\begin{lstlisting}
	PRIVATE void partition(int device, int style)
	{
		int i;
		int drive = DRV_OF_DEV(device);
		struct hd_info * hdi = &hd_info[drive];

		struct part_ent part_tbl[NR_SUB_PER_DRIVE];

		// 如果是主分区
		if (style == P_PRIMARY) {
			get_part_table(drive, drive, part_tbl);

			int nr_prim_parts = 0;
			for (i = 0; i < NR_PART_PER_DRIVE; i++) { /* 0~3 */
				if (part_tbl[i].sys_id == NO_PART) 
					continue;

				nr_prim_parts++;
				int dev_nr = i + 1;		  /* 1~4 */
				hdi->primary[dev_nr].base = part_tbl[i].start_sect;
				hdi->primary[dev_nr].size = part_tbl[i].nr_sects;

				if (part_tbl[i].sys_id == EXT_PART) /* extended */
					partition(device + dev_nr, P_EXTENDED);
			}
			assert(nr_prim_parts != 0);
		}
		// 如果是扩展分区
		else if (style == P_EXTENDED) {
			int j = device % NR_PRIM_PER_DRIVE; /* 1~4 */
			int ext_start_sect = hdi->primary[j].base;
			int s = ext_start_sect;
			int nr_1st_sub = (j - 1) * NR_SUB_PER_PART; /* 0/16/32/48 */

			for (i = 0; i < NR_SUB_PER_PART; i++) {
				int dev_nr = nr_1st_sub + i;/* 0~15/16~31/32~47/48~63 */

				get_part_table(drive, s, part_tbl);

				hdi->logical[dev_nr].base = s + part_tbl[0].start_sect;
				hdi->logical[dev_nr].size = part_tbl[0].nr_sects;

				s = ext_start_sect + part_tbl[1].start_sect;

				/* no more logical partitions
				in this extended partition */
				if (part_tbl[1].sys_id == NO_PART)
					break;
			}
		}
		else {
			assert(0);
		}
	}

	PRIVATE void get_part_table(int drive, int sect_nr, struct part_ent * entry)
	{
		struct hd_cmd cmd;

		// 先填充命令块寄存器
		cmd.features	= 0;
		cmd.count	= 1;
		cmd.lba_low	= sect_nr & 0xFF;
		cmd.lba_mid	= (sect_nr >>  8) & 0xFF;
		cmd.lba_high	= (sect_nr >> 16) & 0xFF;
		cmd.device	= MAKE_DEVICE_REG(1, /* LBA mode*/
						drive,
						(sect_nr >> 24) & 0xF);
		
		// 再指定命令ATA_READ
		cmd.command	= ATA_READ;
		hd_cmd_out(&cmd);
		interrupt_wait();

		// 读取数据到hdbuf缓冲区
		port_read(REG_DATA, hdbuf, SECTOR_SIZE);

		memcpy(entry,
			hdbuf + PARTITION_TABLE_OFFSET,
			sizeof(struct part_ent) * NR_PART_PER_DRIVE);
	}
	\end{lstlisting}

	打印分区信息:
	\begin{lstlisting}
	PRIVATE void print_hdinfo(struct hd_info * hdi)
	{
		int i;
		for (i = 0; i < NR_PART_PER_DRIVE + 1; i++) {
			printl("%sPART_%d: base %d(0x%x), size %d(0x%x) (in sector)\n",
				i == 0 ? " " : "     ",
				i,
				hdi->primary[i].base,
				hdi->primary[i].base,
				hdi->primary[i].size,
				hdi->primary[i].size);
		}
		for (i = 0; i < NR_SUB_PER_DRIVE; i++) {
			if (hdi->logical[i].size == 0)
				continue;
			printl("         "
				"%d: base %d(0x%x), size %d(0x%x) (in sector)\n",
				i,
				hdi->logical[i].base,
				hdi->logical[i].base,
				hdi->logical[i].size,
				hdi->logical[i].size);
		}
	}
	\end{lstlisting}

\subsubsection{关闭硬盘}
	关闭硬盘的函数很简单,也就是将hd\_info的结构体成员open\_cnt减一,如下所示:
	\begin{lstlisting}
	PRIVATE void hd_close(int device)
	{
		int drive = DRV_OF_DEV(device);
		assert(drive == 0);	/* only one drive */

		hd_info[drive].open_cnt--;
	}
	\end{lstlisting}

\subsubsection{读写硬盘内容}
	实现这个函数的想法如下:
	\begin{itemize}
		\item[1.] 根据Message得到我们所要操作的扇区的扇区号。
		\item[2.] 根据Message的类型,决定是向硬盘发出ATA\_READ命令还是ATA\_WRITE命令。
		\item[3.] 如果是DEV\_READ,就调用port\_read()函数,然后将数据拷贝到消息体中的缓存区。
		\item[4.] 如果是DEV\_WRITE,就调用port\_write()函数,将消息体中附带的数据写入硬盘。
	\end{itemize}

	\begin{lstlisting}
	PRIVATE void hd_rdwt(MESSAGE * p)
	{
		int drive = DRV_OF_DEV(p->DEVICE);

		u64 pos = p->POSITION;
		assert((pos >> SECTOR_SIZE_SHIFT) < (1 << 31));

		assert((pos & 0x1FF) == 0);

		// 得到扇区号
		u32 sect_nr = (u32)(pos >> SECTOR_SIZE_SHIFT); /* pos / SECTOR_SIZE */
		int logidx = (p->DEVICE - MINOR_hd1a) % NR_SUB_PER_DRIVE;
		sect_nr += p->DEVICE < MAX_PRIM ?
			hd_info[drive].primary[p->DEVICE].base :
			hd_info[drive].logical[logidx].base;
		
		// 向硬盘发出命令
		struct hd_cmd cmd;
		cmd.features	= 0;
		cmd.count	= (p->CNT + SECTOR_SIZE - 1) / SECTOR_SIZE;
		cmd.lba_low	= sect_nr & 0xFF;
		cmd.lba_mid	= (sect_nr >>  8) & 0xFF;
		cmd.lba_high	= (sect_nr >> 16) & 0xFF;
		cmd.device	= MAKE_DEVICE_REG(1, drive, (sect_nr >> 24) & 0xF);
		cmd.command	= (p->type == DEV_READ) ? ATA_READ : ATA_WRITE;
		hd_cmd_out(&cmd);

		int bytes_left = p->CNT;
		void * la = (void*)va2la(p->PROC_NR, p->BUF);

		while (bytes_left) {
			int bytes = min(SECTOR_SIZE, bytes_left);
			if (p->type == DEV_READ) {
				interrupt_wait();
				port_read(REG_DATA, hdbuf, SECTOR_SIZE);
				phys_copy(la, (void*)va2la(TASK_HD, hdbuf), bytes);
			}
			else {
				if (!waitfor(STATUS_DRQ, STATUS_DRQ, HD_TIMEOUT))
					panic("hd writing error.");

				port_write(REG_DATA, la, bytes);
				interrupt_wait();
			}
			bytes_left -= SECTOR_SIZE;
			la += SECTOR_SIZE;
		}
	}
	\end{lstlisting}

	这里再写一个port\_write()函数:
	\begin{lstlisting}
	port_write:
		mov	edx, [esp + 4]		; port
		mov	esi, [esp + 4 + 4]	; buf
		mov	ecx, [esp + 4 + 4 + 4]	; n
		shr	ecx, 1
		cld
		rep	outsw
		ret
	\end{lstlisting}

\subsubsection{处理IOCTL}
	目前这个函数只处理一个消息类型,也就是DIOCTL\_GET\_GEO,只是把硬盘的起始扇区和扇区数目返回给调用者:
	\begin{lstlisting}
	PRIVATE void hd_ioctl(MESSAGE * p)
	{
		int device = p->DEVICE;
		int drive = DRV_OF_DEV(device);

		struct hd_info * hdi = &hd_info[drive];

		if (p->REQUEST == DIOCTL_GET_GEO) {
			void * dst = va2la(p->PROC_NR, p->BUF);
			void * src = va2la(TASK_HD,
					device < MAX_PRIM ?
					&hdi->primary[device] :
					&hdi->logical[(device - MINOR_hd1a) %
							NR_SUB_PER_DRIVE]);

			phys_copy(dst, src, sizeof(struct part_info));
		}
		else {
			assert(0);
		}
	}
	\end{lstlisting}

\end{document}
