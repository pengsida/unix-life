% !TeX spellcheck = en_US
%% 字体:方正静蕾简体
%%		 方正粗宋
\documentclass[a4paper,left=2.5cm,right=2.5cm,11pt]{article}

\usepackage[utf8]{inputenc}
\usepackage{fontspec}
\usepackage{cite}
\usepackage{xeCJK}
\usepackage{indentfirst}
\usepackage{titlesec}
\usepackage{longtable}
\usepackage{graphicx}
\usepackage{float}
\usepackage{rotating}
\usepackage{subfigure}
\usepackage{tabu}
\usepackage{amsmath}
\usepackage{setspace}
\usepackage{amsfonts}
\usepackage{appendix}
\usepackage{listings}
\usepackage{xcolor}
\usepackage{geometry}
\setcounter{secnumdepth}{4}
\usepackage{mhchem}
\usepackage{multirow}
\usepackage{extarrows}
\usepackage{hyperref}
\titleformat*{\section}{\LARGE}
\renewcommand\refname{参考文献}
\renewcommand{\abstractname}{\sihao \cjkfzcs 摘{  }要}
%\titleformat{\chapter}{\centering\bfseries\huge\wryh}{}{0.7em}{}{}
%\titleformat{\section}{\LARGE\bf}{\thesection}{1em}{}{}
\titleformat{\subsection}{\Large\bfseries}{\thesubsection}{1em}{}{}
\titleformat{\subsubsection}{\large\bfseries}{\thesubsubsection}{1em}{}{}
\renewcommand{\contentsname}{{\cjkfzcs \centerline{目{  } 录}}}
\setCJKfamilyfont{cjkhwxk}{STXingkai}
\setCJKfamilyfont{cjkfzcs}{STSongti-SC-Regular}
% \setCJKfamilyfont{cjkhwxk}{华文行楷}
% \setCJKfamilyfont{cjkfzcs}{方正粗宋简体}
\newcommand*{\cjkfzcs}{\CJKfamily{cjkfzcs}}
\newcommand*{\cjkhwxk}{\CJKfamily{cjkhwxk}}
\newfontfamily\wryh{Microsoft YaHei}
\newfontfamily\hwzs{STZhongsong}
\newfontfamily\hwst{STSong}
\newfontfamily\hwfs{STFangsong}
\newfontfamily\jljt{MicrosoftYaHei}
\newfontfamily\hwxk{STXingkai}
% \newfontfamily\hwzs{华文中宋}
% \newfontfamily\hwst{华文宋体}
% \newfontfamily\hwfs{华文仿宋}
% \newfontfamily\jljt{方正静蕾简体}
% \newfontfamily\hwxk{华文行楷}
\newcommand{\verylarge}{\fontsize{60pt}{\baselineskip}\selectfont}  
\newcommand{\chuhao}{\fontsize{44.9pt}{\baselineskip}\selectfont}  
\newcommand{\xiaochu}{\fontsize{38.5pt}{\baselineskip}\selectfont}  
\newcommand{\yihao}{\fontsize{27.8pt}{\baselineskip}\selectfont}  
\newcommand{\xiaoyi}{\fontsize{25.7pt}{\baselineskip}\selectfont}  
\newcommand{\erhao}{\fontsize{23.5pt}{\baselineskip}\selectfont}  
\newcommand{\xiaoerhao}{\fontsize{19.3pt}{\baselineskip}\selectfont} 
\newcommand{\sihao}{\fontsize{14pt}{\baselineskip}\selectfont}      % 字号设置  
\newcommand{\xiaosihao}{\fontsize{12pt}{\baselineskip}\selectfont}  % 字号设置  
\newcommand{\wuhao}{\fontsize{10.5pt}{\baselineskip}\selectfont}    % 字号设置  
\newcommand{\xiaowuhao}{\fontsize{9pt}{\baselineskip}\selectfont}   % 字号设置  
\newcommand{\liuhao}{\fontsize{7.875pt}{\baselineskip}\selectfont}  % 字号设置  
\newcommand{\qihao}{\fontsize{5.25pt}{\baselineskip}\selectfont}    % 字号设置 

\usepackage{diagbox}
\usepackage{multirow}
\boldmath
\XeTeXlinebreaklocale "zh"
\XeTeXlinebreakskip = 0pt plus 1pt minus 0.1pt
\definecolor{cred}{rgb}{0.8,0.8,0.8}
\definecolor{cgreen}{rgb}{0,0.3,0}
\definecolor{cpurple}{rgb}{0.5,0,0.35}
\definecolor{cdocblue}{rgb}{0,0,0.3}
\definecolor{cdark}{rgb}{0.95,1.0,1.0}
\lstset{
	language=[x86masm]Assembler,
	numbers=left,
	numberstyle=\tiny\color{black},
	showspaces=false,
	showstringspaces=false,
	basicstyle=\scriptsize,
	keywordstyle=\color{purple},
	commentstyle=\itshape\color{cgreen},
	stringstyle=\color{blue},
	frame=lines,
	% escapeinside=``,
	extendedchars=true, 
	xleftmargin=1em,
	xrightmargin=1em, 
	backgroundcolor=\color{cred},
	aboveskip=1em,
	breaklines=true,
	tabsize=4
} 

\newfontfamily{\consolas}{Consolas}
\newfontfamily{\monaco}{Monaco}
\setmonofont[Mapping={}]{Consolas}	%英文引号之类的正常显示,相当于设置英文字体
\setsansfont{Consolas} %设置英文字体 Monaco, Consolas,  Fantasque Sans Mono
\setmainfont{Times New Roman}

\setCJKmainfont{华文中宋}


\newcommand{\fic}[1]{\begin{figure}[H]
		\center
		\includegraphics[width=0.8\textwidth]{#1}
	\end{figure}}
	
\newcommand{\sizedfic}[2]{\begin{figure}[H]
		\center
		\includegraphics[width=#1\textwidth]{#2}
	\end{figure}}

\newcommand{\codefile}[1]{\lstinputlisting{#1}}

% 改变段间隔
\setlength{\parskip}{0.2em}
\linespread{1.1}

\usepackage{lastpage}
\usepackage{fancyhdr}
\pagestyle{fancy}
\lhead{\space \qquad \space}
\chead{Hello\_World模块 \qquad}
\rhead{\qquad\thepage/\pageref{LastPage}}
\begin{document}

% \tableofcontents

% \clearpage

\section{编写Hello\_World模块}
	首先编写hello.c,代码如下:
	\begin{lstlisting}
	#include <linux/init.h> 
	#include <linux/module.h> 
	MODULE_LICENSE("Dual BSD/GPL"); 

	static int hello_init(void) 
	{ 
		printk(KERN_ALERT "Hello, world\n"); 
		return 0; 
	} 

	static void hello_exit(void) 
	{ 
		printk(KERN_ALERT "Goodbye, cruel world\n"); 
	} 

	module_init(hello_init); 
	module_exit(hello_exit); 
	\end{lstlisting}

	写一个简单的Makefile文件:
	\begin{lstlisting}
	obj-m := hello.o
	KERNELDIR := /lib/modules/$(shell uname -r)/build
	PWD := $(shell pwd)

	modules:
		$(MAKE) -C $(KERNELDIR) M=$(PWD) modules
	\end{lstlisting}

\section{Hello\_World模块的代码解析}
\subsection{解析hello.c}
	hello.c中的代码很简单,这里主要说一下用到的一些API函数。\par
	module\_init()是一个宏,用于指定模块初始化的函数,所以hello\_init()函数在模块装载到内核时会被调用。
	module\_exit()是一个宏,用于指定模块的清除函数,所以hello\_exit()函数从内核卸载时会被调用。\par

	MODULE\_LICENSE()是一个宏,用来告诉内核该模块使用的许可证。如果没有使用MODULE\_LICENSE("Dual BSD/GPL"),内核在装载该模块时会产生抱怨。\par

	printk()函数和printf()函数功能类似,但printk()函数是内核函数。当模块连接到内核时,就可以调用printk()函数。


\subsection{解析Makefile}
	首先说明,这里不会介绍Makefile语法,而是介绍为什么Makefile的内容是这些。\par

	“obj-m := hello.o”,用于说明一个模块需要从目标文件hello.o中构造,而从该目标文件构造的模块名称为hello.ko。\par

	“KERNELDIR := /lib/modules/\$(shell uname -r)/build”是给变量赋值, KERNELDIR的值为当前内核源代码目录。
	“PWD := \$(shell pwd)”中PWD的值为当前目录。\par

	“\$(MAKE) -C \$(KERNELDIR) M=\$(PWD) modules”这个命令是真正用来构造模块的。
	命令首先改变目录到-C选项指定的位置,在这个目录中有内核自己的Makefile文件,这里的Makefile文件用于真正构造模块。
	等到该Makefile要真正地构造modules目标时,返回到-M选项指定的目录,modules目标指向的是obj-m变量中设定的模块。\par

	也就是说,当我们执行“make modules”的时候,将执行“\$(MAKE) -C \$(KERNELDIR) M=\$(PWD) modules”。
	make命令将首先进入-C选项指定的内核源码树,运行内核构造系统,其中Makefile要真正构造模块之前,make命令又回到-M选项指定的目录,
	在该目录下生成obj-m指定的模块。

\section{使用Hello\_World模块}
	\begin{lstlisting}
# 编译模块
pengsida@scholes:~/ldd$ make modules 
make -C /lib/modules/4.8.0-22-generic/build M=/home/pengsida/ldd modules
make[1]: Entering directory '/usr/src/linux-headers-4.8.0-22-generic'
Building modules, stage 2.
MODPOST 1 modules
make[1]: Leaving directory '/usr/src/linux-headers-4.8.0-22-generic'

# 安装hello模块
pengsida@scholes:~/ldd$ sudo insmod ./hello.ko

# 查看hello模块是否在代码中
pengsida@scholes:~/ldd$ lsmod | grep hello
hello                  16384  0

# 卸载hello模块
pengsida@scholes:~/ldd$ sudo rmmod hello

# 查看hello模块运行时输出的信息
pengsida@scholes:~/ldd$ cat /var/log/syslog | grep world
Apr 23 21:32:49 scholes kernel: [ 1779.811190] hello world
Apr 23 21:33:40 scholes kernel: [ 1830.907433] Goodbye, cruel world
	\end{lstlisting}

\end{document}
