% !TeX spellcheck = en_US
%% 字体:方正静蕾简体
%%		 方正粗宋
\documentclass[a4paper,left=1.5cm,right=1.5cm,11pt]{article}

\usepackage[utf8]{inputenc}
\usepackage{fontspec}
\usepackage{cite}
\usepackage{xeCJK}
\usepackage{indentfirst}
\usepackage{titlesec}
\usepackage{etoolbox}%
\makeatletter
\patchcmd{\ttlh@hang}{\parindent\z@}{\parindent\z@\leavevmode}{}{}%
\patchcmd{\ttlh@hang}{\noindent}{}{}{}%
\makeatother

\usepackage{hyperref}
\usepackage{longtable}
\usepackage{empheq}
\usepackage{graphicx}
\usepackage{float}
\usepackage{rotating}
\usepackage{subfigure}
\usepackage{tabu}
\usepackage{amsmath}
\usepackage{setspace}
\usepackage{amsfonts}
\usepackage{appendix}
\usepackage{listings}
\usepackage{xcolor}
\usepackage{geometry}
\setcounter{secnumdepth}{4}
%\titleformat*{\section}{\LARGE}
%\renewcommand\refname{参考文献}
%\titleformat{\chapter}{\centering\bfseries\huge}{}{0.7em}{}{}
\titleformat{\section}{\LARGE\bf}{\thesection}{1em}{}{}
\titleformat{\subsection}{\Large\bfseries}{\thesubsection}{1em}{}{}
\titleformat{\subsubsection}{\large\bfseries}{\thesubsubsection}{1em}{}{}
\renewcommand{\contentsname}{{ \centerline{目{  } 录}}}
\setCJKfamilyfont{cjkhwxk}{STXINGKA.TTF}
%\setCJKfamilyfont{cjkhwxk}{华文行楷}
%\setCJKfamilyfont{cjkfzcs}{方正粗宋简体}
%\newcommand*{\cjkfzcs}{\CJKfamily{cjkfzcs}}
\newcommand*{\cjkhwxk}{\CJKfamily{cjkhwxk}}
%\newfontfamily\wryh{Microsoft YaHei}
%\newfontfamily\hwzs{华文中宋}
%\newfontfamily\hwst{华文宋体}
%\newfontfamily\hwfs{华文仿宋}
%\newfontfamily\jljt{方正静蕾简体}
%\newfontfamily\hwxk{华文行楷}
\newcommand{\verylarge}{\fontsize{60pt}{\baselineskip}\selectfont}  
\newcommand{\chuhao}{\fontsize{44.9pt}{\baselineskip}\selectfont}  
\newcommand{\xiaochu}{\fontsize{38.5pt}{\baselineskip}\selectfont}  
\newcommand{\yihao}{\fontsize{27.8pt}{\baselineskip}\selectfont}  
\newcommand{\xiaoyi}{\fontsize{25.7pt}{\baselineskip}\selectfont}  
\newcommand{\erhao}{\fontsize{23.5pt}{\baselineskip}\selectfont}  
\newcommand{\xiaoerhao}{\fontsize{19.3pt}{\baselineskip}\selectfont} 
\newcommand{\sihao}{\fontsize{14pt}{\baselineskip}\selectfont}      % 字号设置  
\newcommand{\xiaosihao}{\fontsize{12pt}{\baselineskip}\selectfont}  % 字号设置  
\newcommand{\wuhao}{\fontsize{10.5pt}{\baselineskip}\selectfont}    % 字号设置  
\newcommand{\xiaowuhao}{\fontsize{9pt}{\baselineskip}\selectfont}   % 字号设置  
\newcommand{\liuhao}{\fontsize{7.875pt}{\baselineskip}\selectfont}  % 字号设置  
\newcommand{\qihao}{\fontsize{5.25pt}{\baselineskip}\selectfont}    % 字号设置 

\usepackage{diagbox}
\usepackage{multirow}
\boldmath
\XeTeXlinebreaklocale "zh"
\XeTeXlinebreakskip = 0pt plus 1pt minus 0.1pt
\definecolor{cred}{rgb}{0.8,0.8,0.8}
\definecolor{cgreen}{rgb}{0,0.3,0}
\definecolor{cpurple}{rgb}{0.5,0,0.35}
\definecolor{cdocblue}{rgb}{0,0,0.3}
\definecolor{cdark}{rgb}{0.95,1.0,1.0}
\lstset{
	language=bash,
	numbers=left,
	numberstyle=\tiny\color{black},
	showspaces=false,
	showstringspaces=false,
	basicstyle=\scriptsize,
	keywordstyle=\color{purple},
	commentstyle=\itshape\color{cgreen},
	stringstyle=\color{blue},
	frame=lines,
	% escapeinside=``,
	extendedchars=true, 
	xleftmargin=1em,
	xrightmargin=1em, 
	backgroundcolor=\color{cred},
	aboveskip=1em,
	breaklines=true,
	tabsize=4
} 

%\newfontfamily{\consolas}{Consolas}
%\newfontfamily{\monaco}{Monaco}
%\setmonofont[Mapping={}]{Consolas}	%英文引号之类的正常显示,相当于设置英文字体
%\setsansfont{Consolas} %设置英文字体 Monaco, Consolas,  Fantasque Sans Mono
%\setmainfont{Times New Roman}
%\setCJKmainfont{STZHONGS.TTF}
%\setmonofont{Consolas}
% \newfontfamily{\consolas}{YaHeiConsolas.ttf}
\newfontfamily{\monaco}{MONACO.TTF}
\setCJKmainfont{STZHONGS.TTF}
%\setmainfont{MONACO.TTF}
%\setsansfont{MONACO.TTF}

\newcommand{\fic}[1]{\begin{figure}[H]
		\center
		\includegraphics[width=0.8\textwidth]{#1}
	\end{figure}}
	
\newcommand{\sizedfic}[2]{\begin{figure}[H]
		\center
		\includegraphics[width=#1\textwidth]{#2}
	\end{figure}}

\newcommand{\codefile}[1]{\lstinputlisting{#1}}

\newcommand{\interval}{\vspace{0.5em}}

\newcommand{\tablestart}{
	\interval
	\begin{longtable}{p{2cm}p{10cm}}
	\hline}
\newcommand{\tableend}{
	\hline
	\end{longtable}
	\interval}

% 改变段间隔
\setlength{\parskip}{0.2em}
\linespread{1.1}

\usepackage{lastpage}
\usepackage{fancyhdr}
\pagestyle{fancy}
\lhead{\space \qquad \space}
\chead{测试动态快照稳定性 \qquad}
\rhead{\qquad\thepage/\pageref{LastPage}}
\begin{document}

% \tableofcontents

\clearpage

\section{测试动态快照稳定性}
\subsection{测试的思路}
	首先我介绍一下当前环境:libvirt中定义了两个虚拟机overlay1和overlay2。
	overlay1的磁盘文件是overlay1.qcow2,overlay2的磁盘文件也是overlay1.qcow2。
	overlay1.qcow2的后端镜像是base.qcow2。\par

	overlay1.qcow2上存放这test\_in\_vm.sh、monitor.sh和judge.sh这三个脚本文件。
	这个三脚本文件的功能分别是:
	\begin{lstlisting}
	test_in_vm.sh:将1~4000000这些数字写入num.txt文件中,每个数字一行。
	monitor.sh:这个文件用于监测test_in_vm.sh这个进程是否运行结束。
	judge.sh:这个文件用于判断num.txt是否有4000000,用于判断数据的完整性。
	\end{lstlisting}

	接下来我说一下测试方案:
	\begin{itemize}
		\item[1.] 首先启动虚拟机overlay1,注意它的磁盘文件为overlay1.qcow2。
		\item[2.] 在宿主机环境中,ssh登录到overlay1上,执行上面的test\_in\_vm.sh脚本。
		\item[3.] 动态创建快照,此时虚拟机overlay1的磁盘文件就变成了overlay2.qcow2。
		\item[4.] 查看overlay1上的数据是否完整,也就是查看overlay2.qcow2磁盘上的数据是否完整。
		\item[5.] 启动虚拟机overlay2,注意它的磁盘文件为overlay2.qcow2。
				  随后查看它上面的数据是否不完整,也就是查看overlay1.qcow2磁盘上的数据是否不完整。
		\item[6.] 关闭虚拟机overlay1和overlay2。
		\item[7.] 清除之前的改动,返回到最初的环境,也就是要把overlay2.qcow2这个增量文件删除,
		          将虚拟机overlay1的磁盘文件更换为原来的overlay1.qcow2,将虚拟机overlay1的磁盘文件更换为overlay1.qcow2,并且删除虚拟机overlay1的快照数据。
	\end{itemize}

\subsection{测试需要的环境}
	首先创建一个base.qcow2镜像:
	\begin{lstlisting}
	qemu-img create -f qcow2 base.qcow2 50G
	\end{lstlisting}

	随后在base.qcow2镜像上安装ubuntu系统,并且在这个镜像的基础上创建增量镜像overlay1.qcow2:
	\begin{lstlisting}
	qemu-img create -f qcow2 -b base.qcow2 overlay1.qcow2
	\end{lstlisting}

	在overlay1.qcow2上启动虚拟机overlay1,overlay1的xml配置文件内容如下:
	\begin{lstlisting}[language = xml]
	<domain type='kvm'>
		<name>overlay1</name>
		<memory>1048576</memory>
		<currentMemory>1048576</currentMemory>
		<vcpu>4</vcpu>
		<os>
			<type arch='x86_64' machine='pc'>hvm</type>
			<boot dev='cdrom'/>
		</os>
		<features>
			<acpi/>
			<apic/>
			<pae/>
		</features>
		<clock offset='localtime'/>
		<on_poweroff>destroy</on_poweroff>
		<on_reboot>restart</on_reboot>
		<on_crash>destroy</on_crash>
		<devices>
			<emulator>/usr/bin/qemu-system-x86_64</emulator>
			<disk type='file' device='disk'>
				<driver name='qemu' type='qcow2'/>
				<source file='/home/pengsida/kvm/openstack/overlay1.qcow2'/>
				<target dev='hda' bus='ide'/>
			</disk>
			<interface type='network'>
				<source network='default'/>
			</interface>
			<interface type='network'>
				<source network='default'/>
			</interface>
			<input type='mouse' bus='ps2'/>
			<graphics type='vnc' port='-1' autoport='yes' listen='0.0.0.0' keymap='en-us'/>
			<channel type='unix'>
				<source mode='bind' path='/var/lib/libvirt/qemu/f16x86_64.agent'/>
				<target type='virtio' name='org.qemu.guest_agent.0'/>
			</channel>
		</devices>
	</domain>
	\end{lstlisting}

	在宿主机上安装qemu-guest-agent:
	\begin{lstlisting}
	sudo apt install qemu-guest-agent
	\end{lstlisting}

	随后启动虚拟机overlay1:
	\begin{lstlisting}
	# 在overlay1.xml目录下
	sudo virsh define overlay1.xml
	sudo virsh start overlay1
	\end{lstlisting}

	在虚拟机中安装qemu-guest-agent:
	\begin{lstlisting}
	sudo apt install qemu-guest-agent
	\end{lstlisting}

	然后在overlay1上创建两个test\_in\_vm.sh、monitor.sh和judge.sh这三个脚本文件。\par

	test\_in\_vm.sh脚本内容如下:
	\begin{lstlisting}
	#!/bin/bash

	times=2000

	rm num.txt

	num=1

	for((i=0;i<=times;i++))
	do
		for((j=0;j<=times;j++))
		do
			echo $num >> num.txt
			num=$[$num+1]
		done
	done
	\end{lstlisting}

	monitor.sh脚本内容如下:
	\begin{lstlisting}
	#!/bin/bash

	test=`ps aux | grep ./test.sh`
	num=`echo $test | awk -v RS='/bin/bash ./test.sh' 'END {print --NR}'`

	while (( $num >= 1 )) 
	do
		echo "./test.sh is running"
		sleep 5
		test=`ps aux | grep ./test.sh`
		num=`echo $test | awk -v RS='/bin/bash ./test.sh' 'END {print --NR}'`
	done

	echo "./test.sh is done"
	\end{lstlisting}

	judge.sh脚本内容如下:
	\begin{lstlisting}
	#!/bin/bash

	num=`awk 'END {print NR}' num.txt`

	if (( num == 4000000)) 
	then
		echo "data is complete"
	else
		echo "data is not complete"
	fi
	\end{lstlisting}

	随后在宿主机上创建address.sh脚本文件,这个脚本文件可以根据虚拟机的名字返回虚拟机的ip地址:
	\begin{lstlisting}
	#!/bin/bash

	MAC=`awk '/virbr0/ { print $4 }' /proc/net/arp`
	VM=$1

	line=`virsh dumpxml $VM | grep "mac address"`
	address=`echo $line | awk -F "" '
	{
		for(i=1; i<=NF; i++)
		{
			if($i ~ /'\''/)
				break
		}
		i++
		str=""
		for(;i<=NF;i++)
		{
			if($i ~ /'\''/)
				break
			temp=$i
			str=(str temp)
		}
		print str
	}'`

	ip=`awk '/'$address'/ {print $1}' /proc/net/arp`

	echo $ip
	\end{lstlisting}

	首先ssh登录到虚拟机overlay1:
	\begin{lstlisting}
	# 使用address.sh脚本获得overlay1虚拟机的ip地址
	overlay1=`./address.sh overlay1`
	echo $overlay1 # 如果输出为空,需要重复上面的命令
	ssh pengsida@$overlay1 # 假设虚拟机的用户名是pengsida
	\end{lstlisting}

	随后关闭虚拟机overlay1:
	\begin{lstlisting}
	sudo virsh destroy overlay1
	\end{lstlisting}

	随后开启虚拟机overlay2:
	\begin{lstlisting}
	sudo virsh define overlay2.xml
	sudo virsh start overlay2
	\end{lstlisting}

	同样是ssh登录到虚拟机overlay2:
	\begin{lstlisting}
	overlay2=`./address.sh overlay2`
	echo $overlay2 # 如果输出为空,需要重复上面的命令,直到获得overlay2的ip
	ssh pengsida@$overlay2
	\end{lstlisting}

	随后关闭虚拟机overlay2:
	\begin{lstlisting}
	sudo virsh destroy overlay2
	\end{lstlisting}

	这里之所以要ssh登录到虚拟机overlay1和overlay2的原因是需要信任未连接过的ip地址,如下图所示:
	\fic{1.png}

\subsection{测试的代码}
	在宿主机上创建test.sh脚本,内容如下:
	\begin{lstlisting}
	#!/bin/bash

	# 步骤一
	# 创建虚拟机overlay1,磁盘文件为overlay1.qcow2
	sudo virsh start overlay1
	# 等待虚拟机启动
	sleep 30


	# 步骤二
	# 获取虚拟机ip地址
	overlay1=`./address.sh overlay1`
	# ssh登录到overlay1上,执行上面的test.sh脚本
	sshpass -p p1111111 ssh pengsida@$overlay1 ./test_in_vm.sh &

	# 先让./test.sh脚本运行5秒
	sleep 5


	# 步骤三
	# 动态创建快照,此时虚拟机overlay1的磁盘文件变为overlay2.qcow2
	sudo virsh snapshot-create-as overlay1 snap snap-desc --disk-only --diskspec hda,snapshot=external,file=/home/pengsida/kvm/openstack/overlay2.qcow2 --atomic

	sleep 5


	# 步骤四
	# 等待test.sh脚本执行结束
	sshpass -p p1111111 ssh pengsida@$overlay1 ./monitor.sh
	# 判断overlay1.qcow2上的数据是否完整
	sshpass -p p1111111 ssh pengsida@$overlay1 ./judge.sh


	# 步骤五
	# 创建虚拟机overlay2,磁盘文件为overlay1.qcow2
	sudo virsh start overlay2
	# 等待虚拟机启动
	sleep 60
	# 获取虚拟机ip地址
	overlay2=`./address.sh overlay2`
	# 判断overlay1.qcow2上的数据是否完整
	sshpass -p p1111111 ssh pengsida@$overlay2 ./judge.sh


	# 步骤六
	sudo virsh destroy overlay1
	sudo virsh destroy overlay2


	# 步骤七
	sudo virsh snapshot-delete overlay1 snap --metadata
	sudo rm /home/pengsida/kvm/openstack/overlay2.qcow2

	# 将虚拟机overlay1的磁盘文件更换为overlay1.qcow2
	sudo virsh detach-disk overlay1 hda --persistent
	sudo virsh attach-disk overlay1 /home/pengsida/kvm/openstack/overlay1.qcow2 hda --subdriver qcow2 --persistent
	\end{lstlisting}

\end{document}
