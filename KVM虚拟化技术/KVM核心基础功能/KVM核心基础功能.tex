% !TeX spellcheck = en_US
%% 字体:方正静蕾简体
%%		 方正粗宋
\documentclass[a4paper,left=2.5cm,right=2.5cm,11pt]{article}

\usepackage[utf8]{inputenc}
\usepackage{fontspec}
\usepackage{cite}
\usepackage{xeCJK}
\usepackage{indentfirst}
\usepackage{titlesec}
\usepackage{longtable}
\usepackage{graphicx}
\usepackage{float}
\usepackage{rotating}
\usepackage{subfigure}
\usepackage{tabu}
\usepackage{amsmath}
\usepackage{setspace}
\usepackage{amsfonts}
\usepackage{appendix}
\usepackage{listings}
\usepackage{xcolor}
\usepackage{geometry}
\setcounter{secnumdepth}{4}
\usepackage{mhchem}
\usepackage{multirow}
\usepackage{extarrows}
\usepackage{hyperref}
\titleformat*{\section}{\LARGE}
\renewcommand\refname{参考文献}
\renewcommand{\abstractname}{\sihao \cjkfzcs 摘{  }要}
%\titleformat{\chapter}{\centering\bfseries\huge\wryh}{}{0.7em}{}{}
%\titleformat{\section}{\LARGE\bf}{\thesection}{1em}{}{}
\titleformat{\subsection}{\Large\bfseries}{\thesubsection}{1em}{}{}
\titleformat{\subsubsection}{\large\bfseries}{\thesubsubsection}{1em}{}{}
\renewcommand{\contentsname}{{\cjkfzcs \centerline{目{  } 录}}}
\setCJKfamilyfont{cjkhwxk}{STXingkai}
\setCJKfamilyfont{cjkfzcs}{STSongti-SC-Regular}
% \setCJKfamilyfont{cjkhwxk}{华文行楷}
% \setCJKfamilyfont{cjkfzcs}{方正粗宋简体}
\newcommand*{\cjkfzcs}{\CJKfamily{cjkfzcs}}
\newcommand*{\cjkhwxk}{\CJKfamily{cjkhwxk}}
\newfontfamily\wryh{Microsoft YaHei}
\newfontfamily\hwzs{STZhongsong}
\newfontfamily\hwst{STSong}
\newfontfamily\hwfs{STFangsong}
\newfontfamily\jljt{MicrosoftYaHei}
\newfontfamily\hwxk{STXingkai}
% \newfontfamily\hwzs{华文中宋}
% \newfontfamily\hwst{华文宋体}
% \newfontfamily\hwfs{华文仿宋}
% \newfontfamily\jljt{方正静蕾简体}
% \newfontfamily\hwxk{华文行楷}
\newcommand{\verylarge}{\fontsize{60pt}{\baselineskip}\selectfont}  
\newcommand{\chuhao}{\fontsize{44.9pt}{\baselineskip}\selectfont}  
\newcommand{\xiaochu}{\fontsize{38.5pt}{\baselineskip}\selectfont}  
\newcommand{\yihao}{\fontsize{27.8pt}{\baselineskip}\selectfont}  
\newcommand{\xiaoyi}{\fontsize{25.7pt}{\baselineskip}\selectfont}  
\newcommand{\erhao}{\fontsize{23.5pt}{\baselineskip}\selectfont}  
\newcommand{\xiaoerhao}{\fontsize{19.3pt}{\baselineskip}\selectfont} 
\newcommand{\sihao}{\fontsize{14pt}{\baselineskip}\selectfont}      % 字号设置  
\newcommand{\xiaosihao}{\fontsize{12pt}{\baselineskip}\selectfont}  % 字号设置  
\newcommand{\wuhao}{\fontsize{10.5pt}{\baselineskip}\selectfont}    % 字号设置  
\newcommand{\xiaowuhao}{\fontsize{9pt}{\baselineskip}\selectfont}   % 字号设置  
\newcommand{\liuhao}{\fontsize{7.875pt}{\baselineskip}\selectfont}  % 字号设置  
\newcommand{\qihao}{\fontsize{5.25pt}{\baselineskip}\selectfont}    % 字号设置 

\usepackage{diagbox}
\usepackage{multirow}
\boldmath
\XeTeXlinebreaklocale "zh"
\XeTeXlinebreakskip = 0pt plus 1pt minus 0.1pt
\definecolor{cred}{rgb}{0.8,0.8,0.8}
\definecolor{cgreen}{rgb}{0,0.3,0}
\definecolor{cpurple}{rgb}{0.5,0,0.35}
\definecolor{cdocblue}{rgb}{0,0,0.3}
\definecolor{cdark}{rgb}{0.95,1.0,1.0}
\lstset{
	language=C,
	numbers=left,
	numberstyle=\tiny\color{white},
	showspaces=false,
	showstringspaces=false,
	basicstyle=\scriptsize,
	keywordstyle=\color{purple},
	commentstyle=\itshape\color{cgreen},
	stringstyle=\color{blue},
	frame=lines,
	% escapeinside=``,
	extendedchars=true, 
	xleftmargin=0em,
	xrightmargin=0em, 
	backgroundcolor=\color{cred},
	aboveskip=1em,
	breaklines=true,
	tabsize=4
} 

\newfontfamily{\consolas}{Consolas}
\newfontfamily{\monaco}{Monaco}
\setmonofont[Mapping={}]{Consolas}	%英文引号之类的正常显示,相当于设置英文字体
\setsansfont{Consolas} %设置英文字体 Monaco, Consolas,  Fantasque Sans Mono
\setmainfont{Times New Roman}

\setCJKmainfont{华文中宋}


\newcommand{\fic}[1]{\begin{figure}[H]
		\center
		\includegraphics[width=0.8\textwidth]{#1}
	\end{figure}}
	
\newcommand{\sizedfic}[2]{\begin{figure}[H]
		\center
		\includegraphics[width=#1\textwidth]{#2}
	\end{figure}}

\newcommand{\codefile}[1]{\lstinputlisting{#1}}

\newcommand{\interval}{\vspace{0.5em}}

\newcommand{\tablestart}{
	\interval
	\begin{longtable}{p{2cm}p{10cm}}
	\hline}
\newcommand{\tableend}{
	\hline
	\end{longtable}
	\interval}

% 改变段间隔
\setlength{\parskip}{0.2em}
\linespread{1.1}

\usepackage{lastpage}
\usepackage{fancyhdr}
\pagestyle{fancy}
\lhead{\space \qquad \space}
\chead{kvm核心基础功能 \qquad}
\rhead{\qquad\thepage/\pageref{LastPage}}
\begin{document}

\tableofcontents

\clearpage

\section{CPU配置}
\subsection{-smp参数项}
	qemu-system-x86\_64命令行中,"-smp"参数可以用来配置客户机的SMP系统,具体参数如下:
	\begin{lstlisting}[numberstyle = \color{white}]
	qemu-system-x86_64 -smp n[,maxcpus=cpus][,cores=cores][,threads=threads][,sockets=sockets]
	\end{lstlisting}

	各个选项介绍如下:
	\interval
	\begin{longtable}{p{1.5cm}p{10cm}}
	\hline
	n & 用于设置客户机中使用的逻辑PCU数量 \\
	\hline
	maxcpus & 用于设置客户机中最大可能被使用的CPU数量 \\
	\hline
	cores & 用于设置每个CPU socket上的core数量 \\
	\hline
	threads & 用于设置每个CPU core上的线程数 \\
	\hline
	sockets & 用于设置客户机中看到的总的CPU socket数量 \\
	\hline
	\end{longtable}
	\interval

	例子如下:
	\begin{lstlisting}
	qemu-system-x86_64 -smp 4,maxcpus=8,sockets=2,cores=2,threads=2 ubuntu1604 -vnc 127.0.0.1:2
	\end{lstlisting}

\subsection{查看cpu配置}
\subsubsection{在客户机中查看cpu信息}
	使用如下命令可以输出cpu当前的信息:
	\begin{lstlisting}
	cat /proc/cpuinfo
	\end{lstlisting}

	这里介绍一下cat命令:
	\interval
	\begin{longtable}{lp{12cm}}
	\hline
	三大功能 & 1.一次显示整个文件: cat filename \\
		   & 2.从键盘创建一个文件: cat > filename \\
		   & 3.将几个文件合并为一个文件: cat file1 file2 > file \\
	\hline
	参数  & -n或--number: 由1开始对所有输出的行数编号 \\
		 & -b或--number-nonblank: 和-n相似,只不过对于空白行不编号 \\
		 & -s或--squeeze-blank: 当遇到有连续两行以上的空白行,就代换为一行的空白行 \\
	\hline  
	\end{longtable}
	\interval

\subsubsection{使用qemu监控客户机cpu信息}
	使用qemu-system-x86\_64命令时,加上"-monitor stdio",即可使用monitor command监控客户机使用情况,比如在联网情况下输入如下命令:
	\begin{lstlisting}
	qemu-system-x86_64 ubuntu1604.img -vnc 127.0.0.1:2 -monitor stdio
	\end{lstlisting}

	此时,就开始monitor command来监控客户机。可以在qemu monitor中使用如下命令查询cpu状态:
	\begin{lstlisting}
	info cpus
	\end{lstlisting}

\subsection{-cpu参数项}
	qemu-system-x86\_64命令行中,"-cpu"参数可以用来查看qemu所支持cpu模型,或者指定客户机的CPU模型。
	具体使用如下:
	\begin{lstlisting}
	// 查看qemu所支持的cpu模型
	qemu-system-x86_64 -cpu ?
	// 指定客户机中的cpu模型
	qemu-system-x86_64 -cpu cpu_model 
	\end{lstlisting}

	qemu支持的cpu模型如下所示:
	\fic{1.png}

	如果不加"cpu"参数启动客户机时,采用"qemu64"作为默认的cpu模型。

\subsection{vCPU的绑定}
	vCPU就是客户机的虚拟cpu,vCPU相当于宿主机中一个普通的qemu线程。可以使用taskset工具将vCPU线程绑定到特定的cpu上执行。\par
	在实际应用中,如果想要为客户提供客户机使用,并且要求不受宿主机中其他客户机的影响,就需要将vCPU绑定到特定的cpu上。步骤如下:
	\begin{itemize}
		\item[1.] 启动宿主机时隔离出特定的CPU专门供一个客户机使用。
		\item[2.] 启动客户机,将其vCPU绑定到宿主机特定的CPU上。
	\end{itemize}

\subsubsection{隔离宿主机CPU}
	在grub文件中Linux内核启动的命令行加上"isolcpus"参数,就可以实现CPU的隔离。这里介绍一下"isolcpus"参数项:
	\tablestart
	功能 & 将相应的CPU从调度算法中隔离出来\\
	\hline
	参数选项 & isolcpus= cpu\_number[,cpu\_number,...] \\
	\tableend

	向grub文件中添加"isolcpus"参数的命令如下所示:
	\begin{lstlisting}
	sudo vi /boot/grub/grub.cfg
	/menuentry
	\end{lstlisting}
	
	然后在插入模式下,在initrd参数前一行写入:
	\begin{lstlisting}
	isolcpus = cpu\_number1[,cpu\_number2,...]
	\end{lstlisting}

	如下图所示:
	\fic{2.png}

	重启电脑以后就将相应的cpu隔离出调度算法了。\par

	使用如下命令可以查看cpu上执行的进程和线程总数,用于检查CPU是否成功被隔离。

	\begin{lstlisting}
	ps -eLo psr | grep cpu\_number | wc -l
	\end{lstlisting}

	下面分别介绍命令中的ps和wc:
	\tablestart
	ps & 用于显示当前系统的进程信息的状态\\
	\hline
	参数项 & -e:用于显示所有进程 \\
		& -L:用于显示所有线程 \\
		& -o:用于以特定的格式输出信息,psr指定输出分配给进程运行的处理器编号 \\
	\tableend

	\tablestart
	wc & 该命令统计给定文件中的字节数、字数、行数 \\
	\hline
	参数项 & -c:统计字节数 \\
		  & -l:统计行数 \\
		  & -w:统计字数 \\
	\tableend

	假如成功隔离了cpu2,就会看到在cpu上执行的进程和线程数非常少。
	
\subsubsection{绑定客户机vCPU}
	使用taskset命令就可以将vCPU绑定到特定的CPU上。taskset命令的使用如下所示:
	\begin{lstlisting}
	taskset -p mask pid
	\end{lstlisting}

	这里介绍一下taskset命令:
	\tablestart
	taskset & 将进程绑定到特定的CPU上 \\
	\hline
	参数项 & -p:将已经创建的进程绑定到CPU上 \\
		 & mask:用于指定CPU的掩码,mask第几位为1就代表第几号CPU \\
		 & pid:进程号,用于指定进程 \\
	\tableend

	比如,如果想把进程号为3963的进程绑定到cpu2和cpu3上,就使用如下命令:
	\begin{lstlisting}
	taskset -p 0x6 3963
	\end{lstlisting}

	0x6二进制位1100,代表cpu2和cpu3。而3963指定了进程号为3963的进程。\par
	如此一来,如果想把vCPU绑定到宿主机的cpu上,只要知道vCPU的进程号就行了。可以在qemu monitor中使用如下命令查询vCPU的进程号:
	\begin{lstlisting}
	info cpus
	\end{lstlisting}

	如下所示:
	\fic{3.png}

	可以看到,图中vCPU的进程号分别是5118、5119、5120和5121。

\clearpage

\section{内存配置}
\subsection{-m参数项}
\begin{longtable}{p{2cm}p{8cm}}
\hline
-m megs & 设置客户机的内存位megsMB大小 \\
	    & 默认单位为MB,加上"M"或"G"可以指定单位 \\
		& 不设置-m参数,客户机内存默认为128MB \\
\hline
\end{longtable}
\subsection{查看内存信息}
	linux下有两个命令可以用于查看内存信息。\par
	第一个是free -m,如下图所示:
	\fic{5.png}

	第二个是dmesg。不过因为dmesg存放着内核开机信息,信息量比较多,需要用grep命令来筛选,如下图所示:
	\fic{4.png}

\subsection{EPT扩展页表}
	EPT扩展页表是Intel的第二代硬件虚拟化技术,是针对内存管理单元的虚拟化扩展。\par
	在Linux系统中,可以通过如下命令确定系统是否支持EPT功能:
	\begin{lstlisting}
	grep ept /proc/cpuinfo
	\end{lstlisting}

	可以通过如下命令确定KVM是否打开了EPT功能:
	\begin{lstlisting}
	cat /sys/module/kvm_intel/parameters/ept
	\end{lstlisting}

	在加载kvm\_intel模块时,可以通过设置ept的值来打开EPT。
	\begin{lstlisting}
	modprobe kvm_intel ept=0 // ept代表关闭EPT功能
	\end{lstlisting}

	如果kvm\_intel模块已经处于加载状态,则需要先写在这个模块,在重新加载时加入所需的参数设置。如下所示:
	\begin{lstlisting}
	rmmod kvm_intel
	modprobe kvm_intel ept=1
	\end{lstlisting}

\subsection{-mem-path参数项}
	qemu-kvm提供了"-mem-path"参数项用于将huge page的特性应用到客户机上。\par
	huge page是大小超过4KB的内存页面,它可以让地址转换信息减少,节约页表所占用的内存数量,在整体上提升系统的性能。\par
	可以使用如下命令查看系统中huge page的信息,如下所示:
	\begin{lstlisting}
	cat /proc/meminfo | grep HugePages
	\end{lstlisting}

	可以通过以下几步让客户机使用huge page:
	\begin{itemize}
		\item[(1)] 在宿主机中挂载hugetlbfs文件系统,命令如下所示:
		\begin{lstlisting}
	sudo mount -t hugetlbfs hugetlbfs /dev/hugepages
		\end{lstlisting}

		这里介绍一下mount命令:
		\interval
		\begin{longtable}{p{2cm}p{8cm}}
		\hline
		标准格式 & mount -t type device dir \\
		\hline
		功能 & 让内核将在device上的文件系统挂载到目录dir下,文件系统类型是type \\
		\hline
		参数项 & -t:指定文件系统的类型 \\
		\hline
		\end{longtable}
		\interval
		
		所以之前命令就是将hugetlbfs类型的文件系统挂载到/dev/hugepages上。

		\item[(2)] 设置hugepage的数量,命令如下所示:
		\begin{lstlisting}
	sudo sysctl vm.nr_hugepages=num
		\end{lstlisting}

		\item[(3)] 启动客户机时使用"-mem-path"参数让客户机使用hugepage的内存,如下所示:
		\begin{lstlisting}
	qemu-system-x86_64 ubuntu1604.img -mem-path /dev/hugepages
		\end{lstlisting}
	\end{itemize}

	上述过程的实际操作如下图所示:
	\fic{6.png}

\clearpage

\section{存储配置}
\subsection{与存储相关的参数项}
\subsection{-driver参数项}
\subsection{-boot参数项}
\subsection{qemu-img命令}
\subsection{qemu支持的镜像文件格式}
\subsection{客户机存储方式}

\clearpage

\section{网络配置}
\subsection{使用网桥模式}
\subsection{使用NAT模式}
\subsection{使用用户模式}
\subsection{其他网络选项}

\clearpage

\section{显示配置}
\subsection{使用SDL}
\subsection{使用VNC}
\subsection{使用非图形模式}
\subsection{其他显示选项}



\end{document}
