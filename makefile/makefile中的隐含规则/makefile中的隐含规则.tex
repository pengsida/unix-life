% !TeX spellcheck = en_US
%% 字体:方正静蕾简体
%%		 方正粗宋
\documentclass[a4paper,left=2.5cm,right=2.5cm,11pt]{article}

\usepackage[utf8]{inputenc}
\usepackage{fontspec}
\usepackage{cite}
\usepackage{xeCJK}
\usepackage{indentfirst}
\usepackage{titlesec}
\usepackage{longtable}
\usepackage{graphicx}
\usepackage{float}
\usepackage{rotating}
\usepackage{subfigure}
\usepackage{tabu}
\usepackage{amsmath}
\usepackage{setspace}
\usepackage{amsfonts}
\usepackage{appendix}
\usepackage{listings}
\usepackage{xcolor}
\usepackage{geometry}
\setcounter{secnumdepth}{4}
\usepackage{mhchem}
\usepackage{multirow}
\usepackage{extarrows}
\usepackage{hyperref}
\titleformat*{\section}{\LARGE}
\renewcommand\refname{参考文献}
\renewcommand{\abstractname}{\sihao \cjkfzcs 摘{  }要}
%\titleformat{\chapter}{\centering\bfseries\huge\wryh}{}{0.7em}{}{}
%\titleformat{\section}{\LARGE\bf}{\thesection}{1em}{}{}
\titleformat{\subsection}{\Large\bfseries}{\thesubsection}{1em}{}{}
\titleformat{\subsubsection}{\large\bfseries}{\thesubsubsection}{1em}{}{}
\renewcommand{\contentsname}{{\cjkfzcs \centerline{目{  } 录}}}
\setCJKfamilyfont{cjkhwxk}{STXingkai}
\setCJKfamilyfont{cjkfzcs}{STSongti-SC-Regular}
% \setCJKfamilyfont{cjkhwxk}{华文行楷}
% \setCJKfamilyfont{cjkfzcs}{方正粗宋简体}
\newcommand*{\cjkfzcs}{\CJKfamily{cjkfzcs}}
\newcommand*{\cjkhwxk}{\CJKfamily{cjkhwxk}}
\newfontfamily\wryh{Microsoft YaHei}
\newfontfamily\hwzs{STZhongsong}
\newfontfamily\hwst{STSong}
\newfontfamily\hwfs{STFangsong}
\newfontfamily\jljt{MicrosoftYaHei}
\newfontfamily\hwxk{STXingkai}
% \newfontfamily\hwzs{华文中宋}
% \newfontfamily\hwst{华文宋体}
% \newfontfamily\hwfs{华文仿宋}
% \newfontfamily\jljt{方正静蕾简体}
% \newfontfamily\hwxk{华文行楷}
\newcommand{\verylarge}{\fontsize{60pt}{\baselineskip}\selectfont}  
\newcommand{\chuhao}{\fontsize{44.9pt}{\baselineskip}\selectfont}  
\newcommand{\xiaochu}{\fontsize{38.5pt}{\baselineskip}\selectfont}  
\newcommand{\yihao}{\fontsize{27.8pt}{\baselineskip}\selectfont}  
\newcommand{\xiaoyi}{\fontsize{25.7pt}{\baselineskip}\selectfont}  
\newcommand{\erhao}{\fontsize{23.5pt}{\baselineskip}\selectfont}  
\newcommand{\xiaoerhao}{\fontsize{19.3pt}{\baselineskip}\selectfont} 
\newcommand{\sihao}{\fontsize{14pt}{\baselineskip}\selectfont}      % 字号设置  
\newcommand{\xiaosihao}{\fontsize{12pt}{\baselineskip}\selectfont}  % 字号设置  
\newcommand{\wuhao}{\fontsize{10.5pt}{\baselineskip}\selectfont}    % 字号设置  
\newcommand{\xiaowuhao}{\fontsize{9pt}{\baselineskip}\selectfont}   % 字号设置  
\newcommand{\liuhao}{\fontsize{7.875pt}{\baselineskip}\selectfont}  % 字号设置  
\newcommand{\qihao}{\fontsize{5.25pt}{\baselineskip}\selectfont}    % 字号设置 

\usepackage{diagbox}
\usepackage{multirow}
\boldmath
\XeTeXlinebreaklocale "zh"
\XeTeXlinebreakskip = 0pt plus 1pt minus 0.1pt
\definecolor{cred}{rgb}{0.8,0.8,0.8}
\definecolor{cgreen}{rgb}{0,0.3,0}
\definecolor{cpurple}{rgb}{0.5,0,0.35}
\definecolor{cdocblue}{rgb}{0,0,0.3}
\definecolor{cdark}{rgb}{0.95,1.0,1.0}
\lstset{
	language=bash,
	numbers=left,
	numberstyle=\tiny\color{black},
	showspaces=false,
	showstringspaces=false,
	basicstyle=\scriptsize,
	keywordstyle=\color{purple},
	commentstyle=\itshape\color{cgreen},
	stringstyle=\color{blue},
	frame=lines,
	% escapeinside=``,
	extendedchars=true, 
	xleftmargin=1em,
	xrightmargin=1em, 
	backgroundcolor=\color{cred},
	aboveskip=1em,
	breaklines=true,
	tabsize=4
} 

\newfontfamily{\consolas}{Consolas}
\newfontfamily{\monaco}{Monaco}
\setmonofont[Mapping={}]{Consolas}	%英文引号之类的正常显示,相当于设置英文字体
\setsansfont{Consolas} %设置英文字体 Monaco, Consolas,  Fantasque Sans Mono
\setmainfont{Times New Roman}

\setCJKmainfont{华文中宋}


\newcommand{\fic}[1]{\begin{figure}[H]
		\center
		\includegraphics[width=0.8\textwidth]{#1}
	\end{figure}}
	
\newcommand{\sizedfic}[2]{\begin{figure}[H]
		\center
		\includegraphics[width=#1\textwidth]{#2}
	\end{figure}}

\newcommand{\codefile}[1]{\lstinputlisting{#1}}

% 改变段间隔
\setlength{\parskip}{0.2em}
\linespread{1.1}

\usepackage{lastpage}
\usepackage{fancyhdr}
\pagestyle{fancy}
\lhead{\space \qquad \space}
\chead{makefile中的隐含规则 \qquad}
\rhead{\qquad\thepage/\pageref{LastPage}}
\begin{document}

\tableofcontents

\clearpage

\section{使用隐含规则}
	隐含规则是make事先约定好的一些东西。如果我们没有写出某个目标的生成规则,make将自动推导产生这个目标的规则,此时,我们就相当于使用了隐含规则。例子如下:
	\begin{lstlisting}
	# 未使用隐含规则
	foo: foo.o bar.o
		cc -o foo foo.o bar.o $(CFLAGS) $(LDFLAGS)
	foo.o: foo.c
		cc -c foo.c $(CFLAGS)
	bar.o: bar.c
		cc -c bar.c $(CFLAGS)

	# 使用隐含规则
	foo: foo.o bar.o
		cc -o foo foo.o bar.o $(CFLAGS) $(LDFLAGS)
	\end{lstlisting}

\section{常用的隐含规则}
	一些常用的隐含规则如下所示:
	\begin{itemize}
		\item 编译C程序的隐含规则,<n>.o的依赖目标默认为<n>.c,命令如下:
		\begin{lstlisting}
	$(CC) -c $(CPPFLAGS) $(CFLAGS)
		\end{lstlisting}

		\item 编译C++程序的隐含规则,<n>.o的依赖目标默认为<n>.cc,命令如下:
		\begin{lstlisting}
	$(CXX) -c $(CPPFLAGS) $(CFLAGS)
		\end{lstlisting}

		\item 编译汇编的隐含规则,<n>.o的依赖目标默认为<n>.s,默认编译器为as,命令如下:
		\begin{lstlisting}
	$(AS) $(ASFLAGS)
		\end{lstlisting}

		\item 汇编预处理的隐含规则,<n>.s的依赖目标默认为<n>.S,默认编译器为cpp,命令如下:
		\begin{lstlisting}
	$(AS) $(ASFLAGS)
		\end{lstlisting}

		\item 链接object文件的隐含规则,<n>的依赖目标默认为<n>.o,命令如下:
		\begin{lstlisting}
	$(CC) $(LDFLAGS) <n>.o $(LOADLIBES) $(LDLIBS)
		\end{lstlisting}
	\end{itemize}

\section{隐含规则使用的变量}
	隐含规则中的变量一般有两种,一种是与命令相关的,一种是与参数相关的。下面介绍一些常用的变量。\par

	与命令相关的变量如下:
	\begin{itemize}
		\item AS,默认值为as。
		\item CC,默认值为cc。
		\item CXX,默认值为g++。
		\item CPP,默认值为C程序的预处理器。
		\item RM,默认值为rm。
	\end{itemize}

	与参数相关的变量如下:
	\begin{itemize}
		\item CFLAGS,C语言编译器参数。
		\item CXXFLAGS,C++语言编译器参数。
		\item CPPFLAGS,C预处理器参数。
	\end{itemize}

\section{模式规则}
\subsection{模式规则的介绍}
	在模式规则中,目标和依赖目标都必须有“\%”,目标中的“\%”值决定了依赖目标中“\%”的值。例子如下:
	\begin{lstlisting}
	%.o: %.c; <command>
	\end{lstlisting}

	上述例子中,如果要生成的目标是“a.o b.o”,那么“\%c”就是“a.c b.c”。

\subsection{自动化变量}
	所有的自动化变量如下所示:
	\begin{itemize}
		\item \$@。表示规则中的目标文件集。
		\item \$\%。如果目标是函数库文件,那么该值是规则中的目标成员名,否则该值为空。
		\item \$<。表示依赖目标中的第一个目标,如果依赖目标由\%定义,那么该值是一系列的文件集,并且是一个一个取出来的。
		\item \$?。表示所有比目标新的依赖目标的集合。
		\item \$\^。表示所有依赖目标的集合,不允许存在重复的依赖目标。
		\item \$+。表示所有依赖目标的集合,允许存在重复的依赖目标。
		\item \$*。表示目标模式中“\%”及其之前的部分。如果目标中没有模式的定义,只要是目标文件的后缀make可以识别,那么\$*就是除了后缀的那一部分,否则\$*就是空值。
	\end{itemize}

\subsection{模式匹配的注意点}
	当一个模式匹配包含有斜杠的文件时,那么在进行模式匹配时,目录部分会首先移开,然后进行匹配,成功后,再把目录加回去。例子如下:
	\begin{lstlisting}
	# 有模式“e%t”,依赖目标的模式是“c%r”
	# 目标文件是“src/eat”
	# 那么依赖目标文件是“src/car”
	\end{lstlisting}

  
\end{document}
