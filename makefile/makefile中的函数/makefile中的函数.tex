% !TeX spellcheck = en_US
%% 字体:方正静蕾简体
%%		 方正粗宋
\documentclass[a4paper,left=2.5cm,right=2.5cm,11pt]{article}

\usepackage[utf8]{inputenc}
\usepackage{fontspec}
\usepackage{cite}
\usepackage{xeCJK}
\usepackage{indentfirst}
\usepackage{titlesec}
\usepackage{longtable}
\usepackage{graphicx}
\usepackage{float}
\usepackage{rotating}
\usepackage{subfigure}
\usepackage{tabu}
\usepackage{amsmath}
\usepackage{setspace}
\usepackage{amsfonts}
\usepackage{appendix}
\usepackage{listings}
\usepackage{xcolor}
\usepackage{geometry}
\setcounter{secnumdepth}{4}
\usepackage{mhchem}
\usepackage{multirow}
\usepackage{extarrows}
\usepackage{hyperref}
\titleformat*{\section}{\LARGE}
\renewcommand\refname{参考文献}
\renewcommand{\abstractname}{\sihao \cjkfzcs 摘{  }要}
%\titleformat{\chapter}{\centering\bfseries\huge\wryh}{}{0.7em}{}{}
%\titleformat{\section}{\LARGE\bf}{\thesection}{1em}{}{}
\titleformat{\subsection}{\Large\bfseries}{\thesubsection}{1em}{}{}
\titleformat{\subsubsection}{\large\bfseries}{\thesubsubsection}{1em}{}{}
\renewcommand{\contentsname}{{\cjkfzcs \centerline{目{  } 录}}}
\setCJKfamilyfont{cjkhwxk}{STXingkai}
\setCJKfamilyfont{cjkfzcs}{STSongti-SC-Regular}
% \setCJKfamilyfont{cjkhwxk}{华文行楷}
% \setCJKfamilyfont{cjkfzcs}{方正粗宋简体}
\newcommand*{\cjkfzcs}{\CJKfamily{cjkfzcs}}
\newcommand*{\cjkhwxk}{\CJKfamily{cjkhwxk}}
\newfontfamily\wryh{Microsoft YaHei}
\newfontfamily\hwzs{STZhongsong}
\newfontfamily\hwst{STSong}
\newfontfamily\hwfs{STFangsong}
\newfontfamily\jljt{MicrosoftYaHei}
\newfontfamily\hwxk{STXingkai}
% \newfontfamily\hwzs{华文中宋}
% \newfontfamily\hwst{华文宋体}
% \newfontfamily\hwfs{华文仿宋}
% \newfontfamily\jljt{方正静蕾简体}
% \newfontfamily\hwxk{华文行楷}
\newcommand{\verylarge}{\fontsize{60pt}{\baselineskip}\selectfont}  
\newcommand{\chuhao}{\fontsize{44.9pt}{\baselineskip}\selectfont}  
\newcommand{\xiaochu}{\fontsize{38.5pt}{\baselineskip}\selectfont}  
\newcommand{\yihao}{\fontsize{27.8pt}{\baselineskip}\selectfont}  
\newcommand{\xiaoyi}{\fontsize{25.7pt}{\baselineskip}\selectfont}  
\newcommand{\erhao}{\fontsize{23.5pt}{\baselineskip}\selectfont}  
\newcommand{\xiaoerhao}{\fontsize{19.3pt}{\baselineskip}\selectfont} 
\newcommand{\sihao}{\fontsize{14pt}{\baselineskip}\selectfont}      % 字号设置  
\newcommand{\xiaosihao}{\fontsize{12pt}{\baselineskip}\selectfont}  % 字号设置  
\newcommand{\wuhao}{\fontsize{10.5pt}{\baselineskip}\selectfont}    % 字号设置  
\newcommand{\xiaowuhao}{\fontsize{9pt}{\baselineskip}\selectfont}   % 字号设置  
\newcommand{\liuhao}{\fontsize{7.875pt}{\baselineskip}\selectfont}  % 字号设置  
\newcommand{\qihao}{\fontsize{5.25pt}{\baselineskip}\selectfont}    % 字号设置 

\usepackage{diagbox}
\usepackage{multirow}
\boldmath
\XeTeXlinebreaklocale "zh"
\XeTeXlinebreakskip = 0pt plus 1pt minus 0.1pt
\definecolor{cred}{rgb}{0.8,0.8,0.8}
\definecolor{cgreen}{rgb}{0,0.3,0}
\definecolor{cpurple}{rgb}{0.5,0,0.35}
\definecolor{cdocblue}{rgb}{0,0,0.3}
\definecolor{cdark}{rgb}{0.95,1.0,1.0}
\lstset{
	language=bash,
	numbers=left,
	numberstyle=\tiny\color{black},
	showspaces=false,
	showstringspaces=false,
	basicstyle=\scriptsize,
	keywordstyle=\color{purple},
	commentstyle=\itshape\color{cgreen},
	stringstyle=\color{blue},
	frame=lines,
	% escapeinside=``,
	extendedchars=true, 
	xleftmargin=1em,
	xrightmargin=1em, 
	backgroundcolor=\color{cred},
	aboveskip=1em,
	breaklines=true,
	tabsize=4
} 

\newfontfamily{\consolas}{Consolas}
\newfontfamily{\monaco}{Monaco}
\setmonofont[Mapping={}]{Consolas}	%英文引号之类的正常显示,相当于设置英文字体
\setsansfont{Consolas} %设置英文字体 Monaco, Consolas,  Fantasque Sans Mono
\setmainfont{Times New Roman}

\setCJKmainfont{华文中宋}


\newcommand{\fic}[1]{\begin{figure}[H]
		\center
		\includegraphics[width=0.8\textwidth]{#1}
	\end{figure}}
	
\newcommand{\sizedfic}[2]{\begin{figure}[H]
		\center
		\includegraphics[width=#1\textwidth]{#2}
	\end{figure}}

\newcommand{\codefile}[1]{\lstinputlisting{#1}}

% 改变段间隔
\setlength{\parskip}{0.2em}
\linespread{1.1}

\usepackage{lastpage}
\usepackage{fancyhdr}
\pagestyle{fancy}
\lhead{\space \qquad \space}
\chead{makefile中的函数 \qquad}
\rhead{\qquad\thepage/\pageref{LastPage}}
\begin{document}

\tableofcontents

\clearpage

\section{函数的调用}
	函数的调用语法如下所示:
	\begin{lstlisting}
	$(<function> <arguments>)
	\end{lstlisting}

	<function>是函数名,<arguments>是函数参数。函数名与函数参数用“空格”分隔,参数间用“,”分隔。函数的变量可以使用参数,例子如下:
	\begin{lstlisting}
	$(subst a,b,$(x))
	\end{lstlisting}

\section{字符串处理函数}
\subsection{subst函数}
	\begin{lstlisting}
	$(subst <from>,<to>,<text>)
	# 功能:把字符串<text>中的<from>字符串替换为<to>字符串
	# 返回:被替换后的字符串
	\end{lstlisting}

\subsection{patsubst}
	\begin{lstlisting}
	$(patsubst <pattern>,<replacement>,<text>)
	# 功能:text中的单词用“空格”、“tab”、“回车”或“换行”分隔。如果单词符合模式pattern,就用replacement代替。pattern中可以有%,用于表示任意长度的字符串。如果replacement也有%,那么这里的%将是pattern中%代表的字符串。
	# 返回:被替换后的字符串
	\end{lstlisting}

\subsection{strip}
	\begin{lstlisting}
	$(strip <string>)
	# 功能:去掉string字符串中开头和结尾的空字符。
	# 返回:被去掉空格的字符串
	\end{lstlisting}

\subsection{findstring}
	\begin{lstlisting}
	$(findstring <find>,<in>)
	# 功能:在字符串<in>中查找<find>字符串
	# 返回:如果找到,就返回find,否则返回空字符串。=
	\end{lstlisting}

\subsection{filter}
	\begin{lstlisting}
	$(filter <pattern>,<text>)
	# 功能:保留text中符合pattern模式的单词。可以有多个模式,模式间用“空格”分隔。
	# 返回:返回符合模式pattern的字符串
	\end{lstlisting}

\subsection{filter-out}
	\begin{lstlisting}
	$(filter-out <pattern>,<text>)
	# 功能:删除text中符合pattern模式的单词。可以有多个模式,模式间用“空格”分隔。
	# 返回:返回不符合模式pattern的字符串
	\end{lstlisting}

\subsection{sort}
	\begin{lstlisting}
	$(sort <list>)
	# 功能:按字典序给list中的单词排序,单词间按“空格”分隔。sort函数还会去掉list中相同的单词。
	# 返回:排序后的字符串。
	\end{lstlisting}

\subsection{word}
	\begin{lstlisting}
	$(word <n>,<text>)
	# 功能:取出text中的第n个单词,从1开始。
	# 返回:返回text中的第n个单词。如果n大于text的长度,则返回空字符串。
	\end{lstlisting}

\subsection{wordlist}
	\begin{lstlisting}
	$(wordlist <s>,<e>,<text>)
	# 返回:text中从s到e的单词,从1开始。
	\end{lstlisting}

\subsection{words}
	\begin{lstlisting}
	$(words <text>)
	# 返回:text中的单词数。
	\end{lstlisting}

\subsection{firstword}
	\begin{lstlisting}
	$(firstword <text>)
	# 返回:text中的第一个单词。
	\end{lstlisting}

\section{文件名操作函数}
\subsection{dir}
	\begin{lstlisting}
	$(dir <names>)
	# 返回:文件名的目录部分。如果没有反斜杠,就返回“./”
	\end{lstlisting}

\subsection{notdir}
	\begin{lstlisting}
	$(notdir <names>)
	# 返回:文件名的非目录部分。
	\end{lstlisting}

\subsection{suffix}
	\begin{lstlisting}
	$(suffix <names>)
	# 返回:各个文件名的后缀。如果文件没有后缀,就返回空字符串。
	\end{lstlisting}

\subsection{basename}
	\begin{lstlisting}
	$(basename <names>)
	# 返回:各个文件名的前缀。如果文件没有前缀,就返回空字符串。
	\end{lstlisting}

\subsection{addsuffix}
	\begin{lstlisting}
	$(addsuffix <suffix>,<names>)
	# 返回:将后缀suffix加到names中每个单词后面,然后返回结果。
	\end{lstlisting}

\subsection{addprefix}
	\begin{lstlisting}
	$(addprefix <prefix>,<names>)
	# 返回:将前缀prefix加到names中每个单词前面,然后返回结果。
	\end{lstlisting}

\subsection{join}
	\begin{lstlisting}
	$(join <list1>,<list2>)
	# 功能:将list2中每个单词对应地加到list1的单词后面。如果两者单词数量不一样,就将相应多出的单词保持原样。
	\end{lstlisting}

\section{foreach函数}
	\begin{lstlisting}
	$(foreach <var>,<list>,<text>)
	# 功能:将list中的单词放到var中指定的变量中,然后执行text中所包含的表达式,每一个text回返回一个字符串。
	\end{lstlisting}

	例子如下:
	\begin{lstlisting}
	names := a b c d
	files := $(foreach n,$(names),$(n).o)
	\end{lstlisting}

	需要注意的是,foreach中的var是一个局部变量,其作用域中只在foreach函数中。

\section{if函数}
	\begin{lstlisting}
	$(if <condition>,<then-part>)
	$(if <condition>,<then-part>,<else-part>)
	# 如果else-part没有定义,那么当条件为假时,将返回空字符串。
	\end{lstlisting}

\section{call函数}
	\begin{lstlisting}
	$(call <expression>,<param1>,<param2>,<param3>...)
	# 功能:向表达式中传递参数
	\end{lstlisting}

	例子如下:
	\begin{lstlisting}
	foo = $(call $(1) $(2),a,b)
	# foo为a b
	\end{lstlisting}

\section{origin函数}
	\begin{lstlisting}
	$(origin <variable>)
	# 功能:返回变量在哪里定义的。
	# 返回值:
	# undefined,变量未定义
	# default,变量是一个默认的定义
	# environment,变量是环境变量
	# file,变量定义在makefile中
	# command line,变量由命令行定义
	# override,变量被override指示符重新定义过
	# automatic,变量是一个自动化变量
	\end{lstlisting}

\section{shell函数}
	\begin{lstlisting}
	$(shell command)
	# 功能:返回命令行的输出
	\end{lstlisting}

	例子如下:
	\begin{lstlisting}
	files := $(shell echo *.c)
	\end{lstlisting}

\section{控制make的函数}
\subsection{error}
	\begin{lstlisting}
	$(error <text>)
	# 功能:产生一个错误,text是错误信息,输出text后,终止make的运行。
	\end{lstlisting}

\subsection{warning}
	\begin{lstlisting}
	$(warning <text>)
	# 功能:text是警告信息,输出警告信息后,让make继续运行。
	\end{lstlisting}

\end{document}
