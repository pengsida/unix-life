% !TeX spellcheck = en_US
%% 字体:方正静蕾简体
%%		 方正粗宋
\documentclass[a4paper,left=2.5cm,right=2.5cm,11pt]{article}

\usepackage[utf8]{inputenc}
\usepackage{fontspec}
\usepackage{cite}
\usepackage{xeCJK}
\usepackage{indentfirst}
\usepackage{titlesec}
\usepackage{longtable}
\usepackage{graphicx}
\usepackage{float}
\usepackage{rotating}
\usepackage{subfigure}
\usepackage{tabu}
\usepackage{amsmath}
\usepackage{setspace}
\usepackage{amsfonts}
\usepackage{appendix}
\usepackage{listings}
\usepackage{xcolor}
\usepackage{geometry}
\setcounter{secnumdepth}{4}
\usepackage{mhchem}
\usepackage{multirow}
\usepackage{extarrows}
\usepackage{hyperref}
\titleformat*{\section}{\LARGE}
\renewcommand\refname{参考文献}
\renewcommand{\abstractname}{\sihao \cjkfzcs 摘{  }要}
%\titleformat{\chapter}{\centering\bfseries\huge\wryh}{}{0.7em}{}{}
%\titleformat{\section}{\LARGE\bf}{\thesection}{1em}{}{}
\titleformat{\subsection}{\Large\bfseries}{\thesubsection}{1em}{}{}
\titleformat{\subsubsection}{\large\bfseries}{\thesubsubsection}{1em}{}{}
\renewcommand{\contentsname}{{\cjkfzcs \centerline{目{  } 录}}}
\setCJKfamilyfont{cjkhwxk}{STXingkai}
\setCJKfamilyfont{cjkfzcs}{STSongti-SC-Regular}
% \setCJKfamilyfont{cjkhwxk}{华文行楷}
% \setCJKfamilyfont{cjkfzcs}{方正粗宋简体}
\newcommand*{\cjkfzcs}{\CJKfamily{cjkfzcs}}
\newcommand*{\cjkhwxk}{\CJKfamily{cjkhwxk}}
\newfontfamily\wryh{Microsoft YaHei}
\newfontfamily\hwzs{STZhongsong}
\newfontfamily\hwst{STSong}
\newfontfamily\hwfs{STFangsong}
\newfontfamily\jljt{MicrosoftYaHei}
\newfontfamily\hwxk{STXingkai}
% \newfontfamily\hwzs{华文中宋}
% \newfontfamily\hwst{华文宋体}
% \newfontfamily\hwfs{华文仿宋}
% \newfontfamily\jljt{方正静蕾简体}
% \newfontfamily\hwxk{华文行楷}
\newcommand{\verylarge}{\fontsize{60pt}{\baselineskip}\selectfont}  
\newcommand{\chuhao}{\fontsize{44.9pt}{\baselineskip}\selectfont}  
\newcommand{\xiaochu}{\fontsize{38.5pt}{\baselineskip}\selectfont}  
\newcommand{\yihao}{\fontsize{27.8pt}{\baselineskip}\selectfont}  
\newcommand{\xiaoyi}{\fontsize{25.7pt}{\baselineskip}\selectfont}  
\newcommand{\erhao}{\fontsize{23.5pt}{\baselineskip}\selectfont}  
\newcommand{\xiaoerhao}{\fontsize{19.3pt}{\baselineskip}\selectfont} 
\newcommand{\sihao}{\fontsize{14pt}{\baselineskip}\selectfont}      % 字号设置  
\newcommand{\xiaosihao}{\fontsize{12pt}{\baselineskip}\selectfont}  % 字号设置  
\newcommand{\wuhao}{\fontsize{10.5pt}{\baselineskip}\selectfont}    % 字号设置  
\newcommand{\xiaowuhao}{\fontsize{9pt}{\baselineskip}\selectfont}   % 字号设置  
\newcommand{\liuhao}{\fontsize{7.875pt}{\baselineskip}\selectfont}  % 字号设置  
\newcommand{\qihao}{\fontsize{5.25pt}{\baselineskip}\selectfont}    % 字号设置 

\usepackage{diagbox}
\usepackage{multirow}
\boldmath
\XeTeXlinebreaklocale "zh"
\XeTeXlinebreakskip = 0pt plus 1pt minus 0.1pt
\definecolor{cred}{rgb}{0.8,0.8,0.8}
\definecolor{cgreen}{rgb}{0,0.3,0}
\definecolor{cpurple}{rgb}{0.5,0,0.35}
\definecolor{cdocblue}{rgb}{0,0,0.3}
\definecolor{cdark}{rgb}{0.95,1.0,1.0}
\lstset{
	language=bash,
	numbers=left,
	numberstyle=\tiny\color{black},
	showspaces=false,
	showstringspaces=false,
	basicstyle=\scriptsize,
	keywordstyle=\color{purple},
	commentstyle=\itshape\color{cgreen},
	stringstyle=\color{blue},
	frame=lines,
	% escapeinside=``,
	extendedchars=true, 
	xleftmargin=1em,
	xrightmargin=1em, 
	backgroundcolor=\color{cred},
	aboveskip=1em,
	breaklines=true,
	tabsize=4
} 

\newfontfamily{\consolas}{Consolas}
\newfontfamily{\monaco}{Monaco}
\setmonofont[Mapping={}]{Consolas}	%英文引号之类的正常显示,相当于设置英文字体
\setsansfont{Consolas} %设置英文字体 Monaco, Consolas,  Fantasque Sans Mono
\setmainfont{Times New Roman}

\setCJKmainfont{华文中宋}


\newcommand{\fic}[1]{\begin{figure}[H]
		\center
		\includegraphics[width=0.8\textwidth]{#1}
	\end{figure}}
	
\newcommand{\sizedfic}[2]{\begin{figure}[H]
		\center
		\includegraphics[width=#1\textwidth]{#2}
	\end{figure}}

\newcommand{\codefile}[1]{\lstinputlisting{#1}}

\newcommand{\interval}{\vspace{0.5em}}

\newcommand{\tablestart}{
	\interval
	\begin{longtable}{p{2cm}p{10cm}}
	\hline}
\newcommand{\tableend}{
	\hline
	\end{longtable}
	\interval}

% 改变段间隔
\setlength{\parskip}{0.2em}
\linespread{1.1}

\usepackage{lastpage}
\usepackage{fancyhdr}
\pagestyle{fancy}
\lhead{\space \qquad \space}
\chead{处理用户输入 \qquad}
\rhead{\qquad\thepage/\pageref{LastPage}}
\begin{document}

\tableofcontents

\clearpage

\section{命令行参数}
	向shell脚本传递数据的最基本方式就是使用命令行参数。
	bash shell将命令行参数赋值给一些特殊变量,这些变量称为位置参数。
	位置参数通过标准数字表示,其中\$0为程序名称,\$1为第一个参数,\$2为第二个参数,以此类推,直到\$9为第九个参数。\par

	使用命令行参数的例子如下所示:
	\begin{lstlisting}
	#!/bin/bash

	factorial=1
	for (( number=1; number <= $1; number++ ))
	do
		factorial=$[ $factorial * $number ]
	done
	\end{lstlisting}

	要使用这个程序,只要在命令行输入:
	\begin{lstlisting}
	# test是文件名,5是程序的参数
	./test 5
	\end{lstlisting}

	有时候可以利用程序名称,从而编写不同名称有不同功能的脚本,例子如下所示:
	\begin{lstlisting}
	#!/bin/bash
	# basename是一个命令,可以只返回程序的名称,不带路径
	name=`basename $0`
	if [ $name = "addem" ]
	then
		total=$[ $1 + $2 ]
	elif [ $name = "multem" ]
	then
		total=$[ $1 * $2 ]
	fi
	\end{lstlisting}

	然后只要将程序名称改为不同的名字,就有不一样的功能了,如下所示:
	\begin{lstlisting}
	# test是原脚本名称
	cp test addem
	cp test multem
	# 执行程序
	./addem 2 5
	./multem 2 5
	\end{lstlisting}

	需要注意的是,在使用参数之前,需要先使用-n参数来检查命令行参数是否存在数据,例子如下:
	\begin{lstlisting}
	#!/bin/bash
	# 带双引号是考虑到了第一个参数有空格的情况
	if [ -n "$1" ]
	then
		echo "Hello $1, glad to meet you"
	else
		echo "Sorry, you didn't identify yourself"
	fi
	\end{lstlisting}

\subsection{特殊的参数变量}
\section{特殊的参数变量\$\#}
	特殊变量\$\#中存储着执行脚本时的命令行参数个数,所以我们就可以利用它在使用参数前测试现有的参数个数,例子如下:
	\begin{lstlisting}
	#!/bin/bash

	if [ $# -ne 2 ]
	then
		echo "Error"
	else
		total=$[ $1 + $2 ]
		echo "The total is $total"
	fi
	\end{lstlisting}

	通过这个变量还可以直接获取最后一个参数的值,不过此时必须用感叹号替代美元符号,格式如下:
	\begin{lstlisting}
	$(!#)
	\end{lstlisting}

\subsection{特殊的参数变量\$*和\$@}
	\$*和\$@这两个变量都包含了所有的命令行参数,\$*将所有命令行参数当作一个单词处理,而\$@把命令行参数当作多个单词处理,所以只有\$@可以被for命令遍历,例子如下所示:
	\begin{lstlisting}
	#!/bin/bash
	count=1
	for param in "$*"
	do
		echo "\$* Parameter #$count = $param"
		count=$[ $count + 1 ]
	done

	count=1
	for param in "$@"
	do
		echo "\$@ Parameter #$count: $param"
		count=$[ $count + 1 ]
	done
	\end{lstlisting}

\section{shift移位命令}
	shift命令可以改变命令行参数的相对位置,将每个参数变量左移一位,\$3的值移给\$2,\$2移给\$1,原先\$1的值被丢弃。
	需要注意的是,\$0的值保持不变。\par
	这样子就可以先对第一个参数进行操作,然后对参数进行一次左移,再对第一个参数进行操作,如下例所示:
	\begin{lstlisting}
	#!/bin/bash
	count=1
	while [ -n $1 ]
	do
		echo "Parameter #$count = $1"
		count=$[ $count + 1]
		shift
	done
	\end{lstlisting}

	shift命令还可以带参数,指定左移几位,格式如下:
	\begin{lstlisting}
	shift n
	\end{lstlisting}

\section{处理参数项}
	shell程序的输入有时候是带参数项的,如下所示:
	\begin{lstlisting}
	./testing -a -b test1 -c
	\end{lstlisting}

	例子如下所示:
	\begin{lstlisting}
	#!/bin/bash
	while [ -n "$1" ]
	do
		case "$1" in
		-a)
			echo "Found the -a option";;
		-b)
			param=$2
			echo "Found the -b option, with parameter value $param"
			shift;;
		-c)
			echo "Found the -c option";;
		--)
			shift
			break;;
		*)
			echo "$1 is not an option";;
		esac
		shift
	done
	\end{lstlisting}

\subsection{getopt命令}
	getopt命令可以接受任意形式的命令行选项和参数列表,并将这些选项和参数转换为适当的格式。\par
	getopt命令的格式如下所示:
	\begin{lstlisting}
	getopt [options] <optstring> <parameters>
	# optstring定义了命令行中的有效选项字母,并且定义了哪些选项字母需要参数值
	# optstring由选项字母组成,需要参数值的选项字母后面需要放置一个冒号
	# optstring的例子:ab:cd
	\end{lstlisting}

	getopt的使用例子如下:
	\begin{lstlisting}
	getopt ab:cd -a -b test1 -cd test2 test3
	# 返回:-a -b test1 -c -d -- test2 test3
	\end{lstlisting}

	当paramter中的选项不在optstring中时,getopt会产生错误消息,通过“-q”参数项可以忽略这个错误消息,例子如下:
	\begin{lstlisting}
	getopt -q ab:cd -a -b test1 -cde test2 test3
	\end{lstlisting}

	通过set命令带“--”参数项可以将命令行参数项替换为set命令的输入值,set命令和getopt的配合使用如下所示:
	\begin{lstlisting}
	set -- `getopt -q ab:cd "$@"`
	\end{lstlisting}

	具体例子如下所示:
	\begin{lstlisting}
	#!/bin/bash

	set -- `getopt -q ab:c "$@"`
	while [ -n "$1" ]
	do
		case "$1" in
		-a)
			echo "Found the -a option";;
		-b)
			param="$2"
			echo "Found the -b option with the parameter $param"
			shift;;
		-c)
			echo "Found the -c option";;
		--)
			shift
			break;;
		*)
			echo "$1 is not an option";;
		esac
		shift
	done
	\end{lstlisting}

\subsection{getopts命令}
	getopts命令的格式和getopt类似,如下所示:
	\begin{lstlisting}
	getopts optstring variable
	\end{lstlisting}

	getopts按顺序处理命令行参数,并给variable赋值。如果要禁止输出错误消息,需要在optstring前加上冒号“:”。\par
	环境变量OPTARG包含需要参数值的选项要使用的值,环境变量OPTIND包含getopts停止处理时在参数列表中的位置。\par
	使用OPTARG的例子如下所示:
	\begin{lstlisting}
	#!/bin/bash

	while getopts :ab:c opt
	do
		case "$opt" in
		# getopts会把破折号-去掉,所以case选项中不需要破折号
		a)
			echo "Found the -a option";;
		b) 
			echo "Found the -b option, with value $OPTARG";;
		c)
			echo "Found the -c option";;
		*)
			# 未知的参数项将变为问好?
			echo "Unknown option: $opt";;
		esac
	done
	\end{lstlisting}

	可以利用OPTIND来跳过命令行参数的参数项,直接获得参数,例子如下所示:
	\begin{lstlisting}
	#!/bin/bash

	while getopts :ab:cd opt
	do
		case "$opt" in
		a)
			echo "Found the -a option";;
		b)
			echo "Found the -b option, with value $OPTARG";;
		c)
			echo "Found the -c option";;
		d)
			echo "Found the -d option";;
		*)
			echo "Unknown option: $opt";;
		esac
	done
	# 利用shift和OPTIND来跳过命令行参数中的参数项
	shift $[ $OPTIND - 1 ]

	count=1
	for param in "$@"
	do
		echo "Parameter $count: $param"
		count=$[ $count + 1 ]
	done
	\end{lstlisting}

\section{标准化选项}
	linux世界中有一些字母选项具有某种标准含义,如果在shell脚本中使用这些选项,会使脚本变得更加便于使用。
	这些命令行选项如下所示:
	\fic{1.png}

\section{获取用户输入}
\subsection{read命令}
	read命令可以将用户输入放入一个标准变量中,格式如下:
	\begin{lstlisting}
	read variable
	\end{lstlisting}

	例子如下:
	\begin{lstlisting}
	#!/bin/bash

	echo -n "Enter your name: "
	read name
	echo "Hello $name, welcome to my program."
	\end{lstlisting}

\subsubsection{read命令的各个选项}

	read命令带“-p”选项时,可以直接在read命令中指定一个提示,如下例所示:
	\begin{lstlisting}
	#!/bin/bash

	read -p "Please enter your age: " age
	days=$[ $age * 365 ]
	echo "That makes you over $days days old!"
	\end{lstlisting}

	如果不指定变量,read命令会把用户输入放到环境变量REPLY中,如下例所示:
	\begin{lstlisting}
	#!/bin/bash

	read -p "Enter a number: "
	factorial=1
	for ((  count=1; count <= $REPLY; count++ ))
	do
		factorial=$[ $factorial * $count ]
	done
	echo "The factorial of $REPLY is $factorial"
	\end{lstlisting}

	使用read命令时,为了防止脚本停下一直等待用户输入数据,可以带上“-t”选项,指定read命令等待输入的秒数。
	当计时器计时数满时,read命令返回一个非零退出状态,例子如下:
	\begin{lstlisting}
	#!/bin/bash

	if read -t 5 -p "Please enter your name: " name
	then
		echo "Hello $name, welcome to my script"
	else
		# 换行
		echo
		echo "Sorry, too slow!"
	fi
	\end{lstlisting}

	read命令还可以带上“-n”选项,当用户输入字符达到预定数目时,自动退出,并将输入的数据赋值给变量,如下例所示:
	\begin{lstlisting}
	#!/bin/bash

	read -n1 -p "Do you want to continue [Y/N]? " answer
	case $answer in
	Y | y)
		echo
		echo "fine, continue on";;
	N | n)
		echo
		echo "OK, goodbye"
		exit;;
	esac
	\end{lstlisting}

	需要注意的是,当read命令自动退出时,不会自动换行。\par

	read命令带上“-s”选项时,不会让用户输入显示在终端上,例子如下:
	\begin{lstlisting}
	#!/bin/bash

	read -s -p "Enter your password: " pass
	echo
	echo "Is your password really $pass"
	\end{lstlisting}

\subsubsection{使用read命令读取文件}
	read命令还可以读取文件,每一次read命令都会读取文件中的一行文本。
	当文件中没有可读的行时,read命令将以非零退出状态退出。read命令一般和cat、管道相配合使用,cat输出文件内容,再交给read命令读取,格式如下:
	\begin{lstlisting}
	cat file | while read line
	do
		commands
	done
	\end{lstlisting}
	
	read命令的使用如下例所示:
	\begin{lstlisting}
	#!/bin/bash
	count=1
	cat test | while read line
	do
		echo "Line $count: $line"
		count=$[ $count + 1 ]
	done
	\end{lstlisting}

\end{document}
