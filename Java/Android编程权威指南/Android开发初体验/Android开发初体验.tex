% !TeX spellcheck = en_US
%% 字体:方正静蕾简体
%%		 方正粗宋
\documentclass[a4paper,left=2.5cm,right=2.5cm,11pt]{article}

\usepackage[utf8]{inputenc}
\usepackage{fontspec}
\usepackage{cite}
\usepackage{xeCJK}
\usepackage{indentfirst}
\usepackage{titlesec}
\usepackage{longtable}
\usepackage{graphicx}
\usepackage{float}
\usepackage{rotating}
\usepackage{subfigure}
\usepackage{tabu}
\usepackage{amsmath}
\usepackage{setspace}
\usepackage{amsfonts}
\usepackage{appendix}
\usepackage{listings}
\usepackage{xcolor}
\usepackage{geometry}
\setcounter{secnumdepth}{4}
\usepackage{mhchem}
\usepackage{multirow}
\usepackage{extarrows}
\usepackage{hyperref}
\titleformat*{\section}{\LARGE}
\renewcommand\refname{参考文献}
\renewcommand{\abstractname}{\sihao \cjkfzcs 摘{  }要}
%\titleformat{\chapter}{\centering\bfseries\huge\wryh}{}{0.7em}{}{}
%\titleformat{\section}{\LARGE\bf}{\thesection}{1em}{}{}
\titleformat{\subsection}{\Large\bfseries}{\thesubsection}{1em}{}{}
\titleformat{\subsubsection}{\large\bfseries}{\thesubsubsection}{1em}{}{}
\renewcommand{\contentsname}{{\cjkfzcs \centerline{目{  } 录}}}
\setCJKfamilyfont{cjkhwxk}{STXingkai}
\setCJKfamilyfont{cjkfzcs}{STSongti-SC-Regular}
% \setCJKfamilyfont{cjkhwxk}{华文行楷}
% \setCJKfamilyfont{cjkfzcs}{方正粗宋简体}
\newcommand*{\cjkfzcs}{\CJKfamily{cjkfzcs}}
\newcommand*{\cjkhwxk}{\CJKfamily{cjkhwxk}}
\newfontfamily\wryh{Microsoft YaHei}
\newfontfamily\hwzs{STZhongsong}
\newfontfamily\hwst{STSong}
\newfontfamily\hwfs{STFangsong}
\newfontfamily\jljt{MicrosoftYaHei}
\newfontfamily\hwxk{STXingkai}
% \newfontfamily\hwzs{华文中宋}
% \newfontfamily\hwst{华文宋体}
% \newfontfamily\hwfs{华文仿宋}
% \newfontfamily\jljt{方正静蕾简体}
% \newfontfamily\hwxk{华文行楷}
\newcommand{\verylarge}{\fontsize{60pt}{\baselineskip}\selectfont}  
\newcommand{\chuhao}{\fontsize{44.9pt}{\baselineskip}\selectfont}  
\newcommand{\xiaochu}{\fontsize{38.5pt}{\baselineskip}\selectfont}  
\newcommand{\yihao}{\fontsize{27.8pt}{\baselineskip}\selectfont}  
\newcommand{\xiaoyi}{\fontsize{25.7pt}{\baselineskip}\selectfont}  
\newcommand{\erhao}{\fontsize{23.5pt}{\baselineskip}\selectfont}  
\newcommand{\xiaoerhao}{\fontsize{19.3pt}{\baselineskip}\selectfont} 
\newcommand{\sihao}{\fontsize{14pt}{\baselineskip}\selectfont}      % 字号设置  
\newcommand{\xiaosihao}{\fontsize{12pt}{\baselineskip}\selectfont}  % 字号设置  
\newcommand{\wuhao}{\fontsize{10.5pt}{\baselineskip}\selectfont}    % 字号设置  
\newcommand{\xiaowuhao}{\fontsize{9pt}{\baselineskip}\selectfont}   % 字号设置  
\newcommand{\liuhao}{\fontsize{7.875pt}{\baselineskip}\selectfont}  % 字号设置  
\newcommand{\qihao}{\fontsize{5.25pt}{\baselineskip}\selectfont}    % 字号设置 

\usepackage{diagbox}
\usepackage{multirow}
\boldmath
\XeTeXlinebreaklocale "zh"
\XeTeXlinebreakskip = 0pt plus 1pt minus 0.1pt
\definecolor{cred}{rgb}{0.8,0.8,0.8}
\definecolor{cgreen}{rgb}{0,0.3,0}
\definecolor{cpurple}{rgb}{0.5,0,0.35}
\definecolor{cdocblue}{rgb}{0,0,0.3}
\definecolor{cdark}{rgb}{0.95,1.0,1.0}
\lstset{
	language=bash,
	numbers=left,
	numberstyle=\tiny\color{black},
	showspaces=false,
	showstringspaces=false,
	basicstyle=\scriptsize,
	keywordstyle=\color{purple},
	commentstyle=\itshape\color{cgreen},
	stringstyle=\color{blue},
	frame=lines,
	% escapeinside=``,
	extendedchars=true, 
	xleftmargin=1em,
	xrightmargin=1em, 
	backgroundcolor=\color{cred},
	aboveskip=1em,
	breaklines=true,
	tabsize=4
} 

\newfontfamily{\consolas}{Consolas}
\newfontfamily{\monaco}{Monaco}
\setmonofont[Mapping={}]{Consolas}	%英文引号之类的正常显示,相当于设置英文字体
\setsansfont{Consolas} %设置英文字体 Monaco, Consolas,  Fantasque Sans Mono
\setmainfont{Times New Roman}

\setCJKmainfont{华文中宋}


\newcommand{\fic}[1]{\begin{figure}[H]
		\center
		\includegraphics[width=0.8\textwidth]{#1}
	\end{figure}}
	
\newcommand{\sizedfic}[2]{\begin{figure}[H]
		\center
		\includegraphics[width=#1\textwidth]{#2}
	\end{figure}}

\newcommand{\codefile}[1]{\lstinputlisting{#1}}

\newcommand{\interval}{\vspace{0.5em}}

% 改变段间隔
\setlength{\parskip}{0.2em}
\linespread{1.1}

\usepackage{lastpage}
\usepackage{fancyhdr}
\pagestyle{fancy}
\lhead{\space \qquad \space}
\chead{Android开发初体验 \qquad}
\rhead{\qquad\thepage/\pageref{LastPage}}
\begin{document}

% \tableofcontents

% \clearpage

\section{第一个用户界面}
	第一个用户界面如下:
	\begin{lstlisting}[language = xml]
	<LinearLayout xmlns:android="http://schemas.android.com/apk/res/android"
	 android:layout_width="match_parent"
	 android:layout_height="match_parent"
	 android:gravity="center"
	 android:orientation="vertical">

	 <TextView
	  android:layout_width="wrap_content"
	  android:layout_height="wrap_content"
	  android:padding="24dp"
	  android:text="@string/question_text"
	 />

	 <LinearLayout
	  android:layout_width="wrap_content"
	  android:layout_height="wrap_content"
	  android:orientation="horizontal"
	  >

	  <Button
	   android:layout_width="wrap_content"
	   android:layout_height="wrap_content"
	   android:text="@string/true_button"
	  />

	  <Button
	   android:layout_width="wrap_content"
	   android:layout_height="wrap_content"
	   android:text="@string/false_button"
	  />

	 </LinearLayout>

	</LinearLayout>
	\end{lstlisting}

\subsection{视图层级结构}
	上述的用户界面的视图层级结构如下:
	\fic{1.png}

	对视图层级结构的解释:
	\begin{itemize}
		\item[1.] 根元素是一个LinearLayout组建,根元素必须指定Android XML资源文件的命名空间属性为http://schemas.android.com/apk/res/android。
		\item[2.] LinearLayout有两个子组件:TextView和LinearLayout。
		\item[3.] 作为子组件的LinearLayout本身还有两个Button子组件。
	\end{itemize}

\subsection{组件属性}
	以下是常见的组件属性:
	\begin{itemize}
		\item[1.] android:layout\_width和android:layout\_height。它们有两个常见的属性值:
		\begin{itemize}
			\item match\_parent,视图与其父视图大小相同。
			\item wrap\_content,视图将根据其展示的内容自动调整大小。
		\end{itemize}

		\item[2.] android:padding,用于告诉组件在决定大小时,除内容本身外,还需要增加额外指定量的空间。

		\item[3.] android:orientation,用于决定组件的子组件是水平放置还是垂直放置,值有vertical和horizon。

		\item[4.] android:text,用于指定组件要显示的文字内容。它的属性值不是字符串值,而是对字符串资源的引用,字符串资源包含在一个独立的名为strings的XML文件中。
	\end{itemize}

\subsection{第一个用户界面的字符串资源}
	strings.xml文件在app/res/values目录下,内容如下:
	\begin{lstlisting}[language=xml]
	<resources>
		<string name="app_name">GeoQuiz</string>
		<string name="question_text">
			Constantionople is the largest city in Turkey.
		</string>
		<string name="true_button">TRUE</string>
		<string name="false_button">FALSE</string>
	</resources>
	\end{lstlisting}

\subsection{从布局XML到视图对象}
	java目录是项目全部java源代码的存放处,其中AppCompatActivity的子类用于把XML元素转换为视图对象:
	\begin{lstlisting}[language = java]
	package geoquiz.android.bignerdranch.com.geoquiz;

	import android.support.v7.app.AppCompatActivity;
	import android.os.Bundle;

	public class QuizActivity extends AppCompatActivity {

		@Override
		protected void onCreate(Bundle savedInstanceState) {
			super.onCreate(savedInstanceState);
			setContentView(R.layout.activity_quiz);
		}
	}
	\end{lstlisting}

\subsection{创建按键动作}
\subsubsection{资源与资源ID}
	Android项目所有资源存放在app/res的子目录下,R.java文件用于记录项目用到的整个布局文件以及各个字符串的资源ID,它存放在app/build/generated/source/r/debug目录下,只有在编译后才会产生。\par

	需要知道的是,android不会为布局文件中的组件生成资源ID,我们需要通过添加android:id属性为组件生成资源ID:
	\begin{lstlisting}[language = xml]
	<LinearLayout...>
	 <TextView
	  android:layout_width="wrap_content"
	  android:layout_height="wrap_content"
	  android:padding="24dp"
	  android:text="@string/question_text"
	 />

	 <LinearLayout
	  android:layout_width="wrap_content"
	  android:layout_height="wrap_content"
	  android:orientation="horizontal"
	 >

	 <Button
	  android:id="@+id/true_button"
	  android:layout_width="wrap_content"
	  android:layout_height="wrap_content"
	  android:text="@string/true_button"
	 />

	 <Button
	  android:id="@+id/false_button"
	  android:layout_width="wrap_content"
	  android:layout_height="wrap_content"
	  android:text="@string/false_button"
	 />

	 </LinearLayout>
	</LinearLayout>
	\end{lstlisting}

\subsubsection{通过资源ID使用组件}
	使用按钮组件有三个步骤:
	\begin{itemize}
		\item[1.] 添加成员变量。
		\item[2.] 通过资源ID引用生成的视图对象。
		\item[3.] 为对象设置监听器,以响应用户的操作。
	\end{itemize}

	代码如下:
	\begin{lstlisting}[language = java]
	import android.widget.Button;

	public class QuizActivity extends AppCompatActivity
	{
		// 添加成员变量
		private Button mTrueButton;
		private Button mFalseButton;

		@Override
		protected void onCreate(Bundle savedInstanceState)
		{
			super.onCreate(savedInstanceState);
			setContentView(R.layout.activity_quiz);

			// 通过资源ID
			mTrueButton = (Button)findViewById(R.id.true_button);
			mFalseButton = (Button)findViewById(R.id.false_button);
		}
	}
	\end{lstlisting}

	监听器用于等待某个特定时间的发生,是实现特定监听器接口的对象实例。Android SDK为各种事件内置开发了很多监听器接口。\par

	Android中用于监听单击事件的监听器接口是View.OnClickListener(),添加监听器操作如下:
	\begin{lstlisting}[language = java]
	public class QuizActivity extends AppCompatActivity
	{
		// 添加成员变量
		private Button mTrueButton;
		private Button mFalseButton;

		@Override
		protected void onCreate(Bundle savedInstanceState)
		{
			super.onCreate(savedInstanceState);
			setContentView(R.layout.activity_quiz);

			// 通过资源ID
			mTrueButton = (Button)findViewById(R.id.true_button);
			mFalseButton = (Button)findViewById(R.id.false_button);
		}

		mTrueButton.setOnClickListener(new View.OnClickListener(){
			@Override
			public void OnClick(View v)
			{
				Toast.makeText(QuizActivity.this, R.string.incorrect_toast, Toast.LENGTH_SHORT).show();
			}
		});

		mFalseButton.setOnClickListener(new View.OnClickListener(){
			@Override
			public void OnClick(View v)
			{
				Toast.makeText(QuizActivity.this, R.string.correct_toast, Toast.LENGTH_SHORT).show();
			}
		})
	}
	\end{lstlisting}

	当然还需要增加toast字符串:
	\begin{lstlisting}[language = xml]
	<resources>
	 <string name="correct_toast">Correct!</string>
	 <string name="incorrect_toast">Incorrect!</string>
	</resources>
	\end{lstlisting}

\end{document}
