% !TeX spellcheck = en_US
%% 字体:方正静蕾简体
%%		 方正粗宋
\documentclass[a4paper,left=2.5cm,right=2.5cm,11pt]{article}

\usepackage[utf8]{inputenc}
\usepackage{fontspec}
\usepackage{cite}
\usepackage{xeCJK}
\usepackage{indentfirst}
\usepackage{titlesec}
\usepackage{longtable}
\usepackage{graphicx}
\usepackage{float}
\usepackage{rotating}
\usepackage{subfigure}
\usepackage{tabu}
\usepackage{amsmath}
\usepackage{setspace}
\usepackage{amsfonts}
\usepackage{appendix}
\usepackage{listings}
\usepackage{xcolor}
\usepackage{geometry}
\setcounter{secnumdepth}{4}
\usepackage{mhchem}
\usepackage{multirow}
\usepackage{extarrows}
\usepackage{hyperref}
\titleformat*{\section}{\LARGE}
\renewcommand\refname{参考文献}
\renewcommand{\abstractname}{\sihao \cjkfzcs 摘{  }要}
%\titleformat{\chapter}{\centering\bfseries\huge\wryh}{}{0.7em}{}{}
%\titleformat{\section}{\LARGE\bf}{\thesection}{1em}{}{}
\titleformat{\subsection}{\Large\bfseries}{\thesubsection}{1em}{}{}
\titleformat{\subsubsection}{\large\bfseries}{\thesubsubsection}{1em}{}{}
\renewcommand{\contentsname}{{\cjkfzcs \centerline{目{  } 录}}}
\setCJKfamilyfont{cjkhwxk}{STXingkai}
\setCJKfamilyfont{cjkfzcs}{STSongti-SC-Regular}
% \setCJKfamilyfont{cjkhwxk}{华文行楷}
% \setCJKfamilyfont{cjkfzcs}{方正粗宋简体}
\newcommand*{\cjkfzcs}{\CJKfamily{cjkfzcs}}
\newcommand*{\cjkhwxk}{\CJKfamily{cjkhwxk}}
\newfontfamily\wryh{Microsoft YaHei}
\newfontfamily\hwzs{STZhongsong}
\newfontfamily\hwst{STSong}
\newfontfamily\hwfs{STFangsong}
\newfontfamily\jljt{MicrosoftYaHei}
\newfontfamily\hwxk{STXingkai}
% \newfontfamily\hwzs{华文中宋}
% \newfontfamily\hwst{华文宋体}
% \newfontfamily\hwfs{华文仿宋}
% \newfontfamily\jljt{方正静蕾简体}
% \newfontfamily\hwxk{华文行楷}
\newcommand{\verylarge}{\fontsize{60pt}{\baselineskip}\selectfont}  
\newcommand{\chuhao}{\fontsize{44.9pt}{\baselineskip}\selectfont}  
\newcommand{\xiaochu}{\fontsize{38.5pt}{\baselineskip}\selectfont}  
\newcommand{\yihao}{\fontsize{27.8pt}{\baselineskip}\selectfont}  
\newcommand{\xiaoyi}{\fontsize{25.7pt}{\baselineskip}\selectfont}  
\newcommand{\erhao}{\fontsize{23.5pt}{\baselineskip}\selectfont}  
\newcommand{\xiaoerhao}{\fontsize{19.3pt}{\baselineskip}\selectfont} 
\newcommand{\sihao}{\fontsize{14pt}{\baselineskip}\selectfont}      % 字号设置  
\newcommand{\xiaosihao}{\fontsize{12pt}{\baselineskip}\selectfont}  % 字号设置  
\newcommand{\wuhao}{\fontsize{10.5pt}{\baselineskip}\selectfont}    % 字号设置  
\newcommand{\xiaowuhao}{\fontsize{9pt}{\baselineskip}\selectfont}   % 字号设置  
\newcommand{\liuhao}{\fontsize{7.875pt}{\baselineskip}\selectfont}  % 字号设置  
\newcommand{\qihao}{\fontsize{5.25pt}{\baselineskip}\selectfont}    % 字号设置 

\usepackage{diagbox}
\usepackage{multirow}
\boldmath
\XeTeXlinebreaklocale "zh"
\XeTeXlinebreakskip = 0pt plus 1pt minus 0.1pt
\definecolor{cred}{rgb}{0.8,0.8,0.8}
\definecolor{cgreen}{rgb}{0,0.3,0}
\definecolor{cpurple}{rgb}{0.5,0,0.35}
\definecolor{cdocblue}{rgb}{0,0,0.3}
\definecolor{cdark}{rgb}{0.95,1.0,1.0}
\lstset{
	language=java,
	numbers=left,
	numberstyle=\tiny\color{black},
	showspaces=false,
	showstringspaces=false,
	basicstyle=\scriptsize,
	keywordstyle=\color{purple},
	commentstyle=\itshape\color{cgreen},
	stringstyle=\color{blue},
	frame=lines,
	% escapeinside=``,
	extendedchars=true, 
	xleftmargin=1em,
	xrightmargin=1em, 
	backgroundcolor=\color{cred},
	aboveskip=1em,
	breaklines=true,
	tabsize=4
} 

\newfontfamily{\consolas}{Consolas}
\newfontfamily{\monaco}{Monaco}
\setmonofont[Mapping={}]{Consolas}	%英文引号之类的正常显示,相当于设置英文字体
\setsansfont{Consolas} %设置英文字体 Monaco, Consolas,  Fantasque Sans Mono
\setmainfont{Times New Roman}

\setCJKmainfont{华文中宋}


\newcommand{\fic}[1]{\begin{figure}[H]
		\center
		\includegraphics[width=0.8\textwidth]{#1}
	\end{figure}}
	
\newcommand{\sizedfic}[2]{\begin{figure}[H]
		\center
		\includegraphics[width=#1\textwidth]{#2}
	\end{figure}}

\newcommand{\codefile}[1]{\lstinputlisting{#1}}

\newcommand{\interval}{\vspace{0.5em}}

% 改变段间隔
\setlength{\parskip}{0.2em}
\linespread{1.1}

\usepackage{lastpage}
\usepackage{fancyhdr}
\pagestyle{fancy}
\lhead{\space \qquad \space}
\chead{Android中常用的API \qquad}
\rhead{\qquad\thepage/\pageref{LastPage}}
\begin{document}

% \tableofcontents

% \clearpage

\section{Activity类}
\subsection{findViewById}
	函数声明如下:
	\begin{lstlisting}
	public view findViewById(int id);
	\end{lstlisting}

	该方法以组件的资源ID作为参数,返回一个视图对象。需要注意的事,赋值前,需要先将返回的View转型。

\subsection{getIntent}
	函数声明如下:
	\begin{lstlisting}
	Intent getIntent (String uri);
	\end{lstlisting}

	Activity.getIntent()方法返回了由startActivity(Intent)方法转发的Intent对象。

\subsection{onActivityResult}
	函数声明如下:
	\begin{lstlisting}
	protected void onActivityResult(int requestCode, int resultCode, Intent data);
	\end{lstlisting}

	当用户从子activity返回到父activity时,ActivityManager将调用onActivityResult()方法,
	这个方法的参数来自于父activity的原始请求代码,以及setResult()方法参数中的结果代码和intent。\par

	使用例子如下:
	\begin{lstlisting}
	public class QuizActivity extends AppCompatActivity{
		private boolean mIsCheater;
		...
		@Override
		protected void onActivityResult(int requestCode, int resultCode, Intent data)
		{
			if(resultCode != Activity.RESULT_OK)
				return;
			
			if(resultCode == REQUEST_CODE_CHEAT)
			{
				if(data == null)
					return;
				mIsCheater = data.getBooleanExtra(EXTRA_ANSWER_SHOWN, false);
			}
		}
	}
	\end{lstlisting}

\subsection{onCreate}
	函数声明如下:
	\begin{lstlisting}
	void onCreate(Bundle);
	\end{lstlisting}

	当Activity子类被调用后,onCreate(Bundle)方法会被调用,用于初始化。\par

	英文介绍如下:\par
	Called when the activity is starting. 
	This is where most initialization should go: calling setContentView(int) to inflate the activity's UI, 
	using findViewById(int) to programmatically interact with widgets in the UI, 
	calling managedQuery(android.net.Uri, String[], String, String[], String) to retrieve cursors for data being displayed, etc.\par

	activity通过覆盖这个方法来准备以下用户界面相关的工作:
	\begin{itemize}
		\item[1.] 实例化组件并将组件放置在屏幕上,调用setContentView()方法。
		\item[2.] 为组件设置监听器以处理用户交互。
		\item[3.] 访问外部模型数据。
	\end{itemize}

\subsection{onSaveInstanceState}
	函数声明如下:
	\begin{lstlisting}
	protected void onSaveInstanceState(Bundle outState);
	\end{lstlisting}

	这个方法通常在onPause()、onStop()和onDestroy()方法之前由系统调用。\par

	onSaveInstanceState()要求activity将自身状态数据保存在Bundle对象中。Bundle是存储字符串键与限定类型值之间映射关系的一种结构。\par

	一般通过覆盖onSaveInstanceState()方法将一些数据保存在bundle中,然后在onCreate()方法中取回这些数据,从而完成activity销毁与新建之间的数据传递。保存activity的数据:
	\begin{lstlisting}
	public class QuizActivity extends AppCompatActivity {
		private static final String KEY_INDEX = "index";
		...
		@Override
		public void onSaveInstanceState(Bundle savedInstanceState)
		{
			super.onSaveInstanceState(savedInstanceState);
			savedInstanceState.putInt(KEY_INDEX, mCurrentIndex);
		}
	}
	\end{lstlisting}

	取出activity的数据:
	\begin{lstlisting}
	public class QuizActivity extends AppCompatActivity {
		@Override
		protected void onCreate(Bundle savedInstanceState)
		{
			...
			if(savedInstanceState != null)
				mCurrentIndex = savedInstanceState.getInt(KEY_INDEX, 0);
		}
	}
	\end{lstlisting}

\subsection{setContentView}
	函数声明如下:
	\begin{lstlisting}
	public void setContentView(int layoutResID);
	\end{lstlisting}

	根据传入的布局资源ID参数,该方法生成指定布局的视图并将其放置在屏幕上。使用例子如下:
	\begin{lstlisting}
	import android.os.Bundle;
	
	public class QuizActivity extends AppCompatActivity
	{
		@Override
		protected void onCreate(Bundle savedInstanceState)
		{
			super.onCreate(savedInstanceState);
			setContentView(R.layout.activity_quiz);
		}
	}
	\end{lstlisting}

\subsection{setResult}
\subsubsection{只返回结果代码}
	函数声明如下:
	\begin{lstlisting}
	public final void setResult(int resultCode);
	\end{lstlisting}

	如果子activity由startActivityForResult()方法启动,那么result code总是会返回给父activity。
	result code的取值有Activity.RESULT\_OK,Activity.RESULT\_CANCELED。
	如果在没调用setResult()方法的情况下,用户单击了后退按钮,父activity则会收到Activity.RESULT\_CANCELED的结果代码。

\subsubsection{返回附带数据}
	函数声明如下:
	\begin{lstlisting}
	public final void setResult(int resultCode, Intent data);
	\end{lstlisting}

	我们这里通过Intent对象附带数据,从而会传给父activity,例子如下:
	\begin{lstlisting}
	public class CheatActivity extends AppCompatActivity{
		private static final String EXTRA_ANSWER_SHOWN = "com.bignerdranch.android.geoquiz.answer_shown";
		...
		@Override
		protected void onCreate(Bundle savedInstanceState){
			...
			mShowAnswer.setOnClickListener(new View.OnClickListener(){
				@Override
				public void onClick(View v)
				{
					...
					Intent data = new Intent();
					data.putExtra(EXTRA_ANSWER_SHOWN, true);
					setResult(RESULT_OK, data);
				}
			});
		}
	}
	\end{lstlisting}

\subsection{startActivity}
	函数声明如下:
	\begin{lstlisting}
	public void startActivity(Intent intent);
	\end{lstlisting}

	当activity调用startActivity()方法时,调用请求发送给了操作系统的ActivityManager,它负责创建Activity实例并调用其onCreate()方法。
	其中startActivity()方法的Intent类型参数用于指示ActivityManager启动哪个Activity。

	使用例子如下:
	\begin{lstlisting}
	mCheatButton = (Button)findViewById(R.id.cheat_button);
	mCheatButton.setOnClickListener(new View.OnClickListener(){
		@Override
		public void onClick(View v)
		{
			Intent i = new Intent(QuizActivity.this, CheatActivity.class);
			startActivity(i);
		}
	});
	\end{lstlisting}

\subsection{startActivityForResult}
	函数声明如下:
	\begin{lstlisting}
	public void startActivityForResult(Intent intent, int requestCode);
	\end{lstlisting}

	第一个参数是Intent对象,用于指示ActivityManager启动哪个Activity,第二个参数是请求代码。
	请求代码将先被发送给子activity,然后再返回给父activity。
	当一个activity启动多个不同类型的子activity,并且需要判断区分消息回馈方时,通常会用到该请求代码。\par

	使用例子如下:
	\begin{lstlisting}
	public class QuizActivity extends AppCompatActivity{
		...
		private static final int REQUEST_CODE_CHEAT = 0;

		@Override
		protected void onCreate(Bundle savedInstanceState)
		{
			...
			mCheatButton.setOnClickListener(new View.OnClickListener(){
				@Override
				public void onClick(View v)
				{
					...
					Intent i = new Intent(QuizActivity.this, CheatActivity.class);
					startActivityForResult(i, REQUEST_CODE_CHEAT);
				}
			});
		}
	}
	\end{lstlisting}

\section{Bundle类}
	Bundle类对象一般用于保存activity基本数据类型的一些数据,从而实现activity销毁与新建之间的数据传递。

\subsection{getInt}
	函数声明如下:
	\begin{lstlisting}
	int getInt(String key, int defaultValue)
	\end{lstlisting}

\subsection{putInt}
	函数声明如下:
	\begin{lstlisting}
	void putInt (String key, int value)
	\end{lstlisting}

\section{Button类}
\subsection{setOnClickListener}
	函数声明如下:
	\begin{lstlisting}
	void setOnClickListener(View.OnClickListener);
	\end{lstlisting}

	传入SetOnClickListener()方法的监听器参数一般是一个匿名内部类,例子如下:
	\begin{lstlisting}
	import android.widget.Button;

	public class Test extends AppCompatActivity
	{
		private Button mTrueButton;

		@Override
		protected void onCreate(Bundle savedInstanceState)
		{
			super.onCreate(savedInstanceState);

			mTrueButton = (Button)findViewById(R.id.true_button);
			mTrueButton.setOnClickListener(new View.OnClickListener(){
				@Override
				public void onClick(View v){}
			});
		}
	}
	\end{lstlisting}

\section{Intent类}
	Intent对象是component用来与操作系统通信的一种媒介工具。component包括service、activity、broadcast receiver和content provider。

\subsection{Intent构造方法}
	函数声明如下:
	\begin{lstlisting}
	public Intent(Context packageContext, Class<?> cls);
	\end{lstlisting}

	传入该方法的Class类型参数告诉ActivityManager应该启用哪个activiry,Context类型参数告诉ActivityManager参数在哪个actitivy中可以找到它。\par

	使用例子如下:
	\begin{lstlisting}
	mCheatButton = (Button)findViewById(R.id.cheat_button);
	mCheatButton.setOnClickListener(new View.OnClickListener(){
		@Override
		public void onClick(View v)
		{
			Intent i = new Intent(QuizActivity.this, CheatActivity.class);
			startActivity(i);
		}
	});
	\end{lstlisting}

\subsection{getBooleanExtra}
	函数声明如下:
	\begin{lstlisting}
	public boolean getBooleanExtra(String name, boolean defaultValue);
	\end{lstlisting}

\subsection{putExtra}
	函数声明如下:
	\begin{lstlisting}
	public Intent putExtra(String name, boolean value);
	\end{lstlisting}

\section{TextView类}
\subsection{setText}
	函数声明如下:
	\begin{lstlisting}
	void setText(int resid);
	\end{lstlisting}

	介绍:Sets the string value of the TextView.

\section{Toast类}
	Toast类的介绍:A toast is a view containing a quick little message for the user. The toast class helps you create and show those.

\subsection{makeText}
	函数声明如下:
	\begin{lstlisting}
	public static Toast makeText(Context context, int redId, int duration);
	\end{lstlisting}

	该方法用于创建一个toast,第一个Context参数是Activity的一个实例,第二个参数是toast要显示的字符串消息的资源ID,Toast类必须借助Context才能找到并使用字符串的资源ID。
	第三个参数通常是两个Toast常量中的一个,用来指定toast消息显示的持续时间。\par

	makeText函数一般和show()方法配合使用:
	\begin{lstlisting}
	mTrueButton.setOnClickListener(new View.OnClickListener(){
		@Override
		public void onClick(View v)
		{
			Toast.makeText(QuizActivity.this, R.string.incorrect_toast, Toast.LENGTH_SHORT).show();
		}
	});
	\end{lstlisting}

\subsection{show}
	函数声明如下:
	\begin{lstlisting}
	void show();
	\end{lstlisting}

	该方法用于在屏幕上显示toast消息。

\end{document}
