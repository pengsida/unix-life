% !TeX spellcheck = en_US
%% 字体:方正静蕾简体
%%		 方正粗宋
\documentclass[a4paper,left=2.5cm,right=2.5cm,11pt]{article}

\usepackage[utf8]{inputenc}
\usepackage{fontspec}
\usepackage{cite}
\usepackage{xeCJK}
\usepackage{indentfirst}
\usepackage{titlesec}
\usepackage{longtable}
\usepackage{graphicx}
\usepackage{float}
\usepackage{rotating}
\usepackage{subfigure}
\usepackage{tabu}
\usepackage{amsmath}
\usepackage{setspace}
\usepackage{amsfonts}
\usepackage{appendix}
\usepackage{listings}
\usepackage{xcolor}
\usepackage{geometry}
\setcounter{secnumdepth}{4}
\usepackage{mhchem}
\usepackage{multirow}
\usepackage{extarrows}
\usepackage{hyperref}
\titleformat*{\section}{\LARGE}
\renewcommand\refname{参考文献}
\renewcommand{\abstractname}{\sihao \cjkfzcs 摘{  }要}
%\titleformat{\chapter}{\centering\bfseries\huge\wryh}{}{0.7em}{}{}
%\titleformat{\section}{\LARGE\bf}{\thesection}{1em}{}{}
\titleformat{\subsection}{\Large\bfseries}{\thesubsection}{1em}{}{}
\titleformat{\subsubsection}{\large\bfseries}{\thesubsubsection}{1em}{}{}
\renewcommand{\contentsname}{{\cjkfzcs \centerline{目{  } 录}}}
\setCJKfamilyfont{cjkhwxk}{STXingkai}
\setCJKfamilyfont{cjkfzcs}{STSongti-SC-Regular}
% \setCJKfamilyfont{cjkhwxk}{华文行楷}
% \setCJKfamilyfont{cjkfzcs}{方正粗宋简体}
\newcommand*{\cjkfzcs}{\CJKfamily{cjkfzcs}}
\newcommand*{\cjkhwxk}{\CJKfamily{cjkhwxk}}
\newfontfamily\wryh{Microsoft YaHei}
\newfontfamily\hwzs{STZhongsong}
\newfontfamily\hwst{STSong}
\newfontfamily\hwfs{STFangsong}
\newfontfamily\jljt{MicrosoftYaHei}
\newfontfamily\hwxk{STXingkai}
% \newfontfamily\hwzs{华文中宋}
% \newfontfamily\hwst{华文宋体}
% \newfontfamily\hwfs{华文仿宋}
% \newfontfamily\jljt{方正静蕾简体}
% \newfontfamily\hwxk{华文行楷}
\newcommand{\verylarge}{\fontsize{60pt}{\baselineskip}\selectfont}  
\newcommand{\chuhao}{\fontsize{44.9pt}{\baselineskip}\selectfont}  
\newcommand{\xiaochu}{\fontsize{38.5pt}{\baselineskip}\selectfont}  
\newcommand{\yihao}{\fontsize{27.8pt}{\baselineskip}\selectfont}  
\newcommand{\xiaoyi}{\fontsize{25.7pt}{\baselineskip}\selectfont}  
\newcommand{\erhao}{\fontsize{23.5pt}{\baselineskip}\selectfont}  
\newcommand{\xiaoerhao}{\fontsize{19.3pt}{\baselineskip}\selectfont} 
\newcommand{\sihao}{\fontsize{14pt}{\baselineskip}\selectfont}      % 字号设置  
\newcommand{\xiaosihao}{\fontsize{12pt}{\baselineskip}\selectfont}  % 字号设置  
\newcommand{\wuhao}{\fontsize{10.5pt}{\baselineskip}\selectfont}    % 字号设置  
\newcommand{\xiaowuhao}{\fontsize{9pt}{\baselineskip}\selectfont}   % 字号设置  
\newcommand{\liuhao}{\fontsize{7.875pt}{\baselineskip}\selectfont}  % 字号设置  
\newcommand{\qihao}{\fontsize{5.25pt}{\baselineskip}\selectfont}    % 字号设置 

\usepackage{diagbox}
\usepackage{multirow}
\boldmath
\XeTeXlinebreaklocale "zh"
\XeTeXlinebreakskip = 0pt plus 1pt minus 0.1pt
\definecolor{cred}{rgb}{0.8,0.8,0.8}
\definecolor{cgreen}{rgb}{0,0.3,0}
\definecolor{cpurple}{rgb}{0.5,0,0.35}
\definecolor{cdocblue}{rgb}{0,0,0.3}
\definecolor{cdark}{rgb}{0.95,1.0,1.0}
\lstset{
	language=java,
	numbers=left,
	numberstyle=\tiny\color{black},
	showspaces=false,
	showstringspaces=false,
	basicstyle=\scriptsize,
	keywordstyle=\color{purple},
	commentstyle=\itshape\color{cgreen},
	stringstyle=\color{blue},
	frame=lines,
	% escapeinside=``,
	extendedchars=true, 
	xleftmargin=1em,
	xrightmargin=1em, 
	backgroundcolor=\color{cred},
	aboveskip=1em,
	breaklines=true,
	tabsize=4
} 

\newfontfamily{\consolas}{Consolas}
\newfontfamily{\monaco}{Monaco}
\setmonofont[Mapping={}]{Consolas}	%英文引号之类的正常显示,相当于设置英文字体
\setsansfont{Consolas} %设置英文字体 Monaco, Consolas,  Fantasque Sans Mono
\setmainfont{Times New Roman}

\setCJKmainfont{华文中宋}


\newcommand{\fic}[1]{\begin{figure}[H]
		\center
		\includegraphics[width=0.8\textwidth]{#1}
	\end{figure}}
	
\newcommand{\sizedfic}[2]{\begin{figure}[H]
		\center
		\includegraphics[width=#1\textwidth]{#2}
	\end{figure}}

\newcommand{\codefile}[1]{\lstinputlisting{#1}}

\newcommand{\interval}{\vspace{0.5em}}

% 改变段间隔
\setlength{\parskip}{0.2em}
\linespread{1.1}

\usepackage{lastpage}
\usepackage{fancyhdr}
\pagestyle{fancy}
\lhead{\space \qquad \space}
\chead{JavaMail的学习 \qquad}
\rhead{\qquad\thepage/\pageref{LastPage}}
\begin{document}

% \tableofcontents

% \clearpage

\section{参考网站}
	\url{http://www.yiibai.com/javamail_api/}

\section{Address类}
	Javax.mail.Address、Javax.mail.internet.InternetAddress:一旦您创建了Session 和Message,
	并将内容填入消息后,就可以用 Address 确定信件地址了。和 Message 一样,Address也是个抽象类。
	您用的是Javax.mail.internet.InternetAddress 类。

\section{Authenticator}
	与 Java.net 类一样,JavaMailAPI 也可以利用 Authenticator 通过用户名和密码访问受保护的资源。
	对于JavaMail API来说,这些资源就是邮件服务器。
	JavaMail Authenticator在Javax.mail包中,而且它和Java.net中同名的类Authenticator不同。
	两者并不共享同一个Authenticator,因为JavaMail API 用于Java 1.1,它没有Java.net 类别。\par

	要使用 Authenticator,先创建一个抽象类的子类,并从getPasswordAuthentication() 方法中返回PasswordAuthentication 实例。
	创建完成后,您必需向session 注册Authenticator。
	然后,在需要认证的时候,就会通知Authenticator。
	您可以弹出窗口,也可以从配置文件中(虽然没有加密是不安全的)读取用户名和密码,
	将它们作为PasswordAuthentication 对象返回给调用程序。

\section{Folder类}
	该类用于描述邮件的分级组织,如收件箱、草稿箱。\par

\subsection{打开分级组织}
	\begin{lstlisting}
	public abstract void open(int mode);
	\end{lstlisting}

\subsection{关闭分级组织}
	\begin{lstlisting}
	public abstract void close(boolean expunge);
	\end{lstlisting}

\subsection{获得分级组织中的邮件}
	\begin{lstlisting}
	Message[] getMessages();
	Message[] getMessages(int[] msgnums);
	Message[] getMessages(int start, int end);
	\end{lstlisting}

\subsection{获得分级组织中的新邮件数量}
	\begin{lstlisting}
	int getNewMessageCount();
	\end{lstlisting}

\subsection{获得分级组织中未读邮件的数量}
	\begin{lstlisting}
	int getUnreadMessageCount();
	\end{lstlisting}

\section{Message类}
	 javax.mail.Message 类是创建和解析邮件的核心API,它的实例对象代表一封电子邮件。

	 发送邮件的步骤:
	 \begin{itemize}
		 \item[1.] 首先使用创建邮件的API。
		 \item[2.] 将要创建邮件的数据封装到Message对象中。
		 \item[3.] 最后把这个对象传递给邮件发送API发送出去。  
	 \end{itemize}

	 接收邮件的步骤:
	 \begin{itemize}
		 \item[1.] 邮件接收API把接收到的邮件数据封装在Message类的实例对象中。
		 \item[2.] 客户端程序再使用邮件解析API。
		 \item[3.] 从这个对象中解析出接收到的邮件数据。 
	 \end{itemize}

	 API查询网站:\par
	 \url{file:///Library/Java/JavaVirtualMachines/jdk1.8.0_112.jdk/Contents/Home/jre/lib/ext/javamail-1.4.5/docs/javadocs/javax/mail/Message.html}

\subsection{MIME解析}
	如果在邮件中要添加附件,必须定义multipart/mixed段;
	如果存在内嵌资源,至少要定义multipart/related段;
	如果纯文本与超文本共存,至少要定义multipart/alternative段。
	什么是“至少”?举个例子说,如果只有纯文本与超文本正文,
	那么在邮件头中将类型扩大化,定义为multipart/related,
	甚至multipart/mixed,都是允许的。\par

	multipart诸类型的共同特征是,
	在段头指定“boundary”参数字符串,段体内的每个子段以此串定界。
	所有的子段都以“--”+boundary行开始,父段则以“--”+boundary+“--”行结束。
	段与段之间也以空行分隔。
	在邮件体是multipart类型的情况下,
	邮件体的开始部分(第一个“--”+boundary行之前)可以有一些附加的文本行,
	相当于注释,解码时应忽略。
	段间也可以有一些附加的文本行,不会显示出来,如果有兴趣,不妨验证一下。

\subsection{常用API}
\subsubsection{设置sender}
	\begin{lstlisting}
	public abstract void setFrom(Address address);
	\end{lstlisting}

\subsubsection{设置recipient}
	\begin{lstlisting}
	public abstract void setRecipient(Address address);
	\end{lstlisting}

\subsection{获得发件人}
	\begin{lstlisting}
	abstract Address[] getFrom();
	\end{lstlisting}

\subsection{获得收件人}
	\begin{lstlisting}
	abstract Address[] getRecipients();
	\end{lstlisting}

\subsection{获得邮件的主题}
	\begin{lstlisting}
	abstrct String getSubject();
	\end{lstlisting}

\subsection{获得发送日期}
	\begin{lstlisting}
	abstract Date getSentDate();
	\end{lstlisting}

\subsection{获得邮件的ID}
	邮件的ID表示一封邮件,函数如下:
	\begin{lstlisting}
	String getMessageID();
	\end{lstlisting}

\subsection{获得邮件的内容}
	\begin{lstlisting}
	Object getContent();
	\end{lstlisting}

\section{Properties类}
	由于JavaMail需要和邮件服务器进行通信,
	这就要求程序提供许多诸如服务器地址、端口、用户名、密码等信息,
	JavaMail通过Properties对象封装这些属性的信息。\par

	针对不同的的邮件协议,
	JavaMail规定了服务提供者必须支持一系列属性,
	下表是针对SMTP协议的一些常见属性(属性值都以String类型进行设置,属性类型栏仅表示属性是如何被解析的):\par

	\begin{itemize}
		\item[1.] mail.smtp.host,类型为String,SMTP服务器地址。
		\item[2.] mail.smtp.port,类型为int,SMTP服务器端口号,默认为25
		\item[3.] mail.smtp.auth,类型为boolean,STMP服务器是否需要用户认证
		\item[4.] mail.smtp.user,类型为String,SMTP默认的登陆用户名。
		\item[5.] mail.smtp.from,类型为String,默认的邮件发送源地址。
		\item[6.] mail.smtp.timeout,类型为int,I/O连接超时时间,单位为毫秒,默认为永不超时。
	\end{itemize}

	使用的例子:
	\begin{lstlisting}
	public class MailSenderInfo
	{
		private String mailServerHost;
		private String mailServerPort = "25";
		private String fromAddress;
		private String toAddress;

		private String userName;
		private String password;
		private boolean validate = true;

		private String subject;
		private String content;
		private String[] attachFileNames;

		public Properties getProperties()
		{
			Properties p = new Properties();
			p.put("mail.smtp.host", this.mailServerHost);
			p.put("mail.smtp.port", this.mailServerPort);
			p.put("mail.smtp.auth", validate ? "true" : "false");
			return p;
		}
		...
	}

	public class MailSender
	{
		public static boolean sendTextMail(MailSenderInfo mailInfo)
		{
			MyAuthenticator authenticator = null;
			if(mailInfo.isValidate())
				authenticator = new MyAuthenticator(mailInfo.getUserName(), mailInfo.getPassword());

			Session sendMailSession = Session.getInstance(mailInfo.getProperties(), authenticator);

			...

			return false;
		}
	}
	\end{lstlisting}

	可以查询SMTP、POP3和IMAP的属性的网站:\par
	\url{file:///Library/Java/JavaVirtualMachines/jdk1.8.0_112.jdk/Contents/Home/jre/lib/ext/javamail-1.4.5/docs/javadocs/com/sun/mail/smtp/package-summary.html}\par
	\url{file:///Library/Java/JavaVirtualMachines/jdk1.8.0_112.jdk/Contents/Home/jre/lib/ext/javamail-1.4.5/docs/javadocs/com/sun/mail/pop3/package-summary.html}\par
	\url{file:///Library/Java/JavaVirtualMachines/jdk1.8.0_112.jdk/Contents/Home/jre/lib/ext/javamail-1.4.5/docs/javadocs/com/sun/mail/imap/package-summary.html}

\section{Session类}
	javax.mail.Session 类用于定义整个应用程序所需的环境信息,以及收集客户端与邮件服务器建立网络连接的会话信息,如邮件服务器的主机名、端口号、采用的邮件发送和接收协议等。
	Session对象根据这些会话信息构建邮件收发的Store和Transport对象,以及为客户端创建Message对象提供信息支持。

\section{Store类}
	javax.mail.Store 类是接收邮件的核心API类,它的实例对象代表实现某个邮件接收协议的邮件接收对象,例如POP3和IMAP协议。
	客户端程序接收邮件时,只需要使用邮件接收API得到Store对象,然后调用Store对象的接收方法,就可以从指定的 POP3服务器中获得邮件数据,并把这些邮件数据封装到表示邮件的Message对象中。

\subsection{打开Folder}
	\begin{lstlisting}
	public abstract void getFolder();
	\end{lstlisting}

\section{Transport类}
	javax.mail.Transport 类是发送邮件的核心API类,它的实例对象,代表实现了某个邮件发送协议的邮件发送对象,例如SMTP 协议。
	客户端程序创建好Message对象后,只需要使用邮件发送API得到 Transport 对象,然后把 Message 对象传递给 Transport 对象,并调用它的发送方法,就可以把邮件发送给指定的SMTP服务器。

\section{参考网站}
	\url{http://mojijs.com/2016/09/219349/index.html}

\end{document}
