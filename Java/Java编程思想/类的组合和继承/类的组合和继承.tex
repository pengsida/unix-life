% !TeX spellcheck = en_US
%% 字体:方正静蕾简体
%%		 方正粗宋
\documentclass[a4paper,left=2.5cm,right=2.5cm,11pt]{article}

\usepackage[utf8]{inputenc}
\usepackage{fontspec}
\usepackage{cite}
\usepackage{xeCJK}
\usepackage{indentfirst}
\usepackage{titlesec}
\usepackage{longtable}
\usepackage{graphicx}
\usepackage{float}
\usepackage{rotating}
\usepackage{subfigure}
\usepackage{tabu}
\usepackage{amsmath}
\usepackage{setspace}
\usepackage{amsfonts}
\usepackage{appendix}
\usepackage{listings}
\usepackage{xcolor}
\usepackage{geometry}
\setcounter{secnumdepth}{4}
\usepackage{mhchem}
\usepackage{multirow}
\usepackage{extarrows}
\usepackage{hyperref}
\titleformat*{\section}{\LARGE}
\renewcommand\refname{参考文献}
\renewcommand{\abstractname}{\sihao \cjkfzcs 摘{  }要}
%\titleformat{\chapter}{\centering\bfseries\huge\wryh}{}{0.7em}{}{}
%\titleformat{\section}{\LARGE\bf}{\thesection}{1em}{}{}
\titleformat{\subsection}{\Large\bfseries}{\thesubsection}{1em}{}{}
\titleformat{\subsubsection}{\large\bfseries}{\thesubsubsection}{1em}{}{}
\renewcommand{\contentsname}{{\cjkfzcs \centerline{目{  } 录}}}
\setCJKfamilyfont{cjkhwxk}{STXingkai}
\setCJKfamilyfont{cjkfzcs}{STSongti-SC-Regular}
% \setCJKfamilyfont{cjkhwxk}{华文行楷}
% \setCJKfamilyfont{cjkfzcs}{方正粗宋简体}
\newcommand*{\cjkfzcs}{\CJKfamily{cjkfzcs}}
\newcommand*{\cjkhwxk}{\CJKfamily{cjkhwxk}}
\newfontfamily\wryh{Microsoft YaHei}
\newfontfamily\hwzs{STZhongsong}
\newfontfamily\hwst{STSong}
\newfontfamily\hwfs{STFangsong}
\newfontfamily\jljt{MicrosoftYaHei}
\newfontfamily\hwxk{STXingkai}
% \newfontfamily\hwzs{华文中宋}
% \newfontfamily\hwst{华文宋体}
% \newfontfamily\hwfs{华文仿宋}
% \newfontfamily\jljt{方正静蕾简体}
% \newfontfamily\hwxk{华文行楷}
\newcommand{\verylarge}{\fontsize{60pt}{\baselineskip}\selectfont}  
\newcommand{\chuhao}{\fontsize{44.9pt}{\baselineskip}\selectfont}  
\newcommand{\xiaochu}{\fontsize{38.5pt}{\baselineskip}\selectfont}  
\newcommand{\yihao}{\fontsize{27.8pt}{\baselineskip}\selectfont}  
\newcommand{\xiaoyi}{\fontsize{25.7pt}{\baselineskip}\selectfont}  
\newcommand{\erhao}{\fontsize{23.5pt}{\baselineskip}\selectfont}  
\newcommand{\xiaoerhao}{\fontsize{19.3pt}{\baselineskip}\selectfont} 
\newcommand{\sihao}{\fontsize{14pt}{\baselineskip}\selectfont}      % 字号设置  
\newcommand{\xiaosihao}{\fontsize{12pt}{\baselineskip}\selectfont}  % 字号设置  
\newcommand{\wuhao}{\fontsize{10.5pt}{\baselineskip}\selectfont}    % 字号设置  
\newcommand{\xiaowuhao}{\fontsize{9pt}{\baselineskip}\selectfont}   % 字号设置  
\newcommand{\liuhao}{\fontsize{7.875pt}{\baselineskip}\selectfont}  % 字号设置  
\newcommand{\qihao}{\fontsize{5.25pt}{\baselineskip}\selectfont}    % 字号设置 

\usepackage{diagbox}
\usepackage{multirow}
\boldmath
\XeTeXlinebreaklocale "zh"
\XeTeXlinebreakskip = 0pt plus 1pt minus 0.1pt
\definecolor{cred}{rgb}{0.8,0.8,0.8}
\definecolor{cgreen}{rgb}{0,0.3,0}
\definecolor{cpurple}{rgb}{0.5,0,0.35}
\definecolor{cdocblue}{rgb}{0,0,0.3}
\definecolor{cdark}{rgb}{0.95,1.0,1.0}
\lstset{
	language=[x86masm]Assembler,
	numbers=left,
	numberstyle=\tiny\color{black},
	showspaces=false,
	showstringspaces=false,
	basicstyle=\scriptsize,
	keywordstyle=\color{purple},
	commentstyle=\itshape\color{cgreen},
	stringstyle=\color{blue},
	frame=lines,
	% escapeinside=``,
	extendedchars=true, 
	xleftmargin=1em,
	xrightmargin=1em, 
	backgroundcolor=\color{cred},
	aboveskip=1em,
	breaklines=true,
	tabsize=4
} 

\newfontfamily{\consolas}{Consolas}
\newfontfamily{\monaco}{Monaco}
\setmonofont[Mapping={}]{Consolas}	%英文引号之类的正常显示,相当于设置英文字体
\setsansfont{Consolas} %设置英文字体 Monaco, Consolas,  Fantasque Sans Mono
\setmainfont{Times New Roman}

\setCJKmainfont{华文中宋}


\newcommand{\fic}[1]{\begin{figure}[H]
		\center
		\includegraphics[width=0.8\textwidth]{#1}
	\end{figure}}
	
\newcommand{\sizedfic}[2]{\begin{figure}[H]
		\center
		\includegraphics[width=#1\textwidth]{#2}
	\end{figure}}

\newcommand{\codefile}[1]{\lstinputlisting{#1}}

% 改变段间隔
\setlength{\parskip}{0.2em}
\linespread{1.1}

\usepackage{lastpage}
\usepackage{fancyhdr}
\pagestyle{fancy}
\lhead{\space \qquad \space}
\chead{复用类 \qquad}
\rhead{\qquad\thepage/\pageref{LastPage}}
\begin{document}

\tableofcontents

\clearpage

\section{复用类}
	Java中复用类的两种方式:
	\begin{itemize}
		\item 在新的类中产生现有类的对象,这种方法称为组合。
		\item 按照现有类的类型来创建新类,这种方法称为继承。
	\end{itemize}

\subsection{类的组合}
	组合技术很直观,只要将对象引用置于新类中即可。例子如下:
	\begin{lstlisting}[language = Java]
	class WaterSource
	{
		private String s;
		WaterSource()
		{
			System.out.println("WaterSource");
			s = "Constructed";
		}
	}
	\end{lstlisting}

\subsection{类的继承}
	Java中继承的语法和C++类似,不过Java中使用关键字extends声明。
	如果继承基类,新类就会得到基类中所有非私有的域和成员函数。例子如下:
	\begin{lstlisting}[language = Java]
	class Cleanser
	{
		private String s = "Cleanser";
		public void append(String a)
		{
			s += a;
		}
		public static void main(String[] args)
		{
			Cleanser x = new Cleanser();
			x.append(" hello world");
		}
	}

	public class Detergent extends Cleanser
	{
		public static void main(String[] args)
		{
			Detergent x = new Detergent();
			x.append(" hello world");
			Cleanser.main(args);
		}
	}
	\end{lstlisting}

\subsubsection{基类的初始化}
	如果没有特别声明,将调用基类默认的构造器或者无参数构造器。
	如果想调用一个带参数的基类构造器,就必须使用super显式地调用基类构造器。例子如下:
	\begin{lstlisting}[language = Java]
	class Game
	{
		Game(int i)
		{
			System.out.println("Hello World");
		}
	}

	public class Chess extends Game
	{
		Chess()
		{
			super(1);
			System.out.println("Chess constructor");
		}
		public static void main(String[] args)
		{
			Chess c = new Chess();
		}
	}
	\end{lstlisting}

\subsubsection{名称屏蔽}
	与C++不同的是,Java中导出类如果重载基类中的函数,并不会屏蔽其在基类中该函数的任何版本。例子如下:
	\begin{lstlisting}[language = Java]
	class Homer
	{
		char doh(char c)
		{
			return c;
		}
		float doh(float c)
		{
			return c;
		}
	}

	class Bart extends Homer
	{
		String doh(String s)
		{
			return s;
		}
	}
	\end{lstlisting}

	需要注意的是,因为这个语法特点,Java中其实是没有名称屏蔽的。那么当我们要覆写基类中的一个函数时,很可能将其重载而非覆写。
	为了防止这个错误的发生,Java提供了@Override注解相应的函数。如果这个函数是重载而非覆写时,编译器就会产生错误:
	\begin{lstlisting}[language = Java]
	class Lisa extends Homer
	{
		@Override
		String doh(String s)
		{
			return s; // 将产生错误
		}
	}
	\end{lstlisting}

\subsubsection{向上转型}
	继承技术最重要的不是为新的类提供函数,而是用于表现新类和基类之间的关系。新类是现有类的一种类型。例子如下:
	\begin{lstlisting}[language = Java]
	class Instrument
	{
		public void play(){}
		static void tune(Instrument i)
		{
			i.play();
		}
	}

	public class Wind extends Instrument
	{
		public static void main(String[] args)
		{
			Wind flute = new Wind();
			Instrument.tune(flute);
			// tune函数接受的是Instrument对象
			// 这里它也可以接受Wind对象
			// 因为Wind是Instrument的一种类型
		}
	}
	\end{lstlisting}

	将导出类引用转换为基类引用的动作称为向上转型。在实现上看,导出类是基类的一个超集。
	在向上转型的过程中,导出类引用转换为基类引用,并且只保留基类拥有的方法。\par
	导出类无法继承private函数。即使在导出类中以相同的名称声明一个函数,也不会覆盖基类中相应的private函数,而是生成了一个新的函数。
	当向上转型时,这个函数将会被丢弃。

\subsubsection{继承技术的用途}
	相对于组合技术,继承技术不常用。只有需要从新类向基类进行向上转型,继承才是必要的。

\subsection{final关键字}
	final关键字可以修饰数据、函数和类。
\subsubsection{final数据}
	Java中使用final告知一块数据是恒定不变的,相当于C中的const关键字。
	需要知道的是,Java中常量必须是基本数据类型。\par
	一个既是static又是final的域只占据一段不能改变的存储空间。\par
	当用final修饰对象引用时,这个引用将恒定不变。也就是说,引用一旦被初始化指向一个对象,就无法再把它改为指向另一个对象,而被引用的对象本身是可以被修改的。
	Java没有提供使任何对象恒定不变的途径。\par
	Java允许生成空白final。也就是这个域被final修饰但又没有赋初值。final域在使用前必须被初始化。\par
	在函数参数列表中将参数指明为final,那么在函数中就无法修改参数引用所指向的对象。
\subsubsection{final函数}
	使用final函数的原因如下:
	\begin{itemize}
		\item 将函数锁定。以防任何继承类修改它的实现。
		\item 追求效率。当一个函数指明为final,编译器就将该函数的所有调用都转为内嵌调用。这和C++的inline关键字的作用一样。
	\end{itemize}

	类中private方法都隐式地指定为final。

\subsubsection{final类}
	当将某个类的整体定义为final,那么这个类就无法被继承。final类中的域不一定是final的。


\end{document}
