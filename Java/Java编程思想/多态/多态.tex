% !TeX spellcheck = en_US
%% 字体:方正静蕾简体
%%		 方正粗宋
\documentclass[a4paper,left=2.5cm,right=2.5cm,11pt]{article}

\usepackage[utf8]{inputenc}
\usepackage{fontspec}
\usepackage{cite}
\usepackage{xeCJK}
\usepackage{indentfirst}
\usepackage{titlesec}
\usepackage{longtable}
\usepackage{graphicx}
\usepackage{float}
\usepackage{rotating}
\usepackage{subfigure}
\usepackage{tabu}
\usepackage{amsmath}
\usepackage{setspace}
\usepackage{amsfonts}
\usepackage{appendix}
\usepackage{listings}
\usepackage{xcolor}
\usepackage{geometry}
\setcounter{secnumdepth}{4}
\usepackage{mhchem}
\usepackage{multirow}
\usepackage{extarrows}
\usepackage{hyperref}
\titleformat*{\section}{\LARGE}
\renewcommand\refname{参考文献}
\renewcommand{\abstractname}{\sihao \cjkfzcs 摘{  }要}
%\titleformat{\chapter}{\centering\bfseries\huge\wryh}{}{0.7em}{}{}
%\titleformat{\section}{\LARGE\bf}{\thesection}{1em}{}{}
\titleformat{\subsection}{\Large\bfseries}{\thesubsection}{1em}{}{}
\titleformat{\subsubsection}{\large\bfseries}{\thesubsubsection}{1em}{}{}
\renewcommand{\contentsname}{{\cjkfzcs \centerline{目{  } 录}}}
\setCJKfamilyfont{cjkhwxk}{STXingkai}
\setCJKfamilyfont{cjkfzcs}{STSongti-SC-Regular}
% \setCJKfamilyfont{cjkhwxk}{华文行楷}
% \setCJKfamilyfont{cjkfzcs}{方正粗宋简体}
\newcommand*{\cjkfzcs}{\CJKfamily{cjkfzcs}}
\newcommand*{\cjkhwxk}{\CJKfamily{cjkhwxk}}
\newfontfamily\wryh{Microsoft YaHei}
\newfontfamily\hwzs{STZhongsong}
\newfontfamily\hwst{STSong}
\newfontfamily\hwfs{STFangsong}
\newfontfamily\jljt{MicrosoftYaHei}
\newfontfamily\hwxk{STXingkai}
% \newfontfamily\hwzs{华文中宋}
% \newfontfamily\hwst{华文宋体}
% \newfontfamily\hwfs{华文仿宋}
% \newfontfamily\jljt{方正静蕾简体}
% \newfontfamily\hwxk{华文行楷}
\newcommand{\verylarge}{\fontsize{60pt}{\baselineskip}\selectfont}  
\newcommand{\chuhao}{\fontsize{44.9pt}{\baselineskip}\selectfont}  
\newcommand{\xiaochu}{\fontsize{38.5pt}{\baselineskip}\selectfont}  
\newcommand{\yihao}{\fontsize{27.8pt}{\baselineskip}\selectfont}  
\newcommand{\xiaoyi}{\fontsize{25.7pt}{\baselineskip}\selectfont}  
\newcommand{\erhao}{\fontsize{23.5pt}{\baselineskip}\selectfont}  
\newcommand{\xiaoerhao}{\fontsize{19.3pt}{\baselineskip}\selectfont} 
\newcommand{\sihao}{\fontsize{14pt}{\baselineskip}\selectfont}      % 字号设置  
\newcommand{\xiaosihao}{\fontsize{12pt}{\baselineskip}\selectfont}  % 字号设置  
\newcommand{\wuhao}{\fontsize{10.5pt}{\baselineskip}\selectfont}    % 字号设置  
\newcommand{\xiaowuhao}{\fontsize{9pt}{\baselineskip}\selectfont}   % 字号设置  
\newcommand{\liuhao}{\fontsize{7.875pt}{\baselineskip}\selectfont}  % 字号设置  
\newcommand{\qihao}{\fontsize{5.25pt}{\baselineskip}\selectfont}    % 字号设置 

\usepackage{diagbox}
\usepackage{multirow}
\boldmath
\XeTeXlinebreaklocale "zh"
\XeTeXlinebreakskip = 0pt plus 1pt minus 0.1pt
\definecolor{cred}{rgb}{0.8,0.8,0.8}
\definecolor{cgreen}{rgb}{0,0.3,0}
\definecolor{cpurple}{rgb}{0.5,0,0.35}
\definecolor{cdocblue}{rgb}{0,0,0.3}
\definecolor{cdark}{rgb}{0.95,1.0,1.0}
\lstset{
	language=[x86masm]Assembler,
	numbers=left,
	numberstyle=\tiny\color{black},
	showspaces=false,
	showstringspaces=false,
	basicstyle=\scriptsize,
	keywordstyle=\color{purple},
	commentstyle=\itshape\color{cgreen},
	stringstyle=\color{blue},
	frame=lines,
	% escapeinside=``,
	extendedchars=true, 
	xleftmargin=1em,
	xrightmargin=1em, 
	backgroundcolor=\color{cred},
	aboveskip=1em,
	breaklines=true,
	tabsize=4
} 

\newfontfamily{\consolas}{Consolas}
\newfontfamily{\monaco}{Monaco}
\setmonofont[Mapping={}]{Consolas}	%英文引号之类的正常显示,相当于设置英文字体
\setsansfont{Consolas} %设置英文字体 Monaco, Consolas,  Fantasque Sans Mono
\setmainfont{Times New Roman}

\setCJKmainfont{华文中宋}


\newcommand{\fic}[1]{\begin{figure}[H]
		\center
		\includegraphics[width=0.8\textwidth]{#1}
	\end{figure}}
	
\newcommand{\sizedfic}[2]{\begin{figure}[H]
		\center
		\includegraphics[width=#1\textwidth]{#2}
	\end{figure}}

\newcommand{\codefile}[1]{\lstinputlisting{#1}}

% 改变段间隔
\setlength{\parskip}{0.2em}
\linespread{1.1}

\usepackage{lastpage}
\usepackage{fancyhdr}
\pagestyle{fancy}
\lhead{\space \qquad \space}
\chead{多态 \qquad}
\rhead{\qquad\thepage/\pageref{LastPage}}
\begin{document}

\tableofcontents

\clearpage

\section{多态}
	多态又称为动态绑定,和C++的多态类似。在讨论多态之前,先感受一下多态的特性。例子如下:
	\begin{lstlisting}[language = Java]
	class Instrument
	{
		public void play()
		{
			System.out.println("Instrument.play()");
		}
	}

	class Wind extends Instrument
	{
		public void play()
		{
			System.out.println("Wind.play()");
		}
	}

	public class Music
	{
		public static void tune(Instrument i)
		{
			i.play();
		}
		public static void main(String[] args)
		{
			Wind flute = new Wind();
			// tune接受Instrument类型
			// 将Wind转为Instrument类型
			// 输出的是:Wind.play()
			tune(flute); 
		}
	}
	\end{lstlisting}

	从这个例子可以看出一个多态的现象:虽然tune函数接受一个Instrument引用,但是它知道这个Instrument引用指向的是Wind对象。
	正是动态绑定实现了这项特性。

\subsection{绑定}
	将一个函数调用和一个函数主体关联起来称为绑定。绑定有两种类型,如下:
	\begin{itemize}
		\item 前期绑定。在程序执行前就将一个函数调用和一个函数主体关联起来。
		\item 后期绑定,又称为动态绑定。在程序运行时根据对象的类型进行绑定。
	\end{itemize}

	Java中除了static方法和final方法,其他所有方法都是后期绑定的。

\subsection{多态的缺陷}
\subsubsection{private函数无法动态绑定}
	程序不能对private函数进行动态绑定。这是因为private函数是final函数,而且导出类无法覆盖基类中的private函数。例子如下:
	\begin{lstlisting}[language = Java]
	class PrivateOvrride
	{
		private void f()
		{
			System.out.println("private f()");
		}

		public static void main(String[] args)
		{
			PrivateOvrride po = new Derived();
			po.f(); // 不会指向Derived类中的f(),而是指向PrivateOvrride类中的f()
		}
	}

	public class Derived extends PrivateOvrride
	{
		public void f()
		{
			System.out.println("public f()");
		}
	}
	\end{lstlisting}

	为了避免造成混乱的代码,导出类中的函数名不要和基类中的private函数名相同。

\subsubsection{域无法动态绑定}
	和C++一样,Java中域是无法动态绑定的。也就是说,程序无法根据对象的类型选择相应的域。例子如下:
	\begin{lstlisting}[language = Java]
	class Super
	{
		public int field = 0;
	}

	Sub extends Super
	{
		public int field = 1;
	}

	public class FieldAccess
	{
		public static void main(String[] args)
		{
			Super sup = new Sub();
			System.out.println(sup.field); // 输出0
			Sub sub = new Sub();
			System.out.println(sub.field); // 输出1
		}
	}
	\end{lstlisting}

	为了避免造成混乱的代码,不要把基类中的域和导出类的域赋予相同的名字。

\subsection{协变返回类型}
	协变返回类型表明,子类覆写基类方法时,返回的类型可以是基类方法返回类型的子类。

	\begin{lstlisting}[language = Java]
	class Grain
	{
		public String toString()
		{
			return "Grain";
		}
	}

	class Wheat extends Grain
	{
		public String toString()
		{
			return "Wheat";
		}
	}

	class Mill
	{
		Grain process()
		{
			return new Grain();
		}
	}

	class WheatMill extends Mill
	{
		Wheat process()
		{
			return new Wheat();
		}
	}

	public class CovariantReturn
	{
		public static void main(String[] args)
		{
			Mill m = new Mill();
			Grain g = m.process();
			System.out.println(g); // 输出Grain
			m = new WheatMill();
			g = m.process();
			System.out.println(g); // 输出Wheat
		}
	}
	\end{lstlisting}

	

\end{document}
