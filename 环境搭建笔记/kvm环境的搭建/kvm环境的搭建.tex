% !TeX spellcheck = en_US
%% 字体:方正静蕾简体
%%		 方正粗宋
\documentclass[a4paper,left=2.5cm,right=2.5cm,11pt]{article}

\usepackage[utf8]{inputenc}
\usepackage{fontspec}
\usepackage{cite}
\usepackage{xeCJK}
\usepackage{indentfirst}
\usepackage{titlesec}
\usepackage{longtable}
\usepackage{graphicx}
\usepackage{float}
\usepackage{rotating}
\usepackage{subfigure}
\usepackage{tabu}
\usepackage{amsmath}
\usepackage{setspace}
\usepackage{amsfonts}
\usepackage{appendix}
\usepackage{listings}
\usepackage{xcolor}
\usepackage{geometry}
\setcounter{secnumdepth}{4}
\usepackage{mhchem}
\usepackage{multirow}
\usepackage{extarrows}
\usepackage{hyperref}
\titleformat*{\section}{\LARGE}
\renewcommand\refname{参考文献}
\renewcommand{\abstractname}{\sihao \cjkfzcs 摘{  }要}
%\titleformat{\chapter}{\centering\bfseries\huge\wryh}{}{0.7em}{}{}
%\titleformat{\section}{\LARGE\bf}{\thesection}{1em}{}{}
\titleformat{\subsection}{\Large\bfseries}{\thesubsection}{1em}{}{}
\titleformat{\subsubsection}{\large\bfseries}{\thesubsubsection}{1em}{}{}
\renewcommand{\contentsname}{{\cjkfzcs \centerline{目{  } 录}}}
\setCJKfamilyfont{cjkhwxk}{STXingkai}
\setCJKfamilyfont{cjkfzcs}{STSongti-SC-Regular}
% \setCJKfamilyfont{cjkhwxk}{华文行楷}
% \setCJKfamilyfont{cjkfzcs}{方正粗宋简体}
\newcommand*{\cjkfzcs}{\CJKfamily{cjkfzcs}}
\newcommand*{\cjkhwxk}{\CJKfamily{cjkhwxk}}
\newfontfamily\wryh{Microsoft YaHei}
\newfontfamily\hwzs{STZhongsong}
\newfontfamily\hwst{STSong}
\newfontfamily\hwfs{STFangsong}
\newfontfamily\jljt{MicrosoftYaHei}
\newfontfamily\hwxk{STXingkai}
% \newfontfamily\hwzs{华文中宋}
% \newfontfamily\hwst{华文宋体}
% \newfontfamily\hwfs{华文仿宋}
% \newfontfamily\jljt{方正静蕾简体}
% \newfontfamily\hwxk{华文行楷}
\newcommand{\verylarge}{\fontsize{60pt}{\baselineskip}\selectfont}  
\newcommand{\chuhao}{\fontsize{44.9pt}{\baselineskip}\selectfont}  
\newcommand{\xiaochu}{\fontsize{38.5pt}{\baselineskip}\selectfont}  
\newcommand{\yihao}{\fontsize{27.8pt}{\baselineskip}\selectfont}  
\newcommand{\xiaoyi}{\fontsize{25.7pt}{\baselineskip}\selectfont}  
\newcommand{\erhao}{\fontsize{23.5pt}{\baselineskip}\selectfont}  
\newcommand{\xiaoerhao}{\fontsize{19.3pt}{\baselineskip}\selectfont} 
\newcommand{\sihao}{\fontsize{14pt}{\baselineskip}\selectfont}      % 字号设置  
\newcommand{\xiaosihao}{\fontsize{12pt}{\baselineskip}\selectfont}  % 字号设置  
\newcommand{\wuhao}{\fontsize{10.5pt}{\baselineskip}\selectfont}    % 字号设置  
\newcommand{\xiaowuhao}{\fontsize{9pt}{\baselineskip}\selectfont}   % 字号设置  
\newcommand{\liuhao}{\fontsize{7.875pt}{\baselineskip}\selectfont}  % 字号设置  
\newcommand{\qihao}{\fontsize{5.25pt}{\baselineskip}\selectfont}    % 字号设置 

\usepackage{diagbox}
\usepackage{multirow}
\boldmath
\XeTeXlinebreaklocale "zh"
\XeTeXlinebreakskip = 0pt plus 1pt minus 0.1pt
\definecolor{cred}{rgb}{0.8,0.8,0.8}
\definecolor{cgreen}{rgb}{0,0.3,0}
\definecolor{cpurple}{rgb}{0.5,0,0.35}
\definecolor{cdocblue}{rgb}{0,0,0.3}
\definecolor{cdark}{rgb}{0.95,1.0,1.0}
\lstset{
	language=[x86masm]Assembler,
	numbers=left,
	numberstyle=\tiny\color{black},
	showspaces=false,
	showstringspaces=false,
	basicstyle=\scriptsize,
	keywordstyle=\color{purple},
	commentstyle=\itshape\color{cgreen},
	stringstyle=\color{blue},
	frame=lines,
	% escapeinside=``,
	extendedchars=true, 
	xleftmargin=0em,
	xrightmargin=0em, 
	backgroundcolor=\color{cred},
	aboveskip=1em,
	breaklines=true,
	tabsize=4
} 

\newfontfamily{\consolas}{Consolas}
\newfontfamily{\monaco}{Monaco}
\setmonofont[Mapping={}]{Consolas}	%英文引号之类的正常显示,相当于设置英文字体
\setsansfont{Consolas} %设置英文字体 Monaco, Consolas,  Fantasque Sans Mono
\setmainfont{Times New Roman}

\setCJKmainfont{华文中宋}


\newcommand{\fic}[1]{\begin{figure}[H]
		\center
		\includegraphics[width=0.8\textwidth]{#1}
	\end{figure}}
	
\newcommand{\sizedfic}[2]{\begin{figure}[H]
		\center
		\includegraphics[width=#1\textwidth]{#2}
	\end{figure}}

\newcommand{\codefile}[1]{\lstinputlisting{#1}}

\newcommand{\interval}{\vspace{0.5em}}

% 改变段间隔
\setlength{\parskip}{0.2em}
\linespread{1.1}

\usepackage{lastpage}
\usepackage{fancyhdr}
\pagestyle{fancy}
\lhead{\space \qquad \space}
\chead{kvm环境的搭建 \qquad}
\rhead{\qquad\thepage/\pageref{LastPage}}
\begin{document}

\tableofcontents

\clearpage

\section{kvm环境的搭建}
	首先声明,这个仅在ubuntu16.04下配置过,配置日期为2016.12.16。

\subsection{搭建硬件环境}
	在x86\_64架构的INTEL处理器中,KVM必需的硬件虚拟化扩展为INTEL的虚拟化技术(INTEL VT)。
	首先处理器要在硬件上支持VT技术。只有在BIOS中将VT打开,才可以使用KVM。
\subsubsection{检查处理器是否支持VT技术}
	在linux系统中,可以通过/proc/cpuinfo文件中的CPU特性标志来查看CPU是否支持VT技术。
	如果CPU支持VT技术,那么文件中的flags就包含"vmx"。\par
	这里使用grep命令来查看/proc/cpuinfo中是否包含"vmx"。首先介绍一下grep命令。
	\begin{lstlisting}[numberstyle=\color{white}]
grep全称是Global Regular Expression Print,使用正则表达式搜索文本,并把匹配的行打印出来

grep [pattern] [file] 用于在file中查找符合pattern的文本,并打印出来

更多的grep信息,可以通过man grep查看。
	\end{lstlisting}

	使用如下命令就可以查看/proc/cpuinfo中是否包含了"vmx":
	\begin{lstlisting}
	grep "vmx" /proc/cpuinfo
	\end{lstlisting}
	
	如果在/proc/cpuinfo中包含了"vmx",说明CPU支持VT技术,否则另外设置。

\subsubsection{设置BIOS}
	如果CPU目前不支持VT技术,就需要设置BIOS中相应的选项。\par
	VT的选项一般在BIOS“Advanced”栏目的"CPU Configuration"中,它由"Intel Virtualization Technology"或"Intel VT"标识。
	找到该标识后,将其设为[Enabled]就可以了。

\subsection{安装KVM}
\subsubsection{下载KVM源代码}
	使用如下命令即可下载KVM源代码:
	\begin{lstlisting}
	git clone https://git.kernel.org/pub/scm/virt/kvm/kvm.git
	\end{lstlisting}

\subsubsection{配置KVM}
	在此我们使用make menuconfig对KVM进行配置。首先安装ncurses库:
	\begin{lstlisting}
	sudo apt-get install libncurses5-dev
	\end{lstlisting}

	然后在KVM文件中打开terminal,然后输入如下命令:
	\begin{lstlisting}
	make menuconfig
	\end{lstlisting}

	然后选择"Virtualization",如下所示:
	\sizedfic{0.7}{1.png}

	进入以后,进行如下图的配置:
	\sizedfic{0.7}{2.png}

	在配置完成后,将在KVM文件夹下生成.config文件。如果想确保KVM相关的配置正确,可以检查.config文件的相关配置项。查看命令如下:
	\begin{lstlisting}
	vi .config
	/CONFIG_HAVE_KVM
	\end{lstlisting}

	光标跳到对应的配置项后,查看是否如下面几个配置:
	\begin{lstlisting}
	CONFIG_HAVE_KVM=y
	CONFIG_HAVE_KVM_IROCHIP=y
	CONFIG_HAVE_KVM_IRQFD=y
	CONFIG_HAVE_KVM_IRQ_ROUTING=y
	CONFIG_HAVE_KVM_ENENTFD=y
	CONFIG_KVM_MMIO=y
	CONFIG_KVM_ASYNC_PF=y
	CONFIG_HAVE_KVM_MSI=y
	CONFIG_HAVE_KVM_CPU_RELAX_INTERCEPT=y
	CONFIG_KVM_VFIO=y
	CONFIG_KVM_GENERIC_DIRTYLOG_READ_PROTECT=y
	CONFIG_KVM_COMPAT=y
	CONFIG_HAVE_KVM_IRQ_BYPASS=y
	CONFIG_VIRTUALIZATION=y
	CONFIG_KVM=m
	CONFIG_KVM_INTEL=m
	\end{lstlisting}

\subsubsection{编译KVM}
	KVM的编译包括三个步骤:
	\begin{itemize}
		\item Kernel的编译。
		\item bzImage的编译。
		\item 内核模块的编译。
	\end{itemize}

	第一步是编译kernel。因为kernel包含了openssl的库,所以在编译之前,需要先安装openssl和相关的库,命令如下:
	\begin{lstlisting}
	sudo apt-get install openssl
	sudo apt-get install libssl-dev
	\end{lstlisting}
	
	接下来,就可以直接开始编译kernel,命令如下:
	\begin{lstlisting}
	make vmlinux
	\end{lstlisting}

	第二步是编译bzImage。命令如下:
	\begin{lstlisting}
	make bzImage
	\end{lstlisting}

	第三步是编译内核的模块。命令如下:
	\begin{lstlisting}
	make modules
	\end{lstlisting}

\subsubsection{安装KVM}
	KVM的安装包括两个步骤:
	\begin{itemize}
		\item module的安装。
		\item kernel与initramfs的安装。
	\end{itemize}

	首先安装module,命令如下:
	\begin{lstlisting}
	sudo make modules_install
	\end{lstlisting}

	然后安装kernel和initramfs,命令如下:
	\begin{lstlisting}
	sudo make install
	\end{lstlisting}

	最后重启系统,命令如下:
	\begin{lstlisting}
	reboot
	\end{lstlisting}

\subsubsection{加载kvm和kvm\_intel模块}
	使用如下命令即可加载kvm和kvm\_intel模块:
	\begin{lstlisting}
	modprobe kvm
	modprobe kvm_intel
	\end{lstlisting}

	执行lsmod指令,会列出所有已载入系统的模块。然后使用grep,可以查看是否存在kvm和kvm\_intel模块,命令如下所示:
	\begin{lstlisting}
	lsmod | grep kvm
	\end{lstlisting}

	如果terminal输出有kvm和kvm\_intel有关的信息,就说明kvm和kvm\_intel模块加载成功了。

\subsection{安装qemu-kvm}

\subsubsection{下载qemu-kvm源代码}
	使用如下命令即可下载qemu-kvm源代码:
	\begin{lstlisting}
	git clone https://git.kernel.org/pub/scm/virt/kvm/qemu-kvm.git
	\end{lstlisting}

\subsubsection{配置qemu-kvm}
	首先安装glib,命令如下:
	\begin{lstlisting}
	sudo apt-get install libglib2.0-dev
	\end{lstlisting}

	使用如下命令即可配置qemu-kvm:
	\begin{lstlisting}
	./configure
	\end{lstlisting}

	这里介绍一下./configure命令:
	\begin{lstlisting}[numberstyle = \color{white}]
	./configure会根据当前系统环境和指定参数生成makefile文件,为下一步的编译做准备。
	可以通过在configure后加上参数来对安装进行控制。
	比如: ./configure –prefix=/usr 
	意思是将该软件安装在/usr下面,执行文件就会安装在/usr/bin
	\end{lstlisting}

	随后,为了不让编译器把警告当作错误处理,应该在Makefile中任意一行添加如下代码:
	\begin{lstlisting}
	QEMU_CFLAGS += -w
	\end{lstlisting}

\subsubsection{编译qemu-kvm}
	使用如下命令即可编译qemu-kvm:
	\begin{lstlisting}
	make
	\end{lstlisting}

	编译的时候可能遇到如下错误:
	\begin{lstlisting}
	./qemu-options.texi:1526: unknown command `list'
  	./qemu-options.texi:1526: table requires an argument: the formatter for @item
  	./qemu-options.texi:1526: warning: @table has text but no @item
	\end{lstlisting}

	这个需要先对源代码打补丁,首先下载补丁,地址如下:
	\begin{lstlisting}
	http://patchwork.ozlabs.org/patch/222212/raw/
	\end{lstlisting}

	将补丁文件移动到qemu-kvm源代码根目录,然后使用如下命令:
	\begin{lstlisting}
	# fix.patch是补丁的文件名
	patch -p1 < fix.patch
	\end{lstlisting}

	然后再次编译,命令如下:
	\begin{lstlisting}
	make
	\end{lstlisting}

\subsubsection{安装qemu-kvm}
	使用如下命令即可安装qemu-kvm:
	\begin{lstlisting}
	sudo make install | tee make-install.log
	\end{lstlisting}

	这里介绍一下tee命令:
	\begin{lstlisting}[numberstyle = \color{white}]
	功能说明:读取标准输入的数据,并将其内容输出成文件。
	语法:tee [-ai][--help][--version][文件]
	参数:
		-a 附加到既有文件的后面,而非覆盖它。
		-i 忽略中断信号。
		--help 在线帮助。
		--version 显示版本信息。
	\end{lstlisting}

\subsection{安装客户机}
\subsubsection{创建镜像文件}
	安装客户机之前,我们需要创建一个镜像文件来存储客户机中的系统和文件。
	镜像文件将作为客户机的硬盘,将客户机的操作系统安装在其中。\par
	
	首先,使用如下命令行创建一个8GB大小的镜像文件ubuntu1604.img:
	\begin{lstlisting}
	dd if=/dev/zero of=ubuntu1604.img bs=1M count=8192
	\end{lstlisting}

	这里介绍一下dd命令:
	\begin{lstlisting}[numberstyle = \color{white}]
	功能说明:把指定的输入文件拷贝到指定的输出文件中。
	语法:dd [选项]
	参数:
		if=输入文件
		of=输出文件
		bs=bytes 同时设置读/写缓冲区的字节数
		count=blocks 只拷贝输入的blocks块
	\end{lstlisting}

\subsubsection{安装客户机}
	在联网的情况下,使用如下命令安装客户机:
	\begin{lstlisting}
	qemu-system-x86_64 -m 2048 -smp 4 -boot order=cd -hda ubuntu1604.img -cdrom ubuntu-16.04.iso  -vnc 127.0.0.1:2
	\end{lstlisting}

	这里介绍一下qemu-system-x86\_64的参数,如下所示:
	\begin{lstlisting}
	-m 2048是给客户机分配2048MB内存
	-smp 4是给客户机分配4个CPU
	-boot order=cd是指定系统的启动顺序为光驱、硬盘
	-hda ubuntu1604.img是分配给客户机的IDE硬盘
	-cdrom ubuntu-16.04.iso是分配给客户机的光驱
	 -vnc 127.0.0.1:2使用vnc方式显示客户机,端口为127.0.0.1:2
	\end{lstlisting}

\subsubsection{查看客户机}
	因为这里使用vnc方式显示客户机,所以我们需要先安装vncserver和vncviewer,命令如下:
	\begin{lstlisting}
	sudo apt-get install vncserver
	sudo apt-get install vncviewer
	\end{lstlisting}

	然后使用vncviewer查看客户机,命令如下:
	\begin{lstlisting}
	vncviewer 127.0.0.1:2
	\end{lstlisting}

\subsection{启动KVM客户机}
	安装好系统之后,就可以使用镜像文件来启动并登陆到自己安装的系统之中。
	在联网的情况下,使用如下命令即可启动一个KVM的客户机:
	\begin{lstlisting}
	qemu-system-x86_64 -m 2048 -smp 4 -hda ubuntu1604.img -vnc 127.0.0.1:2
	\end{lstlisting}

	使用如下命令可以查看KVM客户机:
	\begin{lstlisting}
	vncviewer 127.0.0.1:2
	\end{lstlisting}

\end{document}
